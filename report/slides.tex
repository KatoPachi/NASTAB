% Options for packages loaded elsewhere
\PassOptionsToPackage{unicode}{hyperref}
\PassOptionsToPackage{hyphens}{url}
%
\documentclass[
  ignorenonframetext,
]{beamer}
\usepackage{pgfpages}
\setbeamertemplate{caption}[numbered]
\setbeamertemplate{caption label separator}{: }
\setbeamercolor{caption name}{fg=normal text.fg}
\beamertemplatenavigationsymbolsempty
% Prevent slide breaks in the middle of a paragraph
\widowpenalties 1 10000
\raggedbottom
\setbeamertemplate{part page}{
  \centering
  \begin{beamercolorbox}[sep=16pt,center]{part title}
    \usebeamerfont{part title}\insertpart\par
  \end{beamercolorbox}
}
\setbeamertemplate{section page}{
  \centering
  \begin{beamercolorbox}[sep=12pt,center]{part title}
    \usebeamerfont{section title}\insertsection\par
  \end{beamercolorbox}
}
\setbeamertemplate{subsection page}{
  \centering
  \begin{beamercolorbox}[sep=8pt,center]{part title}
    \usebeamerfont{subsection title}\insertsubsection\par
  \end{beamercolorbox}
}
\AtBeginPart{
  \frame{\partpage}
}
\AtBeginSection{
  \ifbibliography
  \else
    \frame{\sectionpage}
  \fi
}
\AtBeginSubsection{
  \frame{\subsectionpage}
}
\usepackage{lmodern}
\usepackage{amsmath}
\usepackage{ifxetex,ifluatex}
\ifnum 0\ifxetex 1\fi\ifluatex 1\fi=0 % if pdftex
  \usepackage[T1]{fontenc}
  \usepackage[utf8]{inputenc}
  \usepackage{textcomp} % provide euro and other symbols
  \usepackage{amssymb}
\else % if luatex or xetex
  \usepackage{unicode-math}
  \defaultfontfeatures{Scale=MatchLowercase}
  \defaultfontfeatures[\rmfamily]{Ligatures=TeX,Scale=1}
\fi
% Use upquote if available, for straight quotes in verbatim environments
\IfFileExists{upquote.sty}{\usepackage{upquote}}{}
\IfFileExists{microtype.sty}{% use microtype if available
  \usepackage[]{microtype}
  \UseMicrotypeSet[protrusion]{basicmath} % disable protrusion for tt fonts
}{}
\makeatletter
\@ifundefined{KOMAClassName}{% if non-KOMA class
  \IfFileExists{parskip.sty}{%
    \usepackage{parskip}
  }{% else
    \setlength{\parindent}{0pt}
    \setlength{\parskip}{6pt plus 2pt minus 1pt}}
}{% if KOMA class
  \KOMAoptions{parskip=half}}
\makeatother
\usepackage{xcolor}
\IfFileExists{xurl.sty}{\usepackage{xurl}}{} % add URL line breaks if available
\IfFileExists{bookmark.sty}{\usepackage{bookmark}}{\usepackage{hyperref}}
\hypersetup{
  pdftitle={Charitable Giving, Tax Reform, and Political Trust},
  pdfauthor={Hiroki Kato\^{}1; Tsuyoshi Goto\^{}2; Yong-Rok Kim\^{}3},
  hidelinks,
  pdfcreator={LaTeX via pandoc}}
\urlstyle{same} % disable monospaced font for URLs
\newif\ifbibliography
\usepackage{graphicx}
\makeatletter
\def\maxwidth{\ifdim\Gin@nat@width>\linewidth\linewidth\else\Gin@nat@width\fi}
\def\maxheight{\ifdim\Gin@nat@height>\textheight\textheight\else\Gin@nat@height\fi}
\makeatother
% Scale images if necessary, so that they will not overflow the page
% margins by default, and it is still possible to overwrite the defaults
% using explicit options in \includegraphics[width, height, ...]{}
\setkeys{Gin}{width=\maxwidth,height=\maxheight,keepaspectratio}
% Set default figure placement to htbp
\makeatletter
\def\fps@figure{htbp}
\makeatother
\setlength{\emergencystretch}{3em} % prevent overfull lines
\providecommand{\tightlist}{%
  \setlength{\itemsep}{0pt}\setlength{\parskip}{0pt}}
\setcounter{secnumdepth}{-\maxdimen} % remove section numbering
\setbeamertemplate{navigation symbols}{}
\setbeamertemplate{footline}[page number]

\usepackage{bookmark}
\usepackage{booktabs}

\usepackage{xltxtra} 
\usepackage{zxjatype} 
\usepackage[ipa]{zxjafont} 
\ifluatex
  \usepackage{selnolig}  % disable illegal ligatures
\fi

\title{Charitable Giving, Tax Reform, and Political Trust}
\author{Hiroki Kato\(^1\) \and Tsuyoshi Goto\(^2\) \and Yong-Rok
Kim\(^3\)}
\date{2021/01/27}
\institute{\(^1\)Osaka University \and \(^2\)Chiba
University \and \(^3\)Kobe University}

\begin{document}
\frame{\titlepage}

\hypertarget{introduction}{%
\section{Introduction}\label{introduction}}

\begin{frame}{Background of South Korea Tax Reform}
\protect\hypertarget{background-of-south-korea-tax-reform}{}
To investigate the price effect, we use the 2014 tax reform in the South
Korea.

\begin{itemize}
\tightlist
\item
  Before 2014, tax deduction was adopted to subsidize charitable
  donation behavior.
\item
  After 2014, tax credit have been adopted.
\end{itemize}

The main difference is that tax credits reduce taxes directly, while tax
deductions indirectly lower the tax burden by decreasing the taxpayer's
marginal tax rate, which increases with gross income
\end{frame}

\hypertarget{data}{%
\section{Data}\label{data}}

\begin{frame}{National Survey of Tax and Benefit (NaSTaB)}
\protect\hypertarget{national-survey-of-tax-and-benefit-nastab}{}
\begin{itemize}
\tightlist
\item
  The Korea Institute of Taxation and Finance implements the financial
  panel survey to study the tax burden of households and the benefits
  that households receive from goverment.
\item
  The subjects of this survey are general household and household
  members living in 15 cities and provinces nationwide.
\item
  This survey is based on a face-to-face interview. If it is difficult
  for investigators to meet subjects, another family member answers on
  behalf of him.
\item
  Survey items: Annual taxable income (last year), charitable donations
  (last year), trust for politicians (5-Likert scale), and other
  covariates (age, education, gender etc.).
\item
  Survey period: 2008 \textasciitilde{} 2019

  \begin{itemize}
  \tightlist
  \item
    We use survey data after 2013 to focus on tax policy change in 2014.
  \end{itemize}
\end{itemize}
\end{frame}

\begin{frame}{Time Series of Chariable Giving}
\protect\hypertarget{time-series-of-chariable-giving}{}
\begin{figure}
\includegraphics[width=0.9\linewidth]{C:/Users/katoo/Desktop/NASTAB/_assets/SummaryOutcome} \caption{Proportion of Donors and Average Donations among Donors}\label{fig:unnamed-chunk-1}
\end{figure}
\end{frame}

\begin{frame}{Summary Statistics of Covariates}
\protect\hypertarget{summary-statistics-of-covariates}{}
\begin{table}

\caption{\label{tab:kableSummaryCovariate}Summary Statistics of Covariates}
\centering
\begin{tabular}[t]{lcccc}
\toprule
 & 2012 & 2013 & 2014 & 2015\\
\midrule
Female & 0.51 & 0.51 & 0.52 & 0.52\\
Age & 38.39 & 39.10 & 39.67 & 40.51\\
Annual taxable income & 1699.86 & 1764.04 & 1838.76 & 1872.54\\
University graduate & 0.28 & 0.28 & 0.29 & 0.30\\
High school graduate & 0.30 & 0.30 & 0.31 & 0.31\\
\#.Respondents & 14138 & 13984 & 13787 & 13524\\
\#.Households & 4756 & 4807 & 4819 & 4832\\
\bottomrule
\end{tabular}
\end{table}
\end{frame}

\begin{frame}{Summary Statistics of Covariates (Cont'd)}
\protect\hypertarget{summary-statistics-of-covariates-contd}{}
\begin{table}

\caption{\label{tab:kableSummaryCovariate2}Summary Statistics of Covariates (Continued)}
\centering
\begin{tabular}[t]{lccc}
\toprule
 & 2016 & 2017 & 2018\\
\midrule
Female & 0.52 & 0.52 & 0.52\\
Age & 41.07 & 41.89 & 42.55\\
Annual taxable income & 1906.91 & 1951.55 & 2039.47\\
University graduate & 0.31 & 0.33 & 0.34\\
High school graduate & 0.31 & 0.31 & 0.31\\
\#.Respondents & 13238 & 12963 & 12795\\
\#.Households & 4790 & 4770 & 4765\\
\bottomrule
\end{tabular}
\end{table}
\end{frame}

\begin{frame}{What is Giving Price?}
\protect\hypertarget{what-is-giving-price}{}
Consider allocation between private consumptions (\(x_i\)) and
charitable giving (\(g_i\)). Let \(y_i\) be pre-tax total income. Then,
the budget constraint is

\[
    x_i + g_i = y_i - T_i(y_i, g_i),
\]

where \(T_i\) is tax amount depending on the pre-tax income and
charitable giving.
\end{frame}

\begin{frame}{Determination of Tax Amount}
\protect\hypertarget{determination-of-tax-amount}{}
Tax deduction reduces taxable income by giving, that is,

\[
    T_i = \tau(y_i - g_i) \cdot (y_i - g_i),
\]

where \(\tau(\cdot)\) is the marginal income tax rate which is
determined by \(y_i - g_i\).

Tax credit reduces tax amount directly, that is,

\[
    T_i = \tau(y_i)\cdot y_i - m g_i,
\]

where \(m \in [0, 1]\) is the tax credit rate.
\end{frame}

\begin{frame}{Derive Giving Price}
\protect\hypertarget{derive-giving-price}{}
Under the tax deduction system, the budget constraint is

\[
    x_i + [1 - \tau(y_i - g_i)]g_i = [1 - \tau(y_i - g_i)] y_i.
\]

Thus, the giving price of tax deduction system is
\(p_i^{d} = 1 - \tau(y_i - g_i)\).

Under the tax credit system, the budget constraint is

\[
    x_i + (1 - m) g_i = [1 - \tau(y_i)] y_i.
\]

Thus, the giving price of tax credit system is \(p_i^c = 1 - m\).
\end{frame}

\begin{frame}{Construct Giving Price}
\protect\hypertarget{construct-giving-price}{}
In the South Korea, the tax policy about charitable giving drastically
changed in 2014.

\begin{itemize}
\tightlist
\item
  tax deduction (before 2014): \(\text{Price}_i = 1 - \tau(y_i - g_i)\)

  \begin{itemize}
  \tightlist
  \item
    the giving price is endogenous because people can manipulate
    \(\tau(y_i - g_i)\) using the charitable giving \(g_i\). Since this
    problem is caused by \emph{last} donations, we use the giving price
    applying to the \emph{first} donations (\textbf{first price}). The
    first price is calculate by \(\tau(y_i)\) where \(y_i\) is the
    annual taxable income reported in the NaSTaB.
  \end{itemize}
\item
  tax credit (after 2014): \(\text{Price}_i = 1 - m\)

  \begin{itemize}
  \tightlist
  \item
    In the South Korea, the tax credit rate determines exogeneity,
    \(m = 0.15\).
  \end{itemize}
\end{itemize}
\end{frame}

\begin{frame}{Income Distribution and Giving Price}
\protect\hypertarget{income-distribution-and-giving-price}{}
\begin{figure}
\includegraphics[width=0.9\linewidth]{C:/Users/katoo/Desktop/NASTAB/_assets/SummaryPriceChange} \caption{Income Distribution and Giving Price in 2013}\label{fig:unnamed-chunk-2}
\end{figure}
\end{frame}

\hypertarget{price-elasticity}{%
\section{Price Elasticity}\label{price-elasticity}}

\begin{frame}{Baseline Regressions}
\protect\hypertarget{baseline-regressions}{}
Our baseline regression equation is

\[
    \log(\text{Giving}_{ijt}) = 
    \alpha_i + \beta_1 \log(\text{Price}_{ijt}) + \delta X_{ijt} + \lambda_t + \epsilon_{ijt}.
\]

\begin{itemize}
\tightlist
\item
  \(\log(\text{Giving}_{ijt})\) is logarithm of individual \(i\)'s
  charitable giving in year \(t\).
\item
  \(\log(\text{Price}_{ijt})\) is logarithm of individual \(i\)'s giving
  price in year \(t\).
\item
  \(\beta_1\) represents the price elasticity of giving.
\item
  \(\alpha_i\) and \(\lambda_t\) are individual and time fixed effect,
  respectively.
\end{itemize}
\end{frame}

\begin{frame}{Result of Baseline Regressions}
\protect\hypertarget{result-of-baseline-regressions}{}
We found the \textbf{price effect} of giving (1\% price increase leads
to about 1.1\% giving decrease)

\begin{table}

\caption{\label{tab:kableEstimateElasticity}Baseline Regressions}
\centering
\fontsize{9}{11}\selectfont
\begin{tabular}[t]{lccccc}
\toprule
 & (1) & (2) & (3) & (4) & (5)\\
\midrule
ln(giving price) & -1.071*** & -1.264*** & -1.298*** & -1.117*** & -1.121***\\
 & (0.201) & (0.212) & (0.229) & (0.228) & (0.228)\\
Logged Income & Y & Y & Y & Y & Y\\
Age & N & Y & Y & Y & Y\\
Year X Educ & N & N & Y & Y & Y\\
Year X Gender & N & N & N & Y & Y\\
Resident Area & N & N & N & N & Y\\
N & 54213 & 54213 & 54211 & 54211 & 54211\\
\bottomrule
\end{tabular}
\end{table}
\end{frame}

\begin{frame}{Intensive Margin and Extensive Margin}
\protect\hypertarget{intensive-margin-and-extensive-margin}{}
Let \(D_{ijt}\) be a dummy variable taking 1 if individual \(i\) whose
resident area \(j\) in year \(t\) donate in year \(t\)

\begin{itemize}
\tightlist
\item
  Intensive margin: Estiamte \(\beta_1\) where outcome variable is
  \(\log(\text{Giving}_{ijt})\), using units with \(D_{ijt} = 1\).
\item
  Extensive margin: Estimate \(\beta_1\) where outcome variable is
  \(D_{ijt}\).

  \begin{itemize}
  \tightlist
  \item
    Extensive-margin price elasticity can be calculated by
    \(\beta_1/\bar{D}\) where \(\bar{D}\) is the sample mean of
    \(D_{ijt}\).
  \end{itemize}
\end{itemize}

Covariates in each column corresponds to a column in a previous slide.
\end{frame}

\begin{frame}{Intensive Margin and Extensive Margin: Result}
\protect\hypertarget{intensive-margin-and-extensive-margin-result}{}
\begin{table}

\caption{\label{tab:kableEstimateElasticityIntExt}Intensive-margin and Extensive-Margin Price Elasticity}
\centering
\fontsize{9}{11}\selectfont
\begin{tabular}[t]{lccccc}
\toprule
 & (1) & (2) & (3) & (4) & (5)\\
\midrule
\addlinespace[0.3em]
\multicolumn{6}{l}{\textbf{Intensive Margin}}\\
\hspace{1em}ln(giving price) & -0.593*** & -0.843*** & -1.022*** & -0.887*** & -0.891***\\
\hspace{1em} & (0.202) & (0.212) & (0.231) & (0.242) & (0.243)\\
\hspace{1em} & 11704 & 11704 & 11704 & 11704 & 11704\\
\addlinespace[0.3em]
\multicolumn{6}{l}{\textbf{Extensive Margin}}\\
\hspace{1em}ln(giving price) & -0.258*** & -0.290*** & -0.274*** & -0.238*** & -0.239***\\
\hspace{1em} & (0.046) & (0.048) & (0.052) & (0.052) & (0.052)\\
\hspace{1em}Elasticity & -1.699*** & -1.907*** & -1.807*** & -1.569*** & -1.573***\\
 & (0.301) & (0.316) & (0.341) & (0.341) & (0.341)\\
 & 54213 & 54213 & 54211 & 54211 & 54211\\
\bottomrule
\end{tabular}
\end{table}
\end{frame}

\begin{frame}{Robustness Check (1)}
\protect\hypertarget{robustness-check-1}{}
First potential concern: Income and donations are determined
simultaneously

\begin{itemize}
\tightlist
\item
  This causes both a change of giving price and a change of an amount of
  donations
\item
  Gruber and Saez (2002) provided that we should use
  \(\log(\text{Price}_{ijt}/\text{Price}_{ij(t-k)})\) as an insturment.
\item
  We estimated the model (5) in the previous slide, using the panel IV
  model for \(k = 1, 2, 3\).

  \begin{itemize}
  \tightlist
  \item
    Note that Alumnia (2020) took a strategy of \(k\)-difference model.
  \end{itemize}
\end{itemize}
\end{frame}

\begin{frame}{Robustness Check 1: Result}
\protect\hypertarget{robustness-check-1-result}{}
\begin{table}

\caption{\label{tab:kablePanelIVEstimateElasticity}Panel IV Regressions}
\centering
\begin{tabular}[t]{lccc}
\toprule
 & k = 1 & k = 2 & k = 3\\
\midrule
ln(giving price) & -1.279*** & -1.155*** & -1.150***\\
 & (0.478) & (0.414) & (0.369)\\
F-stat of IV & 10315.94 & 11506.64 & 11569.61\\
N & 51548 & 49217 & 46399\\
\bottomrule
\end{tabular}
\end{table}
\end{frame}

\begin{frame}{Robustness Check 1: Intensive and Extensive Margin}
\protect\hypertarget{robustness-check-1-intensive-and-extensive-margin}{}
\begin{table}

\caption{\label{tab:kablePanelIVEstimateElasticityIntExt}Intensive- and Extensive-Margin Price Elasticity: Panel IV}
\centering
\fontsize{9}{11}\selectfont
\begin{tabular}[t]{lccc}
\toprule
 & k = 1 & k = 2 & k = 3\\
\midrule
\addlinespace[0.3em]
\multicolumn{4}{l}{\textbf{Intensive Margin}}\\
\hspace{1em}ln(giving price) & -0.0004 & 0.0261 & -0.4378\\
\hspace{1em} & (0.5687) & (0.4410) & (0.3763)\\
\hspace{1em}F-stat of IV & 1679.78 & 2040.66 & 2419.05\\
\hspace{1em}N & 11332 & 10954 & 10451\\
\addlinespace[0.3em]
\multicolumn{4}{l}{\textbf{Extensive Margin}}\\
\hspace{1em}ln(giving price) & -0.3036*** & -0.2944*** & -0.2472***\\
\hspace{1em} & (0.1101) & (0.0934) & (0.0847)\\
\hspace{1em}Elasticity & -2.000*** & -1.939*** & -1.628***\\
\hspace{1em} & (0.725) & (0.615) & (0.558)\\
\hspace{1em}F-stat of IV & 10315.94 & 11506.64 & 11569.61\\
\hspace{1em}N & 51548 & 49217 & 46399\\
\bottomrule
\end{tabular}
\end{table}
\end{frame}

\begin{frame}{Robust Check 2}
\protect\hypertarget{robust-check-2}{}
Second potential concern: The effect of presidential transition on
donations

\begin{itemize}
\tightlist
\item
  The presidential transition is one of our major ommited factor to
  affect both political trust and charitable giving.
\item
  To shed light on this concern, we used data in 2013 and 2014
  (President was Park Geun-hye in both years), and estimated the model
  (5) in the previous slide, using the fixed effect model and the panel
  IV model for \(k = 1, 2, 3\).
\end{itemize}
\end{frame}

\begin{frame}{Robustness Check 2: Result}
\protect\hypertarget{robustness-check-2-result}{}
\begin{table}

\caption{\label{tab:kableShortEstimateElasticity}Results with data in 2013 and 2014}
\centering
\fontsize{9}{11}\selectfont
\begin{tabular}[t]{lcccc}
\toprule
\multicolumn{1}{c}{ } & \multicolumn{1}{c}{FE} & \multicolumn{3}{c}{Panel IV with FE} \\
\cmidrule(l{3pt}r{3pt}){2-2} \cmidrule(l{3pt}r{3pt}){3-5}
 &  & k = 1 & k = 2 & k = 3\\
\midrule
ln(giving price) & -1.466*** & -1.535*** & -1.683*** & -1.151***\\
 & (0.327) & (0.360) & (0.378) & (0.385)\\
F-stat of IV &  & 7420.10 & 4490.74 & 5034.58\\
N & 15134 & 13727 & 12902 & 12420\\
\bottomrule
\end{tabular}
\end{table}
\end{frame}

\begin{frame}{Robustness Check 2: Intensive and Extensive Margin}
\protect\hypertarget{robustness-check-2-intensive-and-extensive-margin}{}
\begin{table}

\caption{\label{tab:kableShortEstimateElasticityIntExt}Intensive- and Extensive-Margin Price Elasticity with Data in 2013 and 2014}
\centering
\fontsize{9}{11}\selectfont
\begin{tabular}[t]{lcccc}
\toprule
\multicolumn{1}{c}{ } & \multicolumn{1}{c}{FE} & \multicolumn{3}{c}{Panel IV with FE} \\
\cmidrule(l{3pt}r{3pt}){2-2} \cmidrule(l{3pt}r{3pt}){3-5}
 &  & k = 1 & k = 2 & k = 3\\
\midrule
\addlinespace[0.3em]
\multicolumn{5}{l}{\textbf{Intensive Margin}}\\
\hspace{1em}ln(giving price) & -0.759** & -0.736* & -0.819** & -0.543\\
\hspace{1em} & (0.344) & (0.418) & (0.404) & (0.371)\\
\hspace{1em}F-stat of IV &  & 1920.08 & 1762.03 & 1706.53\\
\hspace{1em}N & 2938 & 2746 & 2615 & 2512\\
\addlinespace[0.3em]
\multicolumn{5}{l}{\textbf{Extensive Margin}}\\
\hspace{1em}ln(giving price) & -0.332*** & -0.341*** & -0.380*** & -0.291***\\
\hspace{1em} & (0.074) & (0.083) & (0.085) & (0.089)\\
\hspace{1em}Elasticity & -2.186*** & -2.249*** & -2.504*** & -1.920***\\
\hspace{1em} & (0.488) & (0.547) & (0.559) & (0.583)\\
\hspace{1em}F-stat of IV &  & 7420.10 & 4490.74 & 5034.58\\
\hspace{1em}N & 15134 & 13727 & 12902 & 12420\\
\bottomrule
\end{tabular}
\end{table}
\end{frame}

\hypertarget{political-trust-and-price-elasticity}{%
\section{Political Trust and Price
Elasticity}\label{political-trust-and-price-elasticity}}

\begin{frame}{Estimation of Trust Index}
\protect\hypertarget{estimation-of-trust-index}{}
The trust for politicans is time-varying variable because it depends on
governments' policies. We make time-invarying trust index using the
fixed effect model.

\[
    \text{Trust}_{ijt} = \text{Trustid}_{ij} + c_j \cdot \lambda_t + \lambda_t + \epsilon_{ijt}.
\]

\begin{itemize}
\tightlist
\item
  \(\text{Trust}_{ijt}\): trust for politicians (5-Likert scale)
\item
  \(\text{Trustid}_i\): individual fixed effect (\textbf{Trust index})
\item
  \(c_j \cdot \lambda_t\) captures local governments' policies effect
\item
  \(\lambda_t\) captures the central government policies effect
\end{itemize}
\end{frame}

\begin{frame}{Five Subgroup Regression by Political Trust}
\protect\hypertarget{five-subgroup-regression-by-political-trust}{}
\end{frame}

\begin{frame}{Intensive and Extensive Margin}
\protect\hypertarget{intensive-and-extensive-margin}{}
\end{frame}

\begin{frame}{Three Subgroup Regression by Trust Index}
\protect\hypertarget{three-subgroup-regression-by-trust-index}{}
\begin{table}

\caption{\label{tab:kableEstimateElasticityByTrustGroup3}Subgroup Regressions}
\centering
\fontsize{9}{11}\selectfont
\begin{tabular}[t]{lccc}
\toprule
 & 1Q & 2Q & 3Q\\
\midrule
ln(giving price) & -0.496 & -1.635*** & -1.157***\\
 & (0.398) & (0.391) & (0.410)\\
N & 17421 & 16810 & 17075\\
\bottomrule
\end{tabular}
\end{table}
\end{frame}

\begin{frame}{Intensive and Extensive Margin}
\protect\hypertarget{intensive-and-extensive-margin-1}{}
\begin{table}

\caption{\label{tab:kableEstimateElasticityIntExtByTrustGroup3}Subgroup Regressions}
\centering
\fontsize{9}{11}\selectfont
\begin{tabular}[t]{lccc}
\toprule
 & 1Q & 2Q & 3Q\\
\midrule
\addlinespace[0.3em]
\multicolumn{4}{l}{\textbf{Intensive Margin}}\\
\hspace{1em}ln(giving price) & -0.997** & -0.980** & -0.208\\
\hspace{1em} & (0.408) & (0.398) & (0.450)\\
\hspace{1em}N & 3516 & 3959 & 3917\\
\addlinespace[0.3em]
\multicolumn{4}{l}{\textbf{Extensive Margin}}\\
\hspace{1em}ln(giving price) & -0.131 & -0.327*** & -0.244***\\
\hspace{1em} & (0.089) & (0.090) & (0.093)\\
\hspace{1em}Elasticity & -0.722 & -1.571*** & -1.088***\\
\hspace{1em} & (0.487) & (0.433) & (0.416)\\
\hspace{1em}N & 17421 & 16810 & 17075\\
\bottomrule
\end{tabular}
\end{table}
\end{frame}

\hypertarget{governement-efficient-and-price-elasticity}{%
\section{Governement Efficient and Price
Elasticity}\label{governement-efficient-and-price-elasticity}}

\begin{frame}{Government Efficiency}
\protect\hypertarget{government-efficiency}{}
From the 2015 survey, NaSTaB asks the current and ideal balance between
tax burden and welfare size.

These variables provide us to investigate the relationship between price
elasticity and govenrment's efficiency more directly.

Thus, we did same excercise, using the current balance bewteen tax
burden and welfare size.
\end{frame}

\begin{frame}{Construct Efficient Index}
\protect\hypertarget{construct-efficient-index}{}
Questionnaire of tax-welfare balance index is

\includegraphics[width=0.5\textwidth,height=\textheight]{_assets/BalanceQuestion.jpg}

To rule out government's policies, we use individual fixed effect as the
\textbf{tax-welare size index}
\end{frame}

\begin{frame}{Five Subgroup Regression by Political Trust}
\protect\hypertarget{five-subgroup-regression-by-political-trust-1}{}
\end{frame}

\begin{frame}{Intensive and Extensive Margin}
\protect\hypertarget{intensive-and-extensive-margin-2}{}
\end{frame}

\begin{frame}{Three Subgroup Regression by Trust Index}
\protect\hypertarget{three-subgroup-regression-by-trust-index-1}{}
\begin{table}

\caption{\label{tab:kableEstimateElasticityByEfficientGroup3}Subgroup Regressions}
\centering
\fontsize{9}{11}\selectfont
\begin{tabular}[t]{lccc}
\toprule
 & 1Q & 2Q & 3Q\\
\midrule
ln(giving price) & -1.321*** & -0.844** & -0.929**\\
 & (0.388) & (0.404) & (0.404)\\
N & 17119 & 16662 & 17525\\
\bottomrule
\end{tabular}
\end{table}
\end{frame}

\begin{frame}{Intensive and Extensive Margin}
\protect\hypertarget{intensive-and-extensive-margin-3}{}
\begin{table}

\caption{\label{tab:kableEstimateElasticityIntExtByEfficientGroup3}Subgroup Regressions}
\centering
\fontsize{9}{11}\selectfont
\begin{tabular}[t]{lccc}
\toprule
 & 1Q & 2Q & 3Q\\
\midrule
\addlinespace[0.3em]
\multicolumn{4}{l}{\textbf{Intensive Margin}}\\
\hspace{1em}ln(giving price) & -0.792** & -0.360 & -1.111**\\
\hspace{1em} & (0.383) & (0.423) & (0.497)\\
\hspace{1em}N & 3696 & 3591 & 4105\\
\addlinespace[0.3em]
\multicolumn{4}{l}{\textbf{Extensive Margin}}\\
\hspace{1em}ln(giving price) & -0.276*** & -0.225** & -0.174*\\
\hspace{1em} & (0.087) & (0.094) & (0.091)\\
\hspace{1em}Elasticity & -1.380*** & -1.115** & -0.787*\\
\hspace{1em} & (0.435) & (0.466) & (0.412)\\
\hspace{1em}N & 17119 & 16662 & 17525\\
\bottomrule
\end{tabular}
\end{table}
\end{frame}

\end{document}
