% Options for packages loaded elsewhere
\PassOptionsToPackage{unicode}{hyperref}
\PassOptionsToPackage{hyphens}{url}
%
\documentclass[
  ignorenonframetext,
  aspectratio=169,
]{beamer}
\usepackage{pgfpages}
\setbeamertemplate{caption}[numbered]
\setbeamertemplate{caption label separator}{: }
\setbeamertemplate{navigation symbols}{}
\setbeamertemplate{footline}[page number]
\setbeamertemplate{itemize item}{\small\raise0.5pt\hbox{$\bullet$}}
\setbeamertemplate{itemize subitem}{\tiny\raise1.5pt\hbox{$\bullet$}}
\setbeamertemplate{itemize subsubitem}{\tiny\raise1.5pt\hbox{$\bullet$}}
\setbeamercolor{caption name}{fg=normal text.fg}
\beamertemplatenavigationsymbolsempty
% Prevent slide breaks in the middle of a paragraph
\widowpenalties 1 10000
\raggedbottom
\setbeamertemplate{part page}{
  \centering
  \begin{beamercolorbox}[sep=16pt,center]{part title}
    \usebeamerfont{part title}\insertpart\par
  \end{beamercolorbox}
}
\setbeamertemplate{section page}{
  \centering
  \begin{beamercolorbox}[sep=12pt,center]{part title}
    \usebeamerfont{section title}\insertsection\par
  \end{beamercolorbox}
}
\setbeamertemplate{subsection page}{
  \centering
  \begin{beamercolorbox}[sep=8pt,center]{part title}
    \usebeamerfont{subsection title}\insertsubsection\par
  \end{beamercolorbox}
}
\AtBeginPart{
  \frame{\partpage}
}
\AtBeginSection{
  \ifbibliography
  \else
    \frame{\sectionpage}
  \fi
}
\AtBeginSubsection{
  \frame{\subsectionpage}
}
\usepackage{amsmath,amssymb}
\usepackage{lmodern}
\usepackage{iftex}
\ifPDFTeX
  \usepackage[T1]{fontenc}
  \usepackage[utf8]{inputenc}
  \usepackage{textcomp} % provide euro and other symbols
\else % if luatex or xetex
  \ifXeTeX
    \usepackage{xltxtra} 
    \usepackage{xeCJK}
    \setCJKmainfont{ipaexm.ttf}
    \setCJKsansfont{ipaexg.ttf}
    \setCJKmonofont{ipaexg.ttf}
  \fi
  \usepackage{unicode-math}
  \defaultfontfeatures{Scale=MatchLowercase}
  \defaultfontfeatures[\rmfamily]{Ligatures=TeX,Scale=1}
\fi
% Use upquote if available, for straight quotes in verbatim environments
\IfFileExists{upquote.sty}{\usepackage{upquote}}{}
\IfFileExists{microtype.sty}{% use microtype if available
  \usepackage[]{microtype}
  \UseMicrotypeSet[protrusion]{basicmath} % disable protrusion for tt fonts
}{}
\makeatletter
\@ifundefined{KOMAClassName}{% if non-KOMA class
  \IfFileExists{parskip.sty}{%
    \usepackage{parskip}
  }{% else
    \setlength{\parindent}{0pt}
    \setlength{\parskip}{6pt plus 2pt minus 1pt}}
}{% if KOMA class
  \KOMAoptions{parskip=half}}
\makeatother
\usepackage{xcolor}
\IfFileExists{xurl.sty}{\usepackage{xurl}}{} % add URL line breaks if available
\IfFileExists{bookmark.sty}{\usepackage{bookmark}}{\usepackage{hyperref}}
\hypersetup{
  pdftitle={Charitable Giving, Tax Reform, and Self-selection of Tax Report: Evidence from South Korea},
  hidelinks,
  pdfcreator={LaTeX via pandoc}}
\urlstyle{same} % disable monospaced font for URLs

\usepackage{setspace}
\usepackage{float}

\newif\ifbibliography
\usepackage{longtable,booktabs,array}
\usepackage{threeparttable, threeparttablex, multirow}
\usepackage{calc} % for calculating minipage widths
\usepackage{caption}
% Make caption package work with longtable
\makeatletter
\def\fnum@table{\tablename~\thetable}
\makeatother
\usepackage{graphicx}
\makeatletter
\def\maxwidth{\ifdim\Gin@nat@width>\linewidth\linewidth\else\Gin@nat@width\fi}
\def\maxheight{\ifdim\Gin@nat@height>\textheight\textheight\else\Gin@nat@height\fi}
\makeatother
% Scale images if necessary, so that they will not overflow the page
% margins by default, and it is still possible to overwrite the defaults
% using explicit options in \includegraphics[width, height, ...]{}
\setkeys{Gin}{width=\maxwidth,height=\maxheight,keepaspectratio}
% Set default figure placement to htbp
\makeatletter
\def\fps@figure{htbp}
\makeatother
\setlength{\emergencystretch}{3em} % prevent overfull lines
\providecommand{\tightlist}{%
  \setlength{\itemsep}{0pt}\setlength{\parskip}{0pt}}
\setcounter{secnumdepth}{-\maxdimen} % remove section numbering
\ifLuaTeX
  \usepackage{selnolig}  % disable illegal ligatures
\fi

\title{Charitable Giving, Tax Reform, and Self-selection of Tax Report: Evidence from South Korea}


      \author[shortname]{ Hiroki Kato \inst{1} \and  Tsuyoshi Goto \inst{2} \and  Yong-Rok Kim \inst{3} \and }
        \institute[shortinst]{ \inst{1} Osaka University \and  \inst{2} Chiba University \and  \inst{3} Kobe University \and }
  
\date{2021/08/20}


\begin{document}
\frame{\titlepage}

\begin{frame}{Evaluate Effect of Tax Incentive on Charitable Giving}
\protect\hypertarget{evaluate-effect-of-tax-incentive-on-charitable-giving}{}
\begin{itemize}
\tightlist
\item
  We estimate price effect on both \emph{declared} and \emph{non-declared} chariable giving, using the financial panel data survey including those who did not declare tax relief
\item
  To take price variation due to declaration into account, we estimate two price elasticities:

  \begin{itemize}
  \tightlist
  \item
    \emph{Applicable} price elasticity: we use giving price when tax payers declare tax relief regardless of actual declaration
  \item
    \emph{Effective} price elasticity: we construct giving price based on information of declaration
  \item
    By comparing two price elasticities, we can infer price effect through declration
  \end{itemize}
\item
  \textbf{Result}: Both price elasticities are around -1, which implies that the price effect through declaration is not so large.
\end{itemize}
\end{frame}

\begin{frame}{Positioning of Our Research}
\protect\hypertarget{positioning-of-our-research}{}
Extant research mainly uses the tax return data

\begin{itemize}
\tightlist
\item
  This data consists on those who declared tax relief
\item
  Thus, extant research estimates the applicable price elasticity of declared donations
\end{itemize}

However, non-declared tax payers also decide an amount of donation and whether to declare tax relief based on tax incentive.

\begin{itemize}
\tightlist
\item
  Our research uses the financial panel data survey including those who did not declare tax relief
\item
  Thus, we can estimate the price effect, taking this fact into consideration
\end{itemize}
\end{frame}

\begin{frame}{2014 tax reform in South Korea}
\protect\hypertarget{tax-reform-in-south-korea}{}
Our major price variation comes from the 2014 tax reform, which has changed from the tax deduction system to the tax credit system

\begin{itemize}
\tightlist
\item
  Consider allocation problem b/w private consumption (\(x_{it}\)) and giving (\(g_{it}\)).
\item
  The budget constraint is \(x_{it} + g_{it} = y_{it} - T(y_{it}, g_{it})\) where \(y_{it}\) is pre-tax total income, and \(T(y_{it}, g_{it})\) is tax amount.
\item
  Let \(R_{it}\) be a dummy of declaration of tax relief, and let \(\tau(\cdot)\) be the income tax rate.
\end{itemize}
\end{frame}

\begin{frame}{2014 tax reform in South Korea (Cont'd)}
\protect\hypertarget{tax-reform-in-south-korea-contd}{}
\textbf{Tax deduction system (until 2013)}

\[T(y_{it}, g_{it}) = \tau(y_{it} - R_{it} g_{it}) (y_{it} - R_{it} g_{it})\]

\begin{itemize}
\tightlist
\item
  In 2012 and 2013, the system of \(\tau(\cdot)\) is same.
\item
  The logged relative giving price is \(R_{it} \ln(1 - \tau(y_{it} - g_{it})) = R_{it} \ln p^d_{it}\).
\end{itemize}

\textbf{Tax credit system (from 2014)}

\[T(y_{it}, g_{it}) = \tau(y_{it}) \cdot y_{it} - R_{it} m g_{it}\]

\begin{itemize}
\tightlist
\item
  \(m = 0.15\)
\item
  The logged relative giving price is \(R_{it} \ln(1 - 0.15) = R_{it} \ln 0.85\).
\end{itemize}
\end{frame}

\begin{frame}{About NaSTaB}
\protect\hypertarget{about-nastab}{}
An annual financial panel survey implemented by The Korea Institute of Taxation and Finance

\begin{itemize}
\tightlist
\item
  The subjects of this survey are general household and household members living in 15 cities and provinces nationwide.
\item
  We use data from 2013 to 2019 to focus on the 2014 tax reform.

  \begin{itemize}
  \tightlist
  \item
    the giving price before 2014 was changed frequently and incorporating the data before 2012 captures the effects of another tax reform than the reform in 2014.
  \item
    NaSTaB asks the amount of donation and the annual labor income last year.
  \end{itemize}
\end{itemize}
\end{frame}

\begin{frame}{Proportion of donors is slightly decreased just after tax reform}
\protect\hypertarget{proportion-of-donors-is-slightly-decreased-just-after-tax-reform}{}
\begin{center}\includegraphics[width=0.85\linewidth]{C:/Users/katoo/Desktop/NASTAB/report/slide_files/figure-beamer/SummaryOutcome-1} \end{center}
\end{frame}

\begin{frame}{2014 tax reform made increasing price group and decreasing price group}
\protect\hypertarget{tax-reform-made-increasing-price-group-and-decreasing-price-group}{}
\begin{center}\includegraphics[width=0.85\linewidth]{C:/Users/katoo/Desktop/NASTAB/report/slide_files/figure-beamer/SummaryPriceChange-1} \end{center}
\end{frame}

\begin{frame}{Fixed Effect Model}
\protect\hypertarget{fixed-effect-model}{}
\textbf{Intensive-margin elasticity}: how much do donors additionally donate reacting to the marginal increase of giving price?

\begin{equation}
    \ln g_{it} = \varepsilon^{int}_p R_{it} \ln p_{it} + \varepsilon^{int}_y \ln y_{it} 
    + X_{it}\beta +\mu_i +\iota_t +u_{it}. \label{eq:intensive}
\end{equation}

\textbf{Extensive-margin elasticity}: how much does the probability to donate change reacting to marginal increase of giving price?

\begin{equation}
D_{it} =  \delta R_{it} \ln p_{it} +\gamma \ln y_{it} + X_{it}\beta +\mu_i  +\iota_t +v_{it}. \label{eq:extensive}
\end{equation}

\begin{itemize}
\tightlist
\item
  Since we use the linear probability model, the estimated coefficient \(\delta\) represents \(\hat{\delta} = \frac{\partial D_{it}}{\partial p_{it}} p_{it}\).
\item
  the implied extensive-margin price are calculated by \(\hat{\delta}/\bar{D}\) where \(\bar{D}\) is sample average of outcome variable \(D_{it}\).
\end{itemize}
\end{frame}

\begin{frame}{Applicable Price Elasticity (ITT Approach)}
\protect\hypertarget{applicable-price-elasticity-itt-approach}{}
\begin{itemize}
\tightlist
\item
  We use giving price when tax payers declare tax relief regardless of actual declaration
\item
  Non-declrared tax payers are treated as if they have declrared tax relief
\item
  \(R_{it} = 1\) for any \(i\) and \(t\) in equation \eqref{eq:intensive} and \eqref{eq:extensive}
\end{itemize}
\end{frame}

\begin{frame}{Effective Price Elasticity (IV approach)}
\protect\hypertarget{effective-price-elasticity-iv-approach}{}
\begin{itemize}
\tightlist
\item
  Declaration affects both giving price and charitable giving
\item
  We took the panel IV model, using the employed dummy as instrument.

  \begin{itemize}
  \tightlist
  \item
    There is a difference of declaration cost of tax relief since self-employed workers have to retain the certificate until they submit tax return although wage earners can submit the certificate at any time.
  \end{itemize}
\item
  First, we estimate the following model:
\end{itemize}

\begin{align}
  R_{it}
  = \alpha_{1i} + \lambda \text{Employed}_{it} + X_{it} \beta_1
  + \mu_{i1} + \iota_{t1} + \eta_{it} \label{eq:stage1}
\end{align}

\begin{itemize}
\tightlist
\item
  Second, we obtain the fitted value of \(R_{it}\) (denoted by \(\hat{R}_{it}\)) and replace \(R_{it}\) with \(\hat{R}_{it}\).
\end{itemize}
\end{frame}

\begin{frame}{Result of Applicable Peice Elasticitiy}
\protect\hypertarget{result-of-applicable-peice-elasticitiy}{}
Overall elasticity is that we do not distinguish intensive and extensive margin

\begin{table}
\centering\begingroup\fontsize{8}{10}\selectfont

\begin{tabular}{lccc}
\toprule
 & Overall & Intensive & Extensive\\
\midrule
$\hat{\varepsilon}_p^{int}$ & -1.241*** & -0.904*** & \\
 & (0.227) & (0.249) & \\
$\hat{\delta}$ &  &  & -0.267***\\
 &  &  & (0.051)\\
$\hat{\delta}/\bar{D}$ &  &  & -1.221***\\
 &  &  & (0.235)\\
Individual FE & Y & Y & Y\\
Time FE & Y & Y & Y\\
Age & Y & Y & Y\\
Year x Education & Y & Y & Y\\
Year x Gender & Y & Y & Y\\
Year x Resident Area & Y & Y & Y\\
N & 53267 & 11637 & 53267\\
Adjusted R-squared & 0.530 & 0.678 & 0.462\\
\bottomrule
\end{tabular}
\endgroup{}
\end{table}
\end{frame}

\begin{frame}{Result of Effective Price Elasticity}
\protect\hypertarget{result-of-effective-price-elasticity}{}
Overall effective price elasticity is slightly more elastic than applicable one. Extensive-margin effective price elasticity is slightly less elastic than applicable one.

\begin{table}
\centering\begingroup\fontsize{8}{10}\selectfont

\begin{tabular}{lccc}
\toprule
 & Overall & Intensive & Extensive\\
\midrule
$\hat{\varepsilon}_p^{int}$ & -1.603*** & -0.987*** & \\
 & (0.466) & (0.342) & \\
$\hat{\delta}$ &  &  & -0.319***\\
 &  &  & (0.110)\\
$\hat{\delta}/\bar{D}$ &  &  & -0.926***\\
 &  &  & (0.320)\\
Individual and time FE & Y & Y & Y\\
log(income) & Y & Y & Y\\
Age & Y & Y & Y\\
Year x Education & Y & Y & Y\\
Year x Gender & Y & Y & Y\\
Year x Resident Area & Y & Y & Y\\
Year x Dummy of industry & Y & Y & Y\\
N & 16946 & 5840 & 16946\\
Adjusted R-squared & 0.514 & 0.697 & 0.428\\
\bottomrule
\end{tabular}
\endgroup{}
\end{table}
\end{frame}

\begin{frame}{Conclusions}
\protect\hypertarget{conclusions}{}
Main message

\begin{itemize}
\tightlist
\item
  Both ITT approach and IV approach show that the giving price elasticity in Korea is around -1.
\item
  It implies that the effect from the declaration cost, which has been ignored, is not so large in South Korea.
\end{itemize}

Some robustness

\begin{itemize}
\tightlist
\item
  Alothough we focus on price variation coming from tax reform and declaration of tax relief, price variation is also caused by a manipulation of giving and income.
\item
  We take other empirical methodologies to control these problemes, and obtain similar results.
\end{itemize}
\end{frame}

\end{document}
