\documentclass[dvipdfmx,10pt]{beamer}
\usepackage[utf8]{inputenc}
\usepackage[T1]{fontenc}
\usepackage{lmodern,bm,otf}
\usepackage{graphicx}
\usetheme{default}
\begin{document}
	\author{加藤大貴
		後藤剛志
		金栄禄}
	\title{Charitable Giving, Tax Reform, and Self-selection of Tax Report:
		Evidence from South Korea}
	%\subtitle{}
	%\logo{}
	\institute{大阪大学, 千葉大学, 関西大学}
	%\date{}
	%\subject{}
	%\setbeamercovered{transparent}
	%\setbeamertemplate{navigation symbols}{}
	\begin{frame}[plain]
		\maketitle
	\end{frame}
	
	\begin{frame}{Introduction}
		\protect\hypertarget{introduction}{}
		\begin{itemize}
			\item 世界の多くの国で寄付に対する税制上の優遇がなされている。
			\item 税制優遇が厚生に与える影響について、寄付の価格弾力性が厚生評価のための重要な指標であると考えられている。(e.g.~Saez, 2004)
			\begin{itemize}
				\item 直感的に言えば寄付の価格弾力性が-1を下回ると、\$1の税制優遇が\$1以上の寄付を生み出すこととなる。
				\item 「寄付の税制優遇で寄付の価格が下がり、寄付が促進されるだろう」
			\end{itemize}
			\item 既存研究の多くは課税申告データを用いて寄付の価格弾力性を推定してきた。(e.g.~Almunia et al., 2020, Auten et al., 2002, and so on)
		\end{itemize}
	\end{frame}
	
	\begin{frame}{Introduction}
		\protect\hypertarget{introduction-1}{}
		\begin{itemize}
			\item しかし、課税申告データは申告された寄付のみしか記録していない。
			\begin{itemize}
				\item この論文の扱う第1の問題: 実際の寄付量と申告された寄付量は異なるものであり、推定バイアスが起きる(Fack and Landais (2016); Gillitzer and Skov (2018))。
				\item この問題に対応するため、課税申告データではなく、韓国のパネルデータを使って分析を行った。
			\end{itemize}
			\item 納税者は寄付を申告するかどうか自ら選択でき、申告する際には寄付証明などの提出などのコストがかかる。			
			\begin{itemize}
				\item この論文の扱う第2の問題: 寄付の申告コストが無視されると推定バイアスが起きる。\\
				$\leftarrow$寄付の申告をしなければ税制優遇は受けられない
				\item この論文では操作変数法(IV)とControl Function (CF)アプローチをつかってこの問題に対処。
			\end{itemize}
			\item 差の差法(DID)をメインの識別戦略として用いて、寄付の価格弾力性の大きさについて韓国のデータを用いて調べた。
		\end{itemize}
	\end{frame}
	
	\begin{frame}{Introduction}
		\protect\hypertarget{introduction-2}{}
		分析の結果
		\begin{enumerate}
			\item ベースラインの結果では、寄付の価格弾力性がIntensive Marginsで-1.4以下、Extensive Marginsで-1.7以下となった。
			\begin{itemize}
				\item 寄付の申告コストを考慮しない結果ではIntensive Marginsで-0.9となった。
			\end{itemize}
			\item 寄付の申告者に絞った分析結果では寄付の価格弾力性はIntensive Marginsで-1.2\(\sim\)-1.6となった。
		\end{enumerate}
		既存研究の多くではIntensive Marginsの寄付の価格弾力性が約-1であり、これよりも概ね弾力的な推定結果が得られた。
		\begin{enumerate}
			\setcounter{enumi}{2}
			\item (論文には未掲載) 推定値をもとに社会厚生を分析すると韓国では寄付の税制優遇を拡充することで厚生が増大する可能性が高いことがわかった。
			\item (論文には未掲載) 寄付の申告コストをなくすことで寄付の量が概ね7$\sim$20\%増加することが推定からわかった。
		\end{enumerate}
	\end{frame}
	
	\begin{frame}{2014年の韓国での税制改正}
		\protect\hypertarget{tax-reform-in-south-korea}{}
		韓国では所得税納税者は寄付に対して税制上の優遇を受けることができる。
		\begin{itemize}
			\item 優遇を受けるためには、寄付の証明書を提出して寄付を申告する必要がある。
			\item 給与所得者は所得税を源泉徴収で納税し、寄付の申告は会社で行う。
			\begin{itemize}
				\item 給与所得者は証明書の提出は随時行うことができる。
			\end{itemize}
			\item 非給与所得者は所得税を確定申告で行い、寄付の申告は国税庁を通じて行う。			
			\begin{itemize}
				\item 非給与所得者は確定申告時まで証明書を保存しておく必要がある。
			\end{itemize}
		\end{itemize}
	\end{frame}
	
	\begin{frame}{2014年の韓国での税制改正}
		\protect\hypertarget{tax-reform-in-south-korea-1}{}
		この分析での識別時の主な価格バリエーションは2014年の税制改正によるものである。		
		\begin{itemize}
			\item 2014年より前は寄付の優遇策として所得控除が用いられていた。			
			\begin{itemize}
				\item 所得控除では所得と寄付額によって所得税率が変化するので、(私的財消費と比べたときの)寄付価格は所得や寄付額に依存。
			\end{itemize}
			\item 2014年以後は税額控除が用いられるようになった。
			\begin{itemize}
				\item 税額控除の率は所得などにかかわらず一律15\%
				\item これはつまり、私的財消費\$1と比べた相対的な寄付価格が\$0.85となることを意味\\
				(以下ではこれを「寄付価格」と呼ぶ)
			\end{itemize}
		\end{itemize}
	\end{frame}
	
	\begin{frame}{2014年の韓国での税制改正}
		\protect\hypertarget{tax-reform-in-south-korea-2}{}
		Model
		\begin{itemize}
			\item 私的財消費(\(x_{i}\))と寄付(\(g_{i}\))を考える。
			\item \(y_{i}\)を課税前所得、\(R_{i}\)を寄付申告を示すダミー、\(T(y_i)\)と\(T(y_{i}, g_{i})\)を個人$i$が寄付申告をしなかったときの課税額と申告したときの課税額、$K$を寄付申告のコストだとする。
			\item 予算制約は以下のように表せる。
		\end{itemize}
		
		\[x_{i} + g_{i} = y_{i} - R_iK- R_iT(y_{i}, g_{i})-(1-R_i)T(y_i)\]
		
		
		\begin{itemize}
			\item 個人$i$が寄付を申告するかは、申告時の課税額の減少が申告コストを上回るかによる。
			\begin{align}
				R_i=\begin{cases}
					1 \text{ if }T(y_i, g_i) - T(y_i)>K\\
					0 \text{ if }T(y_i, g_i) - T(y_i)\le K.
				\end{cases}
			\end{align}
		\end{itemize}
	\end{frame}
	
	\begin{frame}{2014年の韓国での税制改正}
		\textbf{所得控除での寄付申告時の税額(2013年まで)}
		\[T(y_{i}, g_{i}) = T(y_{i} - g_{i})\]
		
		\begin{itemize}
			\item 対数を取ったときの寄付価格は\(R_{i}\ln(1 - T'(y_{i} - g_{i}))\)と表せる。
			\item 2012年と2013年の韓国の所得税率は同じだが、2011年以前は異なる所得税率であった。
		\end{itemize}
		
		\textbf{税額控除での寄付申告時の税額(2014年から)}
		\[T(y_{i}, g_{i}) = T(y_{i}) - m g_{i}\]
		
		\begin{itemize}
			\item \(m\)は税額控除率で\(m = 0.15\)と設定されている。
			\item 対数をとったときの寄付価格は\(R_{i}\ln(1 - 0.15) = R_{i}\ln 0.85\)と表せる。
		\end{itemize}
		
		Note: 寄付申告を行わなかった者の直面する寄付価格は\(\ln1 =0\)となる。
	\end{frame}
	
	\begin{frame}{内生性の問題}
		\begin{enumerate}
			\item 課税申告データの使用は申告された寄付額のみを捉えることとなる。
			\begin{itemize}
				\item 未申告の寄付に寄付の税制優遇は無意味のはず(=推定バイアスに繋がる)
				\item 先行研究ではサーベイデータでこの問題に対処 (e.g.~Rehavi and Shack, 2013).
				\item \textbf{本稿でも韓国のサーベイデータでこの問題に対処。}
			\end{itemize}
			\item 申告コストが無視されると推定バイアスが発生する可能性。
			\begin{itemize}
				\item 申告コストが安くて寄付価格が安くなる人だけを捉えて推定する可能性(Self-selectionバイアス)
				\item 知る限りAlmunia et al.~(2020)のみが申告コストを考慮しているが、課税申告データを使用してしまっている。
				\item\textbf{本稿では給与所得者と非給与所得者の違いを操作変数(IV)として採用し、申告コストを考慮。}
			\end{itemize}
		\end{enumerate}
	\end{frame}
	
	\begin{frame}{Data}
		使用するデータは韓国のNational Survey of Tax and Benefit(NaSTab)のデータ
		\begin{itemize}
			\item このデータの対象は韓国全体の15の市と地域の一般世帯とその構成員。
			\item データは主に対面のインタビュー形式で回答されたもの。
			\item データサンプルは韓国の一般的な社会構成が代表されるように構成されている。
			\item 本稿では所得や資産がないと思われる23歳以下のサンプルは除いて分析している。
			\item データとして
		\end{itemize}
	\end{frame}

\begin{frame}{Data}
	\begin{figure}
		\centering
		\includegraphics[width=0.8\linewidth]{Fig_Average_donation}
		\caption{Proportion of Donors and Average Donations among Donors}
		\label{fig:1}
	\end{figure}
\small
	\begin{itemize}
		\item About 20$\sim$30\% of people make a donation.
		\item The average amount of donations among donors is about 1.5 million KRW.
	\end{itemize}
\end{frame}

\begin{frame}{Data}
	\begin{figure}
		\centering
		\includegraphics[width=0.7\linewidth]{Fig_Income_distribution}
		\caption{Income Distribution and Relative Giving Price in 2013}
		\label{fig:2}
	\end{figure}
\small
	In 2014, relative giving price
	\begin{itemize}
		\item decreases for people whose income is less than 12 million KRW.
		\item is the same for people whose income is between 12 million KRW and 46 million KRW.
		\item increase for people whose income is more than 46 million KRW.
	\end{itemize}
\end{frame}

\begin{frame}{Data}
	\begin{figure}
		\centering
		\includegraphics[width=0.7\linewidth]{Fig_Diff}
		\caption{Average Logged Giving in Three Income Groups}
		\label{fig:3}
	\end{figure}
\small
	Compared to 2012 and 2013, the amount of charitable giving after 2014
	\begin{itemize}
		\item increases for people whose income is less than 12 million KRW.
		\item relatively increases for people whose income is between 12 million KRW and 46 million KRW.
		\item decreases for people whose income is more than 46 million KRW.
	\end{itemize}
\end{frame}

\begin{frame}{Data}
	\begin{figure}
		\centering
		\includegraphics[width=0.7\linewidth]{Fig_Declaration}
		\caption{Share of Declaration of Tax Relief}
		\label{fig:4}
	\end{figure}
\small
	\begin{itemize}
		\item Wage earners are more likely to declare their charitable giving to receive tax relief.\\
		$\to$ This reflects the difference of the declaration cost.
	\end{itemize}
\end{frame}

\begin{frame}{Data}
	\begin{table}
		\centering
		\includegraphics[width=0.9\linewidth]{Tab_Stat}
		\caption{Summary Statistics}
		\label{tab:1}
	\end{table}
\end{frame}

\begin{frame}{Statistical Model}
\begin{itemize}
	\item Main identification source is tax reform in 2014. (DID-like strategy)
	\begin{itemize}
		\item Below 12 million KRW: Giving price decreases.
		\item Btw 12 and 46 million KRW: Giving price is the same.
		\item Above 46 million KRW: Giving price increases.
	\end{itemize}
	\item To capture the difference of declaration cost, we use a dummy to show whether a subject is a wage earner or not as IV.
\end{itemize}
\end{frame}

\begin{frame}{Statistical Model}
	\begin{itemize}
		\item We estimate the following two-way fixed effect model:
		\begin{align}
			\ln g_{it} = \varepsilon_pR_{it} \ln p_{it}(y_{it}, g_{it}) + \varepsilon_y \ln y_{it} + \bm{X}_{it}\bm{\beta} + \mu_i + \iota_t + u_{it}\tag{5}
		\end{align}
	where $\mu_i, \iota_t$ and $ u_{it}$ are an individual fixed effect, a year fixed effect, and a error term, respectively. 
	\item $\bm{X}_{it}$ is a vector of covariates including square of age, industry dummy, and area dummy.
	\item In the literature, estimations in terms of \textbf{intensive} and \textbf{extensive} margins are common.
	\begin{itemize}
		\item \textbf{Intensive margins}: estimate (5) only for $g_{it}>0$.
		\item \textbf{Extensive margins}: estimate (5) but dependent variable is $1[g_{it}>0]$.
	\end{itemize}
	$\to$ Following the literature, we estimate both of them.
	\end{itemize}
\end{frame}

\begin{frame}{Statistical Model: Endogeneity of $p_{it}$}
	\begin{itemize}
		\item The giving price (compared to private good) is
		\begin{align}
			p_{it}(y_{it}, g_{it})=\begin{cases}
			1-T'_t(y_{it}-g_{it})&\text{ if }t<2014\\
			0.85&\text{ if }t\ge2014
			\end{cases}
			\tag{6'}.
		\end{align}
		\item Since the giving price is endogenous to the amount of giving before 2014, we use ``\textbf{the first-price of giving}", which is defined as 
		\begin{align}
			p_{it}^f(y_{it})=p_{it}(y_{it},0)\notag
		\end{align}
		 instead of $p_{it}(y_{it},0)$ in the estimation.
		\item $p_{it}(y_{it},0)$ is called as ``the last-price of giving".
	\end{itemize}
\end{frame}

\begin{frame}{Statistical Model: Endogeneity of $R_{it}$}
	\begin{itemize}
		\item Since donors can choose whether they declare charitable giving or not, declaration, $R_{it}$, is endogenous.
		\item To overcome this, we estimate 
		\begin{align}
			\ln g_{it} = \varepsilon_pR_{it} \ln p_{it}^f(y_{it}) + \varepsilon_y \ln y_{it} + \bm{X}_{it}\bm{\beta} + \mu_i + \iota_t + u_{it}\tag{7}
		\end{align}
	 	where $R_{it} \ln p_{it}^f(y_{it})$ is instrumented by $WageEarner_{it}\times \ln p^f_{it}(y_{it})$. 
	 	\begin{itemize}
	 		\item This estimation is based on 2SLS.
	 	\end{itemize}
		\item $WageEarner_{it}$ is a dummy to show whether a subject is a wage earner or not.
		\begin{itemize}
			\item $WageEarner_{it}$ should not correlate to $u_{it}$ when we control incomes and industry dummies.
		\end{itemize}
	\end{itemize}
\end{frame}

\begin{frame}{Statistical Model: Endogeneity of $R_{it}$}
	\begin{itemize}
		\item In alternative models, we use propensity score to declare $P(Z_{it})$ as an instrument.
		\begin{itemize}
			\item This estimation method is called ``control function (CF)" approach.
			\item The propensity score is estimated by a probit model
			\begin{align}
				R_{it}=1[\delta_0+Z_{it}\delta_1+u_{it1}>0]\tag{8},
			\end{align}
			where $Z_{it}\equiv \{WageEarner_{it}, \ln p_{it}^f(y_{it}), \ln y_{it}, \bm{X}_{it}\}$.
			\item The propensity score is $P(Z_{it})=\Phi(\hat{\delta}_0+Z_{it}\hat{\delta}_1)$.
			\item We consider two cases:\\
			 \ajMaru1 pooled probit model $(\delta_0, \delta_1)$ is constant.\\
			 \ajMaru2 separated probit model $(\delta_0, \delta_1)$ can vary by year.
		\end{itemize}
		\item In addition, we also estimate the following by OLS:
		\begin{align}
			\ln g_{it} = \varepsilon_pP(Z_{it})\ln p_{it}^f(y_{it}) + \varepsilon_y \ln y_{it} + \bm{X}_{it}\bm{\beta} + \mu_i + \iota_t + u_{it}\tag{9}
		\end{align}
	\end{itemize}
\end{frame}

\begin{frame}{Results: Intensive Margins}
	\begin{table}
		\centering
		\includegraphics[width=0.9\linewidth]{Tab_res_1}
		\caption{First-Price Elasticities (Intensive Margins)}
		\label{tab:2}
	\end{table}
The estimated giving price elasticity in terms of intensive margins is about -1.5.
\end{frame}

\begin{frame}{Results: Extensive Margins}
	\begin{table}
		\centering
		\includegraphics[width=0.9\linewidth]{Tab_res_2}
		\caption{First-Price Elasticities (Extensive Margins)}
		\label{tab:3}
	\end{table}
	The estimated giving price elasticity in terms of extensive margins is about -1.7 $\sim$ -2.9.
\end{frame}

\begin{frame}{Results}
	\begin{itemize}
		\item The results show that the giving price elasticity in Korea is more elastic than many papers in the literature show.
		\begin{itemize}
			\item In the literature, the giving price elasticity in terms of intensive margins is typically -1.
		\end{itemize}
		\item The elasticity in terms of extensive margins captures the behavior of whether people donate or not.
		  \begin{itemize}
		  	\item The elastic extensive margins elasticity shows that reducing giving price induce people to donate.
		  	\item IVを使わない場合の推定結果が得られるならば、ここでIVを使わない推定結果と比べたときの含意を書く。
		  \end{itemize}
	\end{itemize}
\end{frame}

\begin{frame}{Results}
 We try several robustness checks.
		\begin{enumerate}
			\item Estimation excluding 2013 and 2014 data: to eliminate announcement effect.
			\item Estimation using the last-price
			\item Estimation using the subsample of those who have applied for tax relief
			\begin{itemize}
				\item We use Semykina and Wooldridge (2010)'s way of the correction of sample selection bias.
				\item Estimated elasticity (intensive margins) is around -1.2\(\sim\)-1.6.
				\item Subsample analysis enables us to use several methods to deal with issues related to tax deduction system.\\
				e.g. fluctuation of income level (Randolph, 1995, and so on.)\\
				$\to$ usage of k-th difference model / lead and lag.  
			\end{itemize}
		\end{enumerate}
		Most of result shows that the giving price elasticity is less than
		\begin{itemize}
			\item -1.4 in terms of intensive margins and 
			\item -1.7 in terms of extensive margins.
		\end{itemize}
\end{frame}

\begin{frame}{Implications for the welfare (not in the paper)}
	Following Almunia et al. (2020), we can derive the welfare implication by specifying the following utility maximization.
	\begin{align}
		\max_{x_i,g_i,R_i}&U(x_i,g_i,G)= x_i - R_iK +\theta u(g_i) + V(G)\notag\\
		\text{s.t. }&x_i + g_i = R_i(y-T(y, g_i)) + (1-R_i)(y-T(y)),\notag\\
		&\text{ and }G=g_i+G_{-i}\notag,
	\end{align}
	where $\theta\in[\underline{\theta},\bar{\theta}$] is the parameter to show the preference for donation, which follows the density $f(\cdot)$. 
	\begin{itemize}
		\item $G$ is the total donation (including the governmental provision).
		\item $G_{-i}$ is the total donation except $i$.
		\item For simplicity, assume tax schedule is now linear and tax rate is $\tau$.
	\end{itemize}
\end{frame}

\begin{frame}{Implications for the welfare (not in the paper)}
	\begin{itemize}
	\item Then, the optimization will be
	\begin{align}
		\max_{g_i,R_i} (1-\tau)y-R_ipg_i-(1-R_i)g_i- R_iK +\theta u(g_i)+V(G).\notag
	\end{align}
	\item Denote $g(p;\theta)$ and $g(1; \theta)$ as the donation when $R_i=1$ and $0$.
	\item As a result of the optimization, the indirect utility of those who declare giving and do not will respectively be
	\begin{align}
		\nu(p;\theta)&=\theta_i u(g(p;\theta))-pg(p;\theta)\notag\\
		\nu(1;\theta)&=\theta_i u(g(1;\theta))-g(1;\theta)\notag.
	\end{align}
\end{itemize}
\end{frame}

\begin{frame}{Implications for the welfare (not in the paper)}
	\begin{itemize}
		\item Depends on $\theta$ and giving price $p$, three types of individuals exist.
		\begin{enumerate}
			\item Non-donor: $\theta\le \theta_0$
			\item Donor but non-declarer: $\theta\in(\theta_0,\theta(p)]$\\
			$\to$ Denote their total donation as $g^0(p)\equiv\int_{\theta_0}^{\theta(p)}g(1;\theta)f(\theta)d\theta$.
			\item Donor and declarer: $\theta>\theta(p)$\\
			$\to$ Denote their total donation as $g^1(p)\equiv\int_{\theta(q)}^{\bar{\theta}}g(p;\theta)f(\theta)d\theta$.
		\end{enumerate}
	\item Social welfare can be written as
	\begin{align}
		W=V(G)&+\int_{\theta(p)}^{\bar{\theta}}(\nu(p;\theta)-K)f(\theta)d\theta\notag\\
		&+\int_{\theta_0}^{\theta(p)}\nu(1;\theta)f(\theta)d\theta+\lambda[ty-(1-p)g^1(p)-G_g]\notag
	\end{align}
	where $\lambda$ is the marginal cost of public finance.
	\end{itemize}
\end{frame}

\begin{frame}{Implications for the welfare (not in the paper)}
	\begin{itemize}
		\item Using $\lambda=V'$ (Saez, 2004), the effect of changing giving price $p$ on the welfare $W$ can be shown as 
		\begin{align}
			\frac{dW}{dp} =\lambda (g_p^0+g_p^1)+(\lambda-1)g^1-\lambda(1-p)g_p^1\notag.
		\end{align}
		\item $\frac{dW}{dp}<0$ is equivalent to
		\begin{align}
			\epsilon\equiv -\frac{pg_p^1}{g^1}>\frac{\lambda-1}{\lambda}+\frac{g_p^0}{g^1}\notag.
		\end{align}
		\begin{itemize}
		\item From the subsample analysis, $\epsilon$ is 1.304 $\sim$ 1.603.
		\item Assuming $\lambda\in[1,2]$, $\frac{\lambda-1}{\lambda}\in[0,\frac12]$.
		\item From the data, $\frac{g_p^0}{g^1}$ is 0.00003.
		\end{itemize}
		\item Our result suggests that more generous tax relief will increase the welfare in Korea.
	\end{itemize}
\end{frame}
\end{document}