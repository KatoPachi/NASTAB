\documentclass[dvipdfmx,10pt]{beamer}
\usepackage[utf8]{inputenc}
\usepackage[T1]{fontenc}
\usepackage{lmodern,bm,otf}
\usepackage{graphicx}
\usetheme{default}
\begin{document}
	\author{加藤大貴・後藤剛志・金栄禄}
	\title{Charitable Giving, Tax Reform, and Self-selection of Tax Report:
		Evidence from South Korea}
	%\subtitle{}
	%\logo{}
	\institute{大阪大学, 千葉大学, 関西大学}
	%\date{}
	%\subject{}
	%\setbeamercovered{transparent}
	%\setbeamertemplate{navigation symbols}{}
	\begin{frame}[plain]
		\maketitle
	\end{frame}
	
	\begin{frame}{Introduction}
		\protect\hypertarget{introduction}{}
		\begin{itemize}
			\item 世界の多くの国で寄付に対する税制上の優遇がなされている。
			\item 税制優遇が厚生に与える影響について、寄付の価格弾力性が厚生評価のための重要な指標であると考えられている。(e.g.~Saez, 2004)
			\begin{itemize}
				\item 直感的に言えば寄付の価格弾力性が-1を下回ると、\$1の税制優遇が\$1以上の寄付を生み出すこととなる。
				\item 「寄付の税制優遇で寄付の価格が下がり、寄付が促進されるだろう」
			\end{itemize}
			\item 既存研究の多くは課税申告データを用いて寄付の価格弾力性を推定してきた。(e.g.~Almunia et al., 2020, Auten et al., 2002, and so on)
		\end{itemize}
	\end{frame}
	
	\begin{frame}{Introduction}
		\protect\hypertarget{introduction-1}{}
		\begin{itemize}
			\item しかし、課税申告データは申告された寄付のみしか記録していない。
			\begin{itemize}
				\item この論文の扱う第1の問題: 実際の寄付量と申告された寄付量は異なるものであり、推定バイアスが起きる(Fack and Landais (2016); Gillitzer and Skov (2018))。
				\item この問題に対応するため、課税申告データではなく、韓国のパネルデータを使って分析を行った。
			\end{itemize}
			\item 納税者は寄付を申告するかどうか自ら選択でき、申告する際には寄付証明などの提出などのコストがかかる。			
			\begin{itemize}
				\item この論文の扱う第2の問題: 寄付の申告コストが無視されると推定バイアスが起きる。\\
				$\leftarrow$寄付の申告をしなければ税制優遇は受けられない
				\item この論文では操作変数法(IV)とControl Function (CF)アプローチをつかってこの問題に対処。
			\end{itemize}
			\item 差の差法(DID)をメインの識別戦略として用いて、寄付の価格弾力性の大きさについて韓国のデータを用いて調べた。
		\end{itemize}
	\end{frame}
	
	\begin{frame}{Introduction}
		\protect\hypertarget{introduction-2}{}
		分析の結果
		\begin{enumerate}
			\item ベースラインの結果では、寄付の価格弾力性がIntensive Marginsで-1.4以下、Extensive Marginsで-1.7以下となった。
			\begin{itemize}
				\item 寄付の申告コストを考慮しない結果ではIntensive Marginsで-0.9となった。
			\end{itemize}
			\item 寄付の申告者に絞った分析結果では寄付の価格弾力性はIntensive Marginsで-1.2\(\sim\)-1.6となった。
		\end{enumerate}
		既存研究の多くではIntensive Marginsの寄付の価格弾力性が約-1であり、これよりも概ね弾力的な推定結果が得られた。
		\begin{enumerate}
			\setcounter{enumi}{2}
			\item (論文には未掲載) 推定値をもとに社会厚生を分析すると韓国では寄付の税制優遇を拡充することで厚生が増大する可能性が高いことがわかった。
			\item (論文には未掲載) 寄付の申告コストをなくすことで寄付の量が概ね7$\sim$20\%増加することが推定からわかった。
		\end{enumerate}
	\end{frame}
	
	\begin{frame}{2014年の韓国での税制改正}
		\protect\hypertarget{tax-reform-in-south-korea}{}
		韓国では所得税納税者は寄付に対して税制上の優遇を受けることができる。
		\begin{itemize}
			\item 優遇を受けるためには、寄付の証明書を提出して寄付を申告する必要がある。
			\item 給与所得者は所得税を源泉徴収で納税し、寄付の申告は会社で行う。
			\begin{itemize}
				\item 給与所得者は証明書の提出は随時行うことができる。
			\end{itemize}
			\item 非給与所得者は所得税を確定申告で行い、寄付の申告は国税庁を通じて行う。			
			\begin{itemize}
				\item 非給与所得者は確定申告時まで証明書を保存しておく必要がある。
			\end{itemize}
		\end{itemize}
	\end{frame}
	
	\begin{frame}{2014年の韓国での税制改正}
		\protect\hypertarget{tax-reform-in-south-korea-1}{}
		この分析での識別時の主な価格バリエーションは2014年の税制改正によるものである。		
		\begin{itemize}
			\item 2014年より前は寄付の優遇策として所得控除が用いられていた。			
			\begin{itemize}
				\item 所得控除では所得と寄付額によって所得税率が変化するので、(私的財消費と比べたときの)寄付価格は所得や寄付額に依存。
			\end{itemize}
			\item 2014年以後は税額控除が用いられるようになった。
			\begin{itemize}
				\item 税額控除の率は所得などにかかわらず一律15\%
				\item これはつまり、私的財消費\$1と比べた相対的な寄付価格が\$0.85となることを意味\\
				(以下ではこれを「寄付価格」と呼ぶ)
			\end{itemize}
		\end{itemize}
	\end{frame}
	
	\begin{frame}{2014年の韓国での税制改正}
		\protect\hypertarget{tax-reform-in-south-korea-2}{}
		Model
		\begin{itemize}
			\item 私的財消費(\(x_{i}\))と寄付(\(g_{i}\))を考える。
			\item \(y_{i}\)を課税前所得、\(R_{i}\)を寄付申告を示すダミー、\(T(y_i)\)と\(T(y_{i}, g_{i})\)を個人$i$が寄付申告をしなかったときの課税額と申告したときの課税額、$K$を寄付申告のコストだとする。
			\item 予算制約は以下のように表せる。
		\end{itemize}
		
		\[x_{i} + g_{i} = y_{i} - R_iK- R_iT(y_{i}, g_{i})-(1-R_i)T(y_i)\]
		
		
		\begin{itemize}
			\item 個人$i$が寄付を申告するかは、申告時の課税額の減少が申告コストを上回るかによる。
			\begin{align}
				R_i=\begin{cases}
					1 \text{ if }T(y_i, g_i) - T(y_i)>K\\
					0 \text{ if }T(y_i, g_i) - T(y_i)\le K.
				\end{cases}
			\end{align}
		\end{itemize}
	\end{frame}
	
	\begin{frame}{2014年の韓国での税制改正}
		\textbf{所得控除での寄付申告時の税額(2013年まで)}
		\[T(y_{i}, g_{i}) = T(y_{i} - g_{i})\]
		
		\begin{itemize}
			\item 対数を取ったときの寄付価格は\(R_{i}\ln(1 - T'(y_{i} - g_{i}))\)と表せる。
			\item 2012年と2013年の韓国の所得税率は同じだが、2011年以前は異なる所得税率であった。
		\end{itemize}
		
		\textbf{税額控除での寄付申告時の税額(2014年から)}
		\[T(y_{i}, g_{i}) = T(y_{i}) - m g_{i}\]
		
		\begin{itemize}
			\item \(m\)は税額控除率で\(m = 0.15\)と設定されている。
			\item 対数をとったときの寄付価格は\(R_{i}\ln(1 - 0.15) = R_{i}\ln 0.85\)と表せる。
		\end{itemize}
		
		Note: 寄付申告を行わなかった者の直面する寄付価格は\(\ln1 =0\)となる。
	\end{frame}
	
	\begin{frame}{内生性の問題}
		\begin{enumerate}
			\item 課税申告データの使用は申告された寄付額のみを捉えることとなる。
			\begin{itemize}
				\item 未申告の寄付に寄付の税制優遇は無意味のはず(=推定バイアスに繋がる)
				\item 先行研究ではサーベイデータでこの問題に対処 (e.g.~Rehavi and Shack, 2013).
				\item \textbf{本稿でも韓国のサーベイデータでこの問題に対処。}
			\end{itemize}
			\item 申告コストが無視されると推定バイアスが発生する可能性。
			\begin{itemize}
				\item 申告コストが安くて寄付価格が安くなる人だけを捉えて推定する可能性(Self-selectionバイアス)
				\item 知る限りAlmunia et al.~(2020)のみが申告コストを考慮しているが、課税申告データを使用してしまっている。
				\item\textbf{本稿では給与所得者と非給与所得者の違いを操作変数(IV)として採用し、申告コストを考慮。}
			\end{itemize}
		\end{enumerate}
	\end{frame}
	
	\begin{frame}{Data}
		使用するデータは韓国のNational Survey of Tax and Benefit(NaSTab)のデータ
		\begin{itemize}
			\item このデータの対象は韓国全体の15の市と地域の一般世帯とその構成員。
			\item データは主に対面のインタビュー形式で回答されたもの。
			\item データサンプルは韓国の一般的な社会構成が代表されるように構成されている。
			\item 本稿では所得や資産がないと思われる23歳以下のサンプルは除いて分析している。
			\item データとして2012$\sim$2017年のデータを使用。(2014年の制度改正に着目)
		\end{itemize}
	\end{frame}

\begin{frame}{Data}
	\begin{figure}
		\centering
		\includegraphics[width=0.8\linewidth]{Fig_Average_donation}
		\caption{寄付者の割合と寄付者の平均寄付額}
		\label{fig:1}
	\end{figure}
\small
	\begin{itemize}
		\item 20$\sim$30\%の人が寄付を実施
		\item 寄付者の平均寄付額は150万KRW($\simeq$15万円)程度
	\end{itemize}
\end{frame}

\begin{frame}{Data}
	\begin{figure}
		\centering
		\includegraphics[width=0.7\linewidth]{Fig_Income_distribution}
		\caption{所得分布と税制改正前後での寄付価格の変化}
		\label{fig:2}
	\end{figure}
\small
	2014年に寄付価格は概ね
	\begin{itemize}
		\item 所得1200万KRW未満の人については減少
		\item 所得1200万KRW$\sim$4600万KRWの人については同じ
		\item 所得4600万KRW以上の人については増加
	\end{itemize}
\end{frame}

\begin{frame}{Data}
	\begin{figure}
		\centering
		\includegraphics[width=0.7\linewidth]{Fig_Diff}
		\caption{各所得グループでの平均寄付額(対数)}
		\label{fig:3}
	\end{figure}
\small
	2012年と2013年に比べ、2014年以降の寄付額は
	\begin{itemize}
		\item 所得1200万KRW未満の人については増加
		\item 所得1200万KRW$\sim$4600万KRWの人についてはやや増加
		\item 所得4600万KRW以上の人については減少
	\end{itemize}
\end{frame}

\begin{frame}{Data}
	\begin{figure}
		\centering
		\includegraphics[width=0.7\linewidth]{Fig_Declaration}
		\caption{寄付申告者の割合}
		\label{fig:4}
	\end{figure}
\small
	\begin{itemize}
		\item 給与所得者は税制優遇を受けるため、より寄付の申告を行う傾向があるとわかる。\\
		$\to$ これは申告コストの違いを捉えたものだと考えられる。
	\end{itemize}
\end{frame}

\begin{frame}{Data}
	\begin{table}
		\centering
		\includegraphics[width=0.9\linewidth]{Tab_Stat}
		\caption{基本統計量}
		\label{tab:1}
	\end{table}
\end{frame}

\begin{frame}{推定モデル}
\begin{itemize}
	\item 主な識別戦略として2014年の税制改正を使い、DIDに準じた推定を行う。
	\begin{itemize}
		\item 所得1200万KRW未満:寄付価格は減少
		\item 所得1200万KRW$\sim$4600万KRW:寄付価格は同じ
		\item 所得4600万KRW以上:寄付価格は増加
	\end{itemize}
	\item さらに申告コストの差の影響を捉えるために、サンプルが給与所得者かどうかをみるダミーをIVとして使用。
\end{itemize}
\end{frame}

\begin{frame}{推定モデル}
	\begin{itemize}
		\item 以下のような二元配置固定効果モデルを考える。
		\begin{align}
			\ln g_{it} = \varepsilon_pR_{it} \ln p_{it}(y_{it}, g_{it}) + \varepsilon_y \ln y_{it} + \bm{X}_{it}\bm{\beta} + \mu_i + \iota_t + u_{it}\tag{5}
		\end{align}
	$\mu_i, \iota_t$と$ u_{it}$はそれぞれ個人固定効果、年の固定効果、誤差項 
	\item $\bm{X}_{it}$は年齢の2乗、産業(職業)ダミー、地域ダミー
	\item 寄付の研究文脈では\textbf{intensive margins}と\textbf{extensive margins}を分けて弾力性を推定することが行われている。
	\begin{itemize}
		\item \textbf{Intensive margins}: (5)を寄付した個人($g_{it}>0$)に限定して分析\\
		$-$寄付価格の変化でどれくらいの量寄付が変化するかに関心
		\item \textbf{Extensive margins}: (5)の被説明変数を寄付するかどうか($1[g_{it}>0]$)にして分析\\
		$-$寄付価格の変化で寄付を行うようになるかどうかに関心
	\end{itemize}
	$\to$ 本稿ではこれらのどちらも分析
	\end{itemize}
\end{frame}

\begin{frame}{推定モデル: $p_{it}$の内生性}
	\begin{itemize}
		\item (私的財\$1と比較した)寄付の価格は
		\begin{align}
			p_{it}(y_{it}, g_{it})=\begin{cases}
			1-T'_t(y_{it}-g_{it})&\text{ if }t<2014\\
			0.85&\text{ if }t\ge2014
			\end{cases}
			\tag{6'}.
		\end{align}
		\item 2014年より前は寄付価格が寄付額に依存し、内生的となってしまうため、``\textbf{first-price of giving}"と呼ばれる寄付額が0の状態の寄付価格
		\begin{align}
			p_{it}^f(y_{it})=p_{it}(y_{it},0)\notag
		\end{align}
		を実際の寄付価格$p_{it}(y_{it},0)$の代わりに使用
		\item 実際の寄付価格$p_{it}(y_{it},0)$は``last-price of giving"と呼ばれ、これについても分析し頑健性を確認。
	\end{itemize}
\end{frame}

\begin{frame}{推定モデル: 寄付申告$R_{it}$の内生性}
	\begin{itemize}
		\item 寄付者は寄付を申告するかどうかを自ら選択できるため、寄付申告$R_{it}$は内生的となる。
		\item そのため、給与所得者ダミー$WageEarner_{it}$を寄付申告$R_{it}$の操作変数として用い、以下を分析。
		\begin{align}
			\ln g_{it} = \varepsilon_pR_{it} \ln p_{it}^f(y_{it}) + \varepsilon_y \ln y_{it} + \bm{X}_{it}\bm{\beta} + \mu_i + \iota_t + u_{it}\tag{7}
		\end{align} 
	 	\begin{itemize}
	 		\item 2段階最小二乗法(2SLS)に基づく推定
			\item 給与所得者ダミー$WageEarner_{it}$は所得や職業ダミーをコントロールした上では誤差項$u_{it}$とは相関しないと考えられる。
		\end{itemize}
	\end{itemize}
\end{frame}

\begin{frame}{推定モデル: 寄付申告$R_{it}$の内生性}
	\begin{itemize}
		\item 2SLSに加え、寄付申告の傾向スコア$P(Z_{it})$を操作変数として用いた推定も実施
		\begin{itemize}
			\item この方法は``control function (CF)"アプローチと呼ばれる。
			\item 傾向スコアは以下のプロビットモデルで推定
			\begin{align}
				R_{it}=1[\delta_0+Z_{it}\delta_1+u_{it1}>0]\tag{8},
			\end{align}
			ただし$Z_{it}\equiv \{WageEarner_{it}, \ln p_{it}^f(y_{it}), \ln y_{it}, \bm{X}_{it}\}$.
			\item 傾向スコアは$P(Z_{it})=\Phi(\hat{\delta}_0+Z_{it}\hat{\delta}_1)$となる。
			\item 2つの異なる仮定に基づいて別々の分析を実施\\
			 \ajMaru1 Pooledプロビット:係数$(\delta_0, \delta_1)$が時間を通じて一定\\
			 \ajMaru2 Separatedプロビット:$(\delta_0, \delta_1)$が時間で可変
		\end{itemize}
		\item さらに、寄付申告$R_{it}$の代わりに傾向スコアを直接使った分析も実施
		\begin{align}
			\ln g_{it} = \varepsilon_pP(Z_{it})\ln p_{it}^f(y_{it}) + \varepsilon_y \ln y_{it} + \bm{X}_{it}\bm{\beta} + \mu_i + \iota_t + u_{it}\tag{9}
		\end{align}
	\end{itemize}
\end{frame}

\begin{frame}{結果: Intensive Margins}
	\begin{table}
		\centering
		\includegraphics[width=0.9\linewidth]{Tab_res_1}
		\caption{First-Price Elasticities (Intensive Margins)}
		\label{tab:2}
	\end{table}
Intensive marginsでの寄付の価格弾力性の推定値は概ね-1.5程度となった。\\
(申告コストを考慮しない通常のモデルでの推定値は-0.91)
\end{frame}

\begin{frame}{結果: Extensive Margins}
	\begin{table}
		\centering
		\includegraphics[width=0.9\linewidth]{Tab_res_2}
		\caption{First-Price Elasticities (Extensive Margins)}
		\label{tab:3}
	\end{table}
	Intensive marginsでの寄付の価格弾力性の推定値は概ね-1.7 $\sim$ -2.9となった。
\end{frame}

\begin{frame}{結果}
	\begin{itemize}
		\item 分析結果から韓国での寄付の価格弾力性は多くの先行研究で示された値より弾力的であることがわかった。
		\begin{itemize}
			\item 多くの先行研究ではIntensive Marginsの寄付の価格弾力性は-1程度
			\item 申告コストをIVに使用しない分析では-0.91となったため、申告コストの無視で過小推定になった可能性
		  	\item Extensive Marginsの寄付の価格弾力性についても先行研究と比べかなり弾力的な値\\
		  	e.g. Almunia et al.(2020)は最も弾力的な値で-0.73\\
		  	e.g. Backus and Grant (2019)では統計的に有意でなく、概ね0
		  \end{itemize}
	\end{itemize}
\end{frame}

\begin{frame}{結果:ロバストネスチェック}
 様々なロバストネスチェックを実施
		\begin{enumerate}
			\item 2013,2014年をOmit:税制改正のアナウンスメント効果を除去
			\item First priceでなく、Last priceで分析
			\item 寄付申告者に限定したサブサンプルによる分析
			\begin{itemize}
				\item Semykina and Wooldridge (2010)のsample selection bias除去の方法を利用
				\item Intensive marginsの寄付の価格弾力性は概ね-1.2\(\sim\)-1.6.
				\item 寄付申告者に限定することで所得控除制度による内生性を考えたロバストネスチェックを実施可能\\
				e.g. 所得の変動 (Randolph, 1995, and so on.)\\
				$\to$ k-thの階差モデル / リード・ラグの使用  
			\end{itemize}
		\end{enumerate}
		ほとんどの分析結果で寄付の価格弾力性が
		\begin{itemize}
			\item Intensive Marginsで-1.4以下 
			\item Extensive Marginsで-1.7以下
		\end{itemize}
		\flushright{となるとわかった。}
\end{frame}

\begin{frame}{Implications for the welfare (not in the paper)}
	Following Almunia et al. (2020), we can derive the welfare implication by specifying the following utility maximization.
	\begin{align}
		\max_{x_i,g_i,R_i}&U(x_i,g_i,G)= x_i - R_iK +\theta u(g_i) + V(G)\notag\\
		\text{s.t. }&x_i + g_i = R_i(y-T(y, g_i)) + (1-R_i)(y-T(y)),\notag\\
		&\text{ and }G=g_i+G_{-i}\notag,
	\end{align}
	where $\theta\in[\underline{\theta},\bar{\theta}$] is the parameter to show the preference for donation, which follows the density $f(\cdot)$. 
	\begin{itemize}
		\item $G$ is the total donation (including the governmental provision).
		\item $G_{-i}$ is the total donation except $i$.
		\item For simplicity, assume tax schedule is now linear and tax rate is $\tau$.
	\end{itemize}
\end{frame}

\begin{frame}{Implications for the welfare (not in the paper)}
	\begin{itemize}
	\item Then, the optimization will be
	\begin{align}
		\max_{g_i,R_i} (1-\tau)y-R_ipg_i-(1-R_i)g_i- R_iK +\theta u(g_i)+V(G).\notag
	\end{align}
	\item Denote $g(p;\theta)$ and $g(1; \theta)$ as the donation when $R_i=1$ and $0$.
	\item As a result of the optimization, the indirect utility of those who declare giving and do not will respectively be
	\begin{align}
		\nu(p;\theta)&=\theta_i u(g(p;\theta))-pg(p;\theta)\notag\\
		\nu(1;\theta)&=\theta_i u(g(1;\theta))-g(1;\theta)\notag.
	\end{align}
\end{itemize}
\end{frame}

\begin{frame}{Implications for the welfare (not in the paper)}
	\begin{itemize}
		\item Depends on $\theta$ and giving price $p$, three types of individuals exist.
		\begin{enumerate}
			\item Non-donor: $\theta\le \theta_0$
			\item Donor but non-declarer: $\theta\in(\theta_0,\theta(p)]$\\
			$\to$ Denote their total donation as $g^0(p)\equiv\int_{\theta_0}^{\theta(p)}g(1;\theta)f(\theta)d\theta$.
			\item Donor and declarer: $\theta>\theta(p)$\\
			$\to$ Denote their total donation as $g^1(p)\equiv\int_{\theta(q)}^{\bar{\theta}}g(p;\theta)f(\theta)d\theta$.
		\end{enumerate}
	\item Social welfare can be written as
	\begin{align}
		W=V(G)&+\int_{\theta(p)}^{\bar{\theta}}(\nu(p;\theta)-K)f(\theta)d\theta\notag\\
		&+\int_{\theta_0}^{\theta(p)}\nu(1;\theta)f(\theta)d\theta+\lambda[ty-(1-p)g^1(p)-G_g]\notag
	\end{align}
	where $\lambda$ is the marginal cost of public finance.
	\end{itemize}
\end{frame}

\begin{frame}{Implications for the welfare (not in the paper)}
	\begin{itemize}
		\item Using $\lambda=V'$ (Saez, 2004), the effect of changing giving price $p$ on the welfare $W$ can be shown as 
		\begin{align}
			\frac{dW}{dp} =\lambda (g_p^0+g_p^1)+(\lambda-1)g^1-\lambda(1-p)g_p^1\notag.
		\end{align}
		\item $\frac{dW}{dp}<0$ is equivalent to
		\begin{align}
			\epsilon\equiv -\frac{pg_p^1}{g^1}>\frac{\lambda-1}{\lambda}+\frac{g_p^0}{g^1}\notag.
		\end{align}
		\begin{itemize}
		\item From the subsample analysis, $\epsilon$ is 1.304 $\sim$ 1.603.
		\item Assuming $\lambda\in[1,2]$, $\frac{\lambda-1}{\lambda}\in[0,\frac12]$.
		\item From the data, $\frac{g_p^0}{g^1}$ is 0.00003.
		\end{itemize}
		\item Our result suggests that more generous tax relief will increase the welfare in Korea.
	\end{itemize}
\end{frame}
\end{document}