% Options for packages loaded elsewhere
\PassOptionsToPackage{unicode}{hyperref}
\PassOptionsToPackage{hyphens}{url}
%
\documentclass[
  ignorenonframetext,
]{beamer}
\usepackage{pgfpages}
\setbeamertemplate{caption}[numbered]
\setbeamertemplate{caption label separator}{: }
\setbeamercolor{caption name}{fg=normal text.fg}
\beamertemplatenavigationsymbolsempty
% Prevent slide breaks in the middle of a paragraph
\widowpenalties 1 10000
\raggedbottom
\setbeamertemplate{part page}{
  \centering
  \begin{beamercolorbox}[sep=16pt,center]{part title}
    \usebeamerfont{part title}\insertpart\par
  \end{beamercolorbox}
}
\setbeamertemplate{section page}{
  \centering
  \begin{beamercolorbox}[sep=12pt,center]{part title}
    \usebeamerfont{section title}\insertsection\par
  \end{beamercolorbox}
}
\setbeamertemplate{subsection page}{
  \centering
  \begin{beamercolorbox}[sep=8pt,center]{part title}
    \usebeamerfont{subsection title}\insertsubsection\par
  \end{beamercolorbox}
}
\AtBeginPart{
  \frame{\partpage}
}
\AtBeginSection{
  \ifbibliography
  \else
    \frame{\sectionpage}
  \fi
}
\AtBeginSubsection{
  \frame{\subsectionpage}
}
\usepackage{lmodern}
\usepackage{amssymb,amsmath}
\usepackage{ifxetex,ifluatex}
\ifnum 0\ifxetex 1\fi\ifluatex 1\fi=0 % if pdftex
  \usepackage[T1]{fontenc}
  \usepackage[utf8]{inputenc}
  \usepackage{textcomp} % provide euro and other symbols
\else % if luatex or xetex
  \usepackage{unicode-math}
  \defaultfontfeatures{Scale=MatchLowercase}
  \defaultfontfeatures[\rmfamily]{Ligatures=TeX,Scale=1}
\fi
% Use upquote if available, for straight quotes in verbatim environments
\IfFileExists{upquote.sty}{\usepackage{upquote}}{}
\IfFileExists{microtype.sty}{% use microtype if available
  \usepackage[]{microtype}
  \UseMicrotypeSet[protrusion]{basicmath} % disable protrusion for tt fonts
}{}
\makeatletter
\@ifundefined{KOMAClassName}{% if non-KOMA class
  \IfFileExists{parskip.sty}{%
    \usepackage{parskip}
  }{% else
    \setlength{\parindent}{0pt}
    \setlength{\parskip}{6pt plus 2pt minus 1pt}}
}{% if KOMA class
  \KOMAoptions{parskip=half}}
\makeatother
\usepackage{xcolor}
\IfFileExists{xurl.sty}{\usepackage{xurl}}{} % add URL line breaks if available
\IfFileExists{bookmark.sty}{\usepackage{bookmark}}{\usepackage{hyperref}}
\hypersetup{
  pdftitle={Comments on Kauer et al.(2021)},
  pdfauthor={XXX},
  hidelinks,
  pdfcreator={LaTeX via pandoc}}
\urlstyle{same} % disable monospaced font for URLs
\newif\ifbibliography
\setlength{\emergencystretch}{3em} % prevent overfull lines
\providecommand{\tightlist}{%
  \setlength{\itemsep}{0pt}\setlength{\parskip}{0pt}}
\setcounter{secnumdepth}{-\maxdimen} % remove section numbering

\title{Comments on Kauer et al.(2021)}
\author{XXX}
\date{2021/8/DD}

\begin{document}
\frame{\titlepage}

\begin{frame}{Summary of this paper (based on my understanding)}
\protect\hypertarget{summary-of-this-paper-based-on-my-understanding}{}

\begin{itemize}
\tightlist
\item
  Using the Indian-states panel data from 2003 to 2019, this paper
  analyzes whether the fiscal grant cause the fly paper effect or not.
\item
  This paper focus on the delegation of some parts of the tax revenue
  from the central government to some state governments.

  \begin{itemize}
  \tightlist
  \item
    This delegation is called ``tax devolution'' or ``ecological fiscal
    transfer (EFT)'' in this paper.
  \item
    The tax devolution was done for forest conservation by state
    governments.
  \item
    The devolution formula depends on the condition of the forests.
  \end{itemize}
\end{itemize}

\end{frame}

\begin{frame}{Summary of this paper (based on my understanding)}
\protect\hypertarget{summary-of-this-paper-based-on-my-understanding-1}{}

\begin{itemize}
\tightlist
\item
  The panel data analysis finds that the tax devolution increase the
  expenditure for forest conservation.
\item
  According to the authors, the result about the flypaper effect is
  mixed.
\end{itemize}

\end{frame}

\begin{frame}{Fly paper effect (based on my understanding)}
\protect\hypertarget{fly-paper-effect-based-on-my-understanding}{}

The lump-sum transfer from the central government sticks to certain
expenditure(s) although it is lump-sum.

\begin{center}
\includegraphics[width=9cm,clip]{figure1.png}\\
Figure: The image of the flypaper effect\\
(Made by T. Goto based on Sato (2009) and Chalil (2018))
\end{center}

\end{frame}

\begin{frame}{Comment 1}
\protect\hypertarget{comment-1}{}

\begin{itemize}
\tightlist
\item
  Considering the flypaper effect, it seems that the analyses on several
  expenditures are needed.

  \begin{itemize}
  \tightlist
  \item
    See Kakamu et al.~(2014) and Langer and Korzhenevych (2019) for
    example.
  \end{itemize}
\item
  However, the current version of this paper considers only the
  expenditure for ecology.
\item
  The analyses on several expenditures are needed.
\end{itemize}

\end{frame}

\begin{frame}{Comment 2}
\protect\hypertarget{comment-2}{}

\begin{itemize}
\tightlist
\item
  The title of this paper is ``Ecological fiscal transfers and local
  government spending: The flypaper effects in
  \textbf{the era of pandemic}''
\item
  Which parts are related to the pandemic of Covid-19 in this paper?

  \begin{itemize}
  \tightlist
  \item
    This paper uses Indian panel data from 2003 to 2019.
  \item
    Is the data affected by the pandemic? If so, why not refer to it in
    the paper?
  \end{itemize}
\end{itemize}

\end{frame}

\begin{frame}{Comment 3}
\protect\hypertarget{comment-3}{}

\begin{itemize}
\tightlist
\item
  The suitable identification strategy for this paper may be
  ``difference-in-difference'' (DD).

  \begin{itemize}
  \tightlist
  \item
    This is because the tax devolution was distributed to the certain
    states with forests.
  \item
    Average Treatment Effect on the state with forests may be derived
    though ATE itself cannot be.
  \end{itemize}
\item
  Please note when the tax devolution began in the paper.
\end{itemize}

\end{frame}

\begin{frame}{Minor Comments}
\protect\hypertarget{minor-comments}{}

\begin{itemize}
\tightlist
\item
  Summary stat must be written in the paper.
\item
  The detailed hypothesis (e.g.~the expected sign of coefficients)
  should be described.
\item
  Interpretations about the results of analyses should be described.
\item
  The omission of outliers seems arbitrary.
\end{itemize}

\end{frame}

\begin{frame}{References}
\protect\hypertarget{references}{}

\begin{itemize}
\tightlist
\item
  Chalil (2018) ``The Size of Flypaper Effect in Decentralizing
  Indonesia.'' The Indonesian Journal of Development Planning. 2(2).
  101-119.
\item
  Kakamu, Yunoue, and Kuramoto (2014) ``Spatial patterns of flypaper
  effects for local expenditure by policy objective in Japan: A Bayesian
  approach.'' Economic Modelling. 37. 500-506.
\item
  Langer and Korzhenevych (2019) ``Equalization Transfers and the
  Pattern of Municipal Spending: An Investigation of the Flypaper Effect
  in Germany.'' Annals of Economics and Finance. 20(2). 737-765.
\item
  Sato (2009) ``Introductory local public finance.'' Shinseisya.
  (Japanese).
\end{itemize}

\end{frame}

\end{document}
