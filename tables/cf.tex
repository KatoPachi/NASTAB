\begin{table}

\caption{Estimation Results of Control Function Model\label{tab:cf}}
\centering
\fontsize{8}{10}\selectfont
\begin{threeparttable}
\begin{tabular}[t]{l>{\centering\arraybackslash}p{10em}>{\centering\arraybackslash}p{10em}}
\toprule
\multicolumn{1}{c}{ } & \multicolumn{1}{c}{Log donation} & \multicolumn{1}{c}{A dummy of donor} \\
\cmidrule(l{3pt}r{3pt}){2-2} \cmidrule(l{3pt}r{3pt}){3-3}
  & (1) & (2)\\
\midrule
Effective price & \num{-1.596}*** & \num{-0.501}***\\
 & (\num{0.384}) & (\num{0.037})\\
Log income & \num{1.641} & \num{1.758}***\\
 & (\num{1.234}) & (\num{0.152})\\
Residuals of Application & \num{-0.555}*** & \num{0.505}***\\
 & (\num{0.201}) & (\num{0.023})\\
Application $\times$ Residuals of Application & \num{0.396}* & \num{0.336}***\\
 & (\num{0.214}) & (\num{0.024})\\
\midrule
\addlinespace[0.3em]
\multicolumn{3}{l}{\textit{Implied price elasticity}}\\
\hspace{1em}Estimates &  & \num{-2.110}***\\
\hspace{1em} &  & (\num{0.157})\\
Num.Obs. & \num{7993} & \num{30487}\\
R2 Adj. & \num{0.091} & \num{0.495}\\
\bottomrule
\end{tabular}
\begin{tablenotes}
\item Notes: * p < 0.1, ** p < 0.05, *** p < 0.01. Standard errors clustered at household level are in parentheses. An outcome variable is logged value of amount of charitable giving in model (1) and a dummy indicating that donor in model (2). For estimation, model (1) uses only donors (intensive-margin sample) and model (2) use both donors and non-donors (extensive-margin sample). We control squared age (divided by 100), number of household members, a dummy that indicates having dependents, a set of dummies of industry, a set of dummies of residential area, and time fixed effects. We use an wage earner dummy as an instrument to obtain residuals of application. Instead individual fixed effects, we control a vector of individual-level sample mean of all exogenous variables including instruments (Chamberlain-Mundlak device).
\end{tablenotes}
\end{threeparttable}
\end{table}
