\begin{table}

\caption{Estimation Results of Control Function Model\label{tab:cf}}
\centering
\fontsize{8}{10}\selectfont
\begin{threeparttable}
\begin{tabular}[t]{lcc}
\toprule
\multicolumn{1}{c}{ } & \multicolumn{1}{c}{Log donation} & \multicolumn{1}{c}{A dummy of donor} \\
\cmidrule(l{3pt}r{3pt}){2-2} \cmidrule(l{3pt}r{3pt}){3-3}
  & (1) & (2)\\
\midrule
Effective price ($\beta_e$) & \num{-1.233}*** & \num{-0.243}***\\
 & (\num{0.352}) & (\num{0.031})\\
Log income & \num{2.300}** & \num{2.564}***\\
 & (\num{1.166}) & (\num{0.147})\\
Residuals of Application ($\psi_1$) & \num{-0.196}*** & \num{0.791}***\\
 & (\num{0.075}) & (\num{0.006})\\
\midrule
Estimate &  & \num{-1.025}***\\
 &  & (\num{0.131})\\
\addlinespace[0.3em]
\multicolumn{3}{l}{\textit{Implied price elasticity}}\\
\hspace{1em}Num.Obs. & \num{7993} & \num{30487}\\
\hspace{1em}R2 Adj. & \num{0.090} & \num{0.493}\\
\bottomrule
\end{tabular}
\begin{tablenotes}
\item Notes: * p < 0.1, ** p < 0.05, *** p < 0.01. Robust standard errors are in parentheses. An outcome variable is logged value of amount of charitable giving in model (1) and a dummy indicating that donor in model (2). For estimation, model (1) uses only donors (intensive-margin sample) and model (2) use both donors and non-donors (extensive-margin sample). We control squared age (divided by 100), number of household members, a dummy that indicates having dependents, a set of dummies of industry, a set of dummies of residential area, and time fixed effects. We use an wage earner dummy as an instrument to obtain residuals of application. Instead individual fixed effects, we control a vector of individual-level sample mean of all exogenous variables including instruments (Chamberlain-Mundlak device).
\end{tablenotes}
\end{threeparttable}
\end{table}
