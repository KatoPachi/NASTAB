\begin{table}

\caption{Regression Results of Two-Way Fixed Effect Model (Extensive-Margin Sample)\label{tab:fe-model-ext}}
\centering
\fontsize{8}{10}\selectfont
\begin{threeparttable}
\begin{tabular}[t]{lccccc}
\toprule
\multicolumn{1}{c}{ } & \multicolumn{5}{c}{A dummy of donation} \\
\cmidrule(l{3pt}r{3pt}){2-6}
  & (1) & (2) & (3) & (4) & (5)\\
\midrule
Decrease x Credit period & \num{0.022} & \num{0.008} &  &  & \\
 & (\num{0.014}) & (\num{0.016}) &  &  & \\
Increase x Credit period & \num{-0.029}** & \num{-0.007} &  &  & \\
 & (\num{0.012}) & (\num{0.018}) &  &  & \\
Log applicable price &  &  & \num{-0.183}*** & \num{0.028} & \\
 &  &  & (\num{0.064}) & (\num{0.095}) & \\
Log effective last-price &  &  &  &  & \num{-2.706}***\\
 &  &  &  &  & (\num{0.087})\\
Wage earner &  & \num{0.024} &  & \num{-0.022} & \\
 &  & (\num{0.015}) &  & (\num{0.025}) & \\
Wage earner x Decrease &  & \num{-0.056} &  &  & \\
 &  & (\num{0.042}) &  &  & \\
Wage earner x Increase &  & \num{0.113}*** &  &  & \\
 &  & (\num{0.036}) &  &  & \\
Wage earner x Decrease x Credit period &  & \num{0.059}* &  &  & \\
 &  & (\num{0.032}) &  &  & \\
Wage earner x Increase x Credit period &  & \num{-0.029} &  &  & \\
 &  & (\num{0.021}) &  &  & \\
Wage earner x Log applicable price &  &  &  & \num{-0.366}*** & \\
 &  &  &  & (\num{0.126}) & \\
Log income & \num{0.866}*** & \num{0.870}*** & \num{0.864}*** & \num{0.904}*** & \num{0.312}\\
 & (\num{0.179}) & (\num{0.178}) & (\num{0.207}) & (\num{0.210}) & (\num{0.195})\\
\midrule
\addlinespace[0.3em]
\multicolumn{6}{l}{\textit{Implied price elasticity}}\\
\hspace{1em}Overall &  &  & -0.760*** &  & -11.247***\\
\hspace{1em} &  &  & (0.266) &  & (0.363)\\
\hspace{1em}Wage earner &  &  &  & -1.499** & \\
\hspace{1em} &  &  &  & (0.663) & \\
\hspace{1em}Non wage earner &  &  &  & 0.117 & \\
\hspace{1em} &  &  &  & (0.393) & \\
Num.Obs. & \num{22791} & \num{22791} & \num{25343} & \num{25341} & \num{25343}\\
\bottomrule
\end{tabular}
\begin{tablenotes}
\item Notes: * p < 0.1, ** p < 0.05, *** p < 0.01. We use standard errors clustered at household level. An outcome variable is a dummy indicating that an amount of charitable giving is positive. For estimation, we use those whose amount of donation is not only positive but also zero. We control squared age (divided by 100), number of household members, a dummy that indicates having dependents, a set of dummies of industry, a set of dummies of residential area, and individual and time fixed effects. We calculate implied price elasticities by dividing estimates by proportion of donors in our sample.
\end{tablenotes}
\end{threeparttable}
\end{table}
