\begin{table}

\caption{Estimation Results of Tax Relief Application Model (Linear Probability Model)\label{tab:application}}
\centering
\fontsize{8}{10}\selectfont
\begin{threeparttable}
\begin{tabular}[t]{l>{\centering\arraybackslash}p{12em}}
\toprule
\multicolumn{1}{c}{ } & \multicolumn{1}{c}{Dummy of application} \\
\cmidrule(l{3pt}r{3pt}){2-2}
  & (1)\\
\midrule
\addlinespace[0.3em]
\multicolumn{2}{l}{\textit{Excluded instrument}}\\
\hspace{1em}Wage earner & \num{0.103}***\\
\hspace{1em} & (\num{0.008})\\
\addlinespace[0.3em]
\multicolumn{2}{l}{\textit{Covariates}}\\
\hspace{1em}Applicable price & \num{-0.222}***\\
\hspace{1em} & (\num{0.045})\\
\hspace{1em}Log income & \num{2.439}***\\
\hspace{1em} & (\num{0.200})\\
\hspace{1em}Squared age (divided by 100) & \num{0.001}**\\
\hspace{1em} & (\num{0.001})\\
\hspace{1em}Number of household members & \num{-0.009}**\\
\hspace{1em} & (\num{0.004})\\
\hspace{1em}Having dependents & \num{0.014}**\\
\hspace{1em} & (\num{0.007})\\
\midrule
Num.Obs. & \num{30487}\\
\bottomrule
\end{tabular}
\begin{tablenotes}
\item Notes: * p < 0.1, ** p < 0.05, *** p < 0.01. Standard errors clustered at household level are in parentheses. An outcome variable is a dummy of application of tax relief. For estimation, we use both donors and non-donors (extensive-margin sample). Additionally, we control a set of dummies of industry, a set of dummies of residential area, and time fixed effects. We use a wage earner dummy as an instrument. Instead individual fixed effects, we control a vector of individual-level sample mean of all exogenous variables including instrument (Chamberlain-Mundlak device).
\end{tablenotes}
\end{threeparttable}
\end{table}
