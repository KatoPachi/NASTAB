\begin{table}
\centering
\begin{threeparttable}
\begin{tabular}[t]{lc}
\toprule
\multicolumn{1}{c}{ } & \multicolumn{1}{c}{Log donation} \\
\cmidrule(l{3pt}r{3pt}){2-2}
  & (1)\\
\midrule
Effective price ($\beta_e$) & \num{-0.359}\\
 & (\num{0.531})\\
Log income & \num{2.211}*\\
 & (\num{1.168})\\
Residuals of Application ($\psi_1$) & \num{-0.315}***\\
 & (\num{0.097})\\
Effective price $\times$ Residuals of Application ($\psi_2$) & \num{-2.065}**\\
 & (\num{1.008})\\
\midrule
\addlinespace[0.3em]
\multicolumn{2}{l}{\textit{Aggregated price elasticity}}\\
\hspace{1em}Sample average & \num{-0.989}\\
\hspace{1em}Sample average among claimants & \num{-1.732}\\
\hspace{1em}Sample average among non-claimants & \num{-0.090}\\
\hspace{1em}Weighted sum (Chaisemartin and D'haultf\oe uille, 2020) & \num{-1.615}\\
Num.Obs. & \num{7993}\\
R2 Adj. & \num{0.090}\\
\bottomrule
\end{tabular}
\begin{tablenotes}
\item Notes: * p < 0.1, ** p < 0.05, *** p < 0.01. Robust standard errors are in parentheses. An outcome variable is logged value of amount of charitable giving in model (1). For estimation, model (1) uses only donors (intensive-margin sample). We control squared age (divided by 100), number of household members, a dummy that indicates having dependents, a set of dummies of industry, a set of dummies of residential area, and time fixed effects. We use an wage earner dummy as an instrument to obtain residuals of application. Instead individual fixed effects, we control a vector of individual-level sample mean of all exogenous variables including instruments (Chamberlain-Mundlak device).
\end{tablenotes}
\end{threeparttable}
\end{table}
