\begin{table}

\caption{First-Price Elasticity of Declaration\label{tab:claim-elasticity}}
\centering
\fontsize{8}{10}\selectfont
\begin{threeparttable}
\begin{tabular}[t]{>{\raggedright\arraybackslash}p{25em}>{\centering\arraybackslash}p{15em}}
\toprule
\multicolumn{1}{c}{ } & \multicolumn{1}{c}{1 = Declaration} \\
\cmidrule(l{3pt}r{3pt}){2-2}
\multicolumn{1}{c}{ } & \multicolumn{1}{c}{FE} \\
\cmidrule(l{3pt}r{3pt}){2-2}
  & (1)\\
\midrule
Applicable price & \num{-0.220}***\\
 & (\num{0.050})\\
Log income & \num{1.380}***\\
 & (\num{0.177})\\
\midrule
\addlinespace[0.3em]
\multicolumn{2}{l}{\textit{Implied price elasticity}}\\
\hspace{1em}Estimate & \num{-1.982}***\\
\hspace{1em} & (\num{0.452})\\
Num.Obs. & \num{30252}\\
\bottomrule
\end{tabular}
\begin{tablenotes}
\item Notes: * $p < 0.1$, ** $p < 0.05$, *** $p < 0.01$. Standard errors clustered at household level are in parentheses. An outcome variable is a dummy of declaration. We control squared age (divided by 100), number of household members, a dummy that indicates having dependents, a dummy that indicates a wage earner, a set of dummies of industry a set of dummies of residential area, and individual and time fixed effects.  To obtain the price elasticity, we calculate implied price elasticities by dividing estimated coefficient on price by sample proportion of claimants.
\end{tablenotes}
\end{threeparttable}
\end{table}
