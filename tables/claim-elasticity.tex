\begin{table}

\caption{First-Price Elasticity of Declaration\label{tab:claim-elasticity}}
\centering
\fontsize{8}{10}\selectfont
\begin{threeparttable}
\begin{tabular}[t]{>{\raggedright\arraybackslash}p{25em}>{\centering\arraybackslash}p{15em}}
\toprule
\multicolumn{1}{c}{ } & \multicolumn{1}{c}{1 = Declaration} \\
\cmidrule(l{3pt}r{3pt}){2-2}
\multicolumn{1}{c}{ } & \multicolumn{1}{c}{FE} \\
\cmidrule(l{3pt}r{3pt}){2-2}
  & (1)\\
\midrule
Applicable price & \num{-0.158}***\\
 & (\num{0.058})\\
Log taxable income & \num{0.261}***\\
 & (\num{0.030})\\
\midrule
\addlinespace[0.3em]
\multicolumn{2}{l}{\textit{Implied price elasticity}}\\
\hspace{1em}Estimate & \num{-1.008}***\\
\hspace{1em} & (\num{0.371})\\
Num.Obs. & \num{23784}\\
\bottomrule
\end{tabular}
\begin{tablenotes}
\item Notes: * $p < 0.1$, ** $p < 0.05$, *** $p < 0.01$. Standard errors clustered at the household level are in parentheses. The outcome variable is a declaration dummy. We control for squared age (divided by 100), number of household members, number of children, number of dependents, a dummy that indicates the highest income in a household, a dummy that indicates that a taxpayer is household head, a dummy that indicates a wage earner, a set of industry dummies, a set of residential area dummies, and individual and time fixed effects. To obtain the price elasticity, we calculate implied price elasticities by dividing the estimated coefficient on price by the sample proportion of claimants.
\end{tablenotes}
\end{threeparttable}
\end{table}
