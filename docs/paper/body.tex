% Options for packages loaded elsewhere
\PassOptionsToPackage{unicode}{hyperref}
\PassOptionsToPackage{hyphens}{url}
%
\documentclass[
  11pt,
  a4paper,
]{article}
\usepackage{amsmath,amssymb}
\usepackage{lmodern}
\usepackage{iftex}
\ifPDFTeX
  \usepackage[T1]{fontenc}
  \usepackage[utf8]{inputenc}
  \usepackage{textcomp} % provide euro and other symbols
\else % if luatex or xetex
  \ifXeTeX
    \usepackage{zxjatype} 
    \usepackage[ipaex]{zxjafont}
    \setromanfont{Times New Roman}
  \fi
  \usepackage{unicode-math}
  \defaultfontfeatures{Scale=MatchLowercase}
  \defaultfontfeatures[\rmfamily]{Ligatures=TeX,Scale=1}
\fi
% Use upquote if available, for straight quotes in verbatim environments
\IfFileExists{upquote.sty}{\usepackage{upquote}}{}
\IfFileExists{microtype.sty}{% use microtype if available
  \usepackage[]{microtype}
  \UseMicrotypeSet[protrusion]{basicmath} % disable protrusion for tt fonts
}{}
\usepackage{xcolor}
\IfFileExists{xurl.sty}{\usepackage{xurl}}{} % add URL line breaks if available
\IfFileExists{bookmark.sty}{\usepackage{bookmark}}{\usepackage{hyperref}}
\hypersetup{
  pdftitle={Estimating Effect of Tax Incentives on Donations Considering Self-Selection of Tax Incentives in South Korea},
  hidelinks,
  pdfcreator={LaTeX via pandoc}}
\urlstyle{same} % disable monospaced font for URLs
\usepackage[left=3cm,right=3cm,top=3cm,bottom=3cm]{geometry}

\usepackage{setspace}
\renewcommand{\baselinestretch}{1.5}
\usepackage{float}

\usepackage{longtable,booktabs,array}
\usepackage{threeparttable, threeparttablex, multirow}
\usepackage{calc} % for calculating minipage widths
% Correct order of tables after \paragraph or \subparagraph
\usepackage{etoolbox}
\makeatletter
\patchcmd\longtable{\par}{\if@noskipsec\mbox{}\fi\par}{}{}
\makeatother
% Allow footnotes in longtable head/foot
\IfFileExists{footnotehyper.sty}{\usepackage{footnotehyper}}{\usepackage{footnote}}
\makesavenoteenv{longtable}
\usepackage{graphicx}
\makeatletter
\def\maxwidth{\ifdim\Gin@nat@width>\linewidth\linewidth\else\Gin@nat@width\fi}
\def\maxheight{\ifdim\Gin@nat@height>\textheight\textheight\else\Gin@nat@height\fi}
\makeatother
% Scale images if necessary, so that they will not overflow the page
% margins by default, and it is still possible to overwrite the defaults
% using explicit options in \includegraphics[width, height, ...]{}
\setkeys{Gin}{width=\maxwidth,height=\maxheight,keepaspectratio}
% Set default figure placement to htbp
\makeatletter
\def\fps@figure{htbp}
\makeatother
\setlength{\emergencystretch}{3em} % prevent overfull lines
\providecommand{\tightlist}{%
  \setlength{\itemsep}{0pt}\setlength{\parskip}{0pt}}
\setcounter{secnumdepth}{5}


\usepackage{booktabs}
\usepackage{longtable}
\usepackage{array}
\usepackage{multirow}
\usepackage{wrapfig}
\usepackage{float}
\usepackage{colortbl}
\usepackage{pdflscape}
\usepackage{tabu}
\usepackage{threeparttable}
\usepackage{threeparttablex}
\usepackage[normalem]{ulem}
\usepackage{makecell}
\usepackage{xcolor}
\ifLuaTeX
  \usepackage{selnolig}  % disable illegal ligatures
\fi

\makeatletter
\def\@fnsymbol#1{\ensuremath{\ifcase#1\or \dagger\or \ddagger\or
   \mathsection\or \mathparagraph\or \|\or **\or \dagger\dagger
   \or \ddagger\ddagger \else\@ctrerr\fi}}
    \makeatother
\title{Estimating Effect of Tax Incentives on Donations Considering Self-Selection of Tax Incentives in South Korea  }
\author{
    Hiroki Kato
  \thanks{Graduate School of Economics, Osaka University, Japan. E-mail: h-kato@econ.osaka-u.ac.jp  }
  \and
    Tsuyoshi Goto
  \thanks{Graduate School of Social Sciences, Chiba University, Japan. E-mail: t.goto@chiba-u.jp  }
  \and
    Youngrok Kim
  \thanks{Graduate School of Economics, Kobe University, Japan.  }
  \and
  }

\date{2022/04/18}


\begin{document}
\begin{spacing}{1}
  \maketitle
\end{spacing}

\hypertarget{intro}{%
\section{Introduction}\label{intro}}

In many countries, governments set a tax relief for charitable giving.
This is because, if subsidizing charitable giving induces a large increase in donations,
it is desirable for public good provision.
As Saez (2004) shows,
it is known that
the elasticity of charitable donations
respect to the giving price relative to the private consumption (giving price, hereafter) is
a key parameter to evaluate the welfare implication.
To receive the tax relief, tax payers have to declare their charitable giving to tax agency.
Then, giving price will be reduced from one to one minus the marginal income tax rate
(if the tax relief is applied through deduction) or one minus the credit rate (if tax credit).

Starting with Feldstein (1975),
there are a large empirical literature examining the giving price elasticity.
Papers in this literature regress the log-valued giving price on
the log-valued amount of charitable giving
to derive the giving price elasticity
using tax filing data
(e.g.~Randolph, 1995; Auten et al., 2002; Fack and Landais, 2010; Bakija and Heim, 2011; Almunia et al., 2020)
or panel survey data (e.g.~Rehavi and Shack, 2013; Yoruk, 2013; Zampelli and Yen, 2016; Backus and Grant, 2019).
Although many of these papers consider the endogeneity issue
such as the endogenous change of the marginal tax rate by the amount of giving,
they pay less attention to the problems caused by the fact that
tax payers have to declare their charitable giving to receive tax relief on charitable giving.
Therefore, if the estimation ignores the declaration of charitable giving,
the self selection bias happens to the estimation
in the sense that the charitable giving is declared by
only those who can expect benefits from the charitable giving or
those who have the giving preference.
To our knowledge,
no paper in the literature considers the self selection
about the declaration of charitable giving.\footnote{Although Almunia et al.~(2020) consider
  the self selection process about the declaration of charitable giving as an exception,
  their reduced-form estimation about the giving price elasticity
  does not consider the self selection.
  They use the giving price elasticities estimated by the reduced-form analysis to
  structurally estimate a fixed compliance cost for declaration,
  where they consider the self selection process.}

Considering that the application of tax relieves is based on self declaration,
the self selection problem is caused by
the existence of compliance costs to apply measures of tax relives
because everyone will apply the measures if there is no compliance cost.
In other words, if there are the compliance costs,
tax payers apply the measures of tax relives only if
their benefits from the measures exceed the compliance costs.
In the literature of tax compliance,
recent papers insist this point and suggest that
the measures of tax relives may not work as
the policy makers expected
since the application of the measures entails compliance costs.
They also suggest compliance costs,
which include record-keeping cost and a fee for accountants,
are considerably high.\footnote{For the individual income tax,
  Benzarti (2020) estimates that the cost of the income tax filing is
  0.6-0.8 percent of adjusted gross income in the U.S..
  For the corporate income tax,
  Zwick(2021) finds that only 37 percent of eligible firms claim
  their refund in their corporate tax returns because of the complexity of tax code.}
In the context of charitable giving,
some papers also suggest the existence of compliance costs to declare charitable giving.
Fack and Landais (2016) study a 1983 reform in France,
which introduced the requirement to submit the certification,
and find the significant reduction of
the declaration of charitable giving after 1983.\footnote{They also show that
  the estimated price elasticity changed before and after the reform.
  This is because the declaration of charitable giving dramatically
  reduced after the reform and they use tax filer data.}
Gillitzer and Skov (2018) examine a 2008 reform in Demmark,
which allowed charities to pre-populate the tax record with the donation record
in order to reduce declaration costs,
and find a large increase of charitable giving declaration after the reform.
Almunia et al.~(2020)
estimate the amount of a fixed compliance cost in the U.K.,
which is estimated as £47.

Although several papers suggest
the existence of compliance costs in the tax compliance literature,
no paper in the literature of giving price elasticity consider the self selection problem.
Therefore, this paper tries to fill this gap and
estimates the giving price elasticity
using the South Korean (Korea, hereafter) survey panel data
called the National Survey of Tax and Benefit (NaSTaB).
The usage of this data brings us to three advantages on the estimation.
Firstly, for the issue of self selection,
we utilize an instrument variable (IV)
which represents the compliance cost for charitable giving declaration.
In the Korean tax system, to receive tax relief,
wage earners can declare tax relief in their company at any time
while self-employed workers have to declare tax relief in the tax agency
at a time of tax return.
Wage earners do not need understand the tax system nor
keep the certification of donation until the tax return season
since their company deals with tax business instead of them
and they can submit the certification at any time.
Self-employed workers, on the other hand, have to understand the tax system
and keep the certification.
Therefore, there is a difference of the compliance cost
between wage earners and self-employed workers.
Whether tax payers are wage earners and self-employed workers is
determined irrespective to their expected benefit from charitable giving or giving preference,
while the different compliance costs between them will make
a difference about the behavior of the declaration.
Instrumented by the wage earner dummy,
the declaration of charitable giving will be less affected
by the benefit from the declaration.
One of the advantages of our IV is that
the compliance costs are not considered to have a direct impact on a giving behavior
although they have a effect on it through the declaration of charitable giving.
Since the difference in compliance cost has no effect on the giving behavior
if there is no declaration system of charitable giving,
tax payers' giving behavior innately have no relation to the compliance cost.
As for another advantage of our IV,
the causality from the compliance cost to the charitable giving declaration
seems credible and their relation is monotone since recent papers
such as Fack and Landais (2016) and Gillitzer and Skov (2018) imply that the existence of compliance costs significantly reduce the declaration of charitable giving.\footnote{In addition,
  another advantage of our setting is that
  we can capture the heterogeneity of compliance costs,
  while all taxpayers experience the change of compliance costs
  in the setting of Fack and Landais (2016) and Gillitzer and Skov (2018).
  (これdiscussionに持ってっても良いかも)}

Secondly, we address the issue of sample selection bias using the panel survey data.
Considering that the giving price is dependent on the declaration of charitable giving
we have to distinguish declared and undeclared charitable giving.
However, tax return data,
which is used in many papers in the literature of giving price elasticity,
do not record undeclared charitable giving and giving by tax withholders.
In addition,
since those who itemize their giving tends to be wealthier than the average taxpayer,
the estimates based on tax filer data will represent the behaviors of high-income earners.
Panel survey data contains information about charitable giving
irrespective of tax filing or giving declaration.
The estimation using panel survey data should be free from
the sample selection bias (Rehavi and Shack, 2013; Backus and Grant, 2019).
Since NaStaB data is constructed by reflecting the Korean society,
we could consider the sample of low-income households,
which are sometimes omitted from the tax filer data.
This is particularly important for the estimation of the price elasticity
in terms of the extensive margin
since the propensity of donation by low-income earners
tend to be less than high-income earners.

Thirdly, to derive the elasticity,
this paper exploits the South Korean tax reform in 2014
as a main identification strategy.
In the 2014 reform,
tax credit was introduced as a way for the tax relief on charitable giving,
though income deduction had been used before.
The 2014 tax reform started to allow 15\% of the total amount of charitable giving
as a tax credit for all declaring taxpayers,
which means that the giving price for 1 KRW donation is 0.85 KRW.\footnote{1 KRW is approximately 0.001 USD. In other words, 1 USD is about 1,000 KRW.}
Since the giving price was determined according to the marginal tax rate of
progressive income tax before 2014,
this tax reform reduced the giving price for low income taxpayers
while it increased the price for high income tax payers.
Since the variation of giving price can be considered to be exogenous for taxpayers,
we exploit this reform and conduct the difference-in-difference (DID) analysis
for identification.\footnote{To our knowledge,
  only Hong(2020) considers the policy change(s) in Korea using panel data
  and derives the estimated giving price elasticity as -1.2.
  However, he does not use DID nor consider the self selection bias.}

Using the IV representing the compliance cost,
the estimated giving price elasticities are in the range between XXX and XXX
in terms of intensive-margins.
If the IV is not used,
we find that the estimated intensive-margins giving price elasticity is XXX,
which is similar value to the estimates in the existing literature.
Considering that those who can expect benefits from the charitable giving and
who have the giving preference are less affected by the change of the giving price,
OLS estimates may overestimate their effects.
As for the extensive-margins giving price elasticity,
the estimates with the IV are in the range between XXX and XXX,
while the estimate without the IV is XXX.
The extensive margins elasticity shows
whether tax payers donate or not depending on the giving price.
Since those who stop donating will not declare their donation and
their giving price will increase to one,
if the endogeneity of declaration is not considered,
the estimates will be elastic and it will be interpreted
as if the donation is stopped by the increase of the giving price.
The usage of the IV prevents a possibility of such a reverse causality.

In addition,
the estimation of the intensive-margins and extensive-margins giving price elasticity considers
well-known challenges for the estimations in this literature
such as the endogenous change of the marginal tax rate by the amount of giving,
simultaneous determination of income and donation and the announcement effect of policy change.
We examine all of these issues in the robustness check and
find that the intensive-margin tax-price elasticities are in the range of -2 and -1.5
and the extensive-margin tax-price elasticities are in the range of -5 and -1.7
in the almost all cases of the estimations.

This paper consists of six sections.
Section 2 and 3 respectively explain the institutional background and data.
Section 4 explains the estimation method.
Section 5 deals with the analysis of price elasticity.
Section 6 concludes.

\hypertarget{background}{%
\subsection{Institutional background and Sources of endogeneity}\label{background}}

In this section,
we describe the income tax relief for charitable giving in Korea
and the endogeneity issues arising from the declaration of giving.

\hypertarget{system-of-tax-relief-on-charitable-giving}{%
\subsection{System of tax relief on charitable giving}\label{system-of-tax-relief-on-charitable-giving}}

To explain how income tax relief on charitable giving is applied and
how the giving price is determined,
we introduce a abstract framework in this subsection.
Thus, let us consider that a household with pre-tax income \(y_i\)
has a choice between private consumption \(x_i\) and charitable giving \(g_i\).
Their budget constraint can be shown as
\begin{align}
x_i + g_i = y_i − R_iK_i - R_iT(y_i, g_i) - (1-R_i)T(y_i). \label{eq:budget}
\end{align}
\(T\) is tax amount which depends on the pre-tax income and charitable giving.
Since the marginal income tax rate is progressive in Korea,
we assume \(T(\cdot)\) satisfy \(T_y(\cdot)>0\) and \(T_{yy}(\cdot)>0\) hereafter,
where the subscript means the partial differentiation.
\(R_i\) is the dummy which takes 1 if \(i\) declares the tax relief and 0 otherwise.
\(K_i\) is a fixed compliance cost for the declaration of charitable giving.
While \(K_i\) includes non-monetary cost such as record-keeping cost and
monetary cost such as a fee for accounts,
it is converted to pecuniary terms for simplicity.
In the literature about the giving price elasticity,
since most of papers implicitly assume \(R_i=1\) and \(K_i=0\).

Tax payers can reduce their income tax payment by declaring their giving.
However, the declaration entails the fixed compliance cost \(K_i\).
Therefore, whether tax payers declare their giving or not is determined according to
\begin{align}
  R_i = \begin{cases}
      1 \text{ if }T(y_i, g_i) + K_i<T(y_i)\\
      0 \text{ if }T(y_i, g_i) + K_i\ge T(y_i). \label{eq:R}
  \end{cases}
\end{align}
When tax payers do not declare their giving, the amount of income tax will be \(T(y_i)\).
When tax payers declare their giving, the amount of tax is
\begin{align}
  T(y_i, g_i) =
  \begin{cases} 
      T(y_i-g_i)&\text{ if tax relief is applied by income deduction.}\\
      T(y_i)-mg_i &\text{ if tax relief is applied by tax credit.} \label{eq:relief}
  \end{cases}
\end{align}

To derive the giving price,
let us differentiate the budget constraint \eqref{eq:budget} by \(x_i\) and \(g_i\).
Then, it derives that the giving price
(relative to the private consumption) is
\(\frac{dx_i}{dg_i}=1+R_iT_g(y_i,g_i)\),
which we denote \(p\).
Therefore, according to \eqref{eq:relief},
the giving price \(p\) is
\begin{align}
    p=
    \begin{cases} 
        1-R_iT'(y_i-g_i)&\text{ if tax relief is applied by income deduction.}\\
        1-R_im&\text{ if tax relief is applied by tax credit.}\label{giving price}
    \end{cases}
\end{align}
Hereafter,
let us \(q\) to show the amount of tax relief for each declared giving
(i.e.~\(p=1-R_iq\) and \(q=-T_g(y_i, g_i)\)).
The government can change \(q\) by the tax reform as the next subsection explains.

\begin{table}

\caption{\label{tab:mtr}Marginal Income Tax Rate}
\centering
\fontsize{9}{11}\selectfont
\begin{threeparttable}
\begin{tabular}[t]{lccccccc}
\toprule
Income/Year & 2008 & 2009 & 2010 \textasciitilde{} 2011 & 2012 \textasciitilde{} 2013 & 2014 \textasciitilde{} 2016 & 2017 & 2018\\
\midrule
(A) \textasciitilde{} 1200 & 8\% & 6\% & 6\% & 6\% & 6\% & 6\% & 6\%\\
\cmidrule{1-8}
(B) 1200 \textasciitilde{} 4600 & 17\% & 16\% & 15\% & 15\% & 15\% & 15\% & 15\%\\
\cmidrule{1-8}
(C) 4600 \textasciitilde{} 8800 & 26\% & 25\% & 24\% & 24\% & 24\% & 24\% & 24\%\\
\cmidrule{1-8}
(D) 8800 \textasciitilde{} 15000 &  &  &  &  & 35\% &  & 35\%\\
\cmidrule{1-1}
\cmidrule{6-6}
\cmidrule{8-8}
(E) 15000 \textasciitilde{} 30000 &  &  &  & \multirow{-2}{*}{\centering\arraybackslash 35\%} &  & \multirow{-2}{*}{\centering\arraybackslash 35\%} & 38\%\\
\cmidrule{1-1}
\cmidrule{5-5}
\cmidrule{7-8}
(F) 30000 \textasciitilde{} 50000 &  &  &  &  &  & 38\% & 40\%\\
\cmidrule{1-1}
\cmidrule{7-8}
(G) 50000 \textasciitilde{} & \multirow{-4}{*}{\centering\arraybackslash 35\%} & \multirow{-4}{*}{\centering\arraybackslash 35\%} & \multirow{-4}{*}{\centering\arraybackslash 35\%} & \multirow{-2}{*}{\centering\arraybackslash 38\%} & \multirow{-3}{*}{\centering\arraybackslash 38\%} & 40\% & 42\%\\
\bottomrule
\end{tabular}
\begin{tablenotes}
\item Notes: Marginal income tax rates applied from 2008 to 2018 are summarized.
  The income level is shown in terms of 10,000 KRW,
  which is approximately 10 United States dollars (USD)
  at an exchange rate of 1,000 KRW to one USD.
\end{tablenotes}
\end{threeparttable}
\end{table}

\hypertarget{taxreform}{%
\subsection{Korean tax system and tax reform in 2014}\label{taxreform}}

Korean tax system offers a tax relief for charitable giving in income tax.
To mitigate the administrative cost,
the Korean National Tax Service introduce different taxation methods
and different ways of giving declaration for wage earners and self-employed workers.
Wage earners pay income tax by tax withholding and
can declare their giving via their company at anytime
since, instead of them, their company is supposed to
close the comprehensive income tax return including giving declaration
through year-end settlement.
Self-employed workers have to calculate the amount of income
earned during a year and pay income tax through tax return by May of the following year.
To receive tax relief on charitable giving,
they have to submit the certificate of donations when they submit tax return.
Therefore, there is a difference of compliance cost of tax relief
since self-employed workers have to understand tax system to precisely populate tax return
and retain the certificate until they submit tax return
although wage earners need not to understand tax system and
can submit the certificate at any time.

Before 2014, tax relief on charitable giving is conducted by income deduction in Korea.
Since the marginal income tax rate was progressively determined as Table \ref{tab:mtr},
tax payer facing the higher marginal income tax rate
can enjoy the lower giving price for each 1 KRW of donation,
i.e.~the giving price was regressive before 2014.
In 2014, aiming at the relaxation of regressivity of giving price,
the Korean government reformed tax system,
where the tax credit was introduced instead of income deduction.
Since then, 15\% of the total amount of charitable giving has been allowed as a tax credit,
which means that the giving price from 2014 is 0.85 KRW
for each 1 KRW of donation irrelevant to the income level.
Therefore, compared to tax credit system,
the high income household,
whose (average) income tax rate is more than 15\%,
get benefit from charitable giving under the income deduction system.
However, middle or low income households would enjoy tax relief in tax credit system
more than income deduction system.
We exploit the variation of giving price brought from the policy change
as a main identification source to estimate the giving price elasticity.

\hypertarget{nastab}{%
\section{National Survey of Tax and Benefit (NaSTaB)}\label{nastab}}

\hypertarget{estimation}{%
\section{Empirical Strategy}\label{estimation}}

\hypertarget{result}{%
\section{Estimation Results}\label{result}}

\hypertarget{references}{%
\section*{References}\label{references}}
\addcontentsline{toc}{section}{References}

\end{document}
