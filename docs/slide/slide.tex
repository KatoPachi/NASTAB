% Options for packages loaded elsewhere
\PassOptionsToPackage{unicode}{hyperref}
\PassOptionsToPackage{hyphens}{url}
%
\documentclass[
  ignorenonframetext,
  aspectratio=169,
]{beamer}
\usepackage{pgfpages}
\setbeamertemplate{caption}[numbered]
\setbeamertemplate{caption label separator}{: }
\setbeamertemplate{navigation symbols}{}
\setbeamertemplate{footline}[page number]
\setbeamertemplate{itemize item}{\small\raise0.5pt\hbox{$\bullet$}}
\setbeamertemplate{itemize subitem}{\tiny\raise1.5pt\hbox{$\bullet$}}
\setbeamertemplate{itemize subsubitem}{\tiny\raise1.5pt\hbox{$\bullet$}}
\setbeamercolor{caption name}{fg=normal text.fg}
\beamertemplatenavigationsymbolsempty
% Prevent slide breaks in the middle of a paragraph
\widowpenalties 1 10000
\raggedbottom
\setbeamertemplate{part page}{
  \centering
  \begin{beamercolorbox}[sep=16pt,center]{part title}
    \usebeamerfont{part title}\insertpart\par
  \end{beamercolorbox}
}
\setbeamertemplate{section page}{
  \centering
  \begin{beamercolorbox}[sep=12pt,center]{part title}
    \usebeamerfont{section title}\insertsection\par
  \end{beamercolorbox}
}
\setbeamertemplate{subsection page}{
  \centering
  \begin{beamercolorbox}[sep=8pt,center]{part title}
    \usebeamerfont{subsection title}\insertsubsection\par
  \end{beamercolorbox}
}
\AtBeginPart{
  \frame{\partpage}
}
\AtBeginSection{
  \ifbibliography
  \else
    \frame{\sectionpage}
  \fi
}
\AtBeginSubsection{
  \frame{\subsectionpage}
}
\usepackage{amsmath,amssymb}
\usepackage{lmodern}
\usepackage{iftex}
\ifPDFTeX
  \usepackage[T1]{fontenc}
  \usepackage[utf8]{inputenc}
  \usepackage{textcomp} % provide euro and other symbols
\else % if luatex or xetex
  \ifXeTeX
    \usepackage{xltxtra} 
    \usepackage{xeCJK}
    \setCJKmainfont{ipaexm.ttf}
    \setCJKsansfont{ipaexg.ttf}
    \setCJKmonofont{ipaexg.ttf}
  \fi
  \usepackage{unicode-math}
  \defaultfontfeatures{Scale=MatchLowercase}
  \defaultfontfeatures[\rmfamily]{Ligatures=TeX,Scale=1}
\fi
% Use upquote if available, for straight quotes in verbatim environments
\IfFileExists{upquote.sty}{\usepackage{upquote}}{}
\IfFileExists{microtype.sty}{% use microtype if available
  \usepackage[]{microtype}
  \UseMicrotypeSet[protrusion]{basicmath} % disable protrusion for tt fonts
}{}
\makeatletter
\@ifundefined{KOMAClassName}{% if non-KOMA class
  \IfFileExists{parskip.sty}{%
    \usepackage{parskip}
  }{% else
    \setlength{\parindent}{0pt}
    \setlength{\parskip}{6pt plus 2pt minus 1pt}}
}{% if KOMA class
  \KOMAoptions{parskip=half}}
\makeatother
\usepackage{xcolor}
\IfFileExists{xurl.sty}{\usepackage{xurl}}{} % add URL line breaks if available
\IfFileExists{bookmark.sty}{\usepackage{bookmark}}{\usepackage{hyperref}}
\hypersetup{
  pdftitle={Estimating Effect of Tax Incentives on Charitable Giving Considering Self-Selection of Tax Relief in South Korea},
  hidelinks,
  pdfcreator={LaTeX via pandoc}}
\urlstyle{same} % disable monospaced font for URLs

\usepackage{setspace}
\usepackage{float}

\newif\ifbibliography
\usepackage{longtable,booktabs,array}
\usepackage{threeparttable, threeparttablex, multirow}
\usepackage{calc} % for calculating minipage widths
\usepackage{caption}
% Make caption package work with longtable
\makeatletter
\def\fnum@table{\tablename~\thetable}
\makeatother
\usepackage{graphicx}
\makeatletter
\def\maxwidth{\ifdim\Gin@nat@width>\linewidth\linewidth\else\Gin@nat@width\fi}
\def\maxheight{\ifdim\Gin@nat@height>\textheight\textheight\else\Gin@nat@height\fi}
\makeatother
% Scale images if necessary, so that they will not overflow the page
% margins by default, and it is still possible to overwrite the defaults
% using explicit options in \includegraphics[width, height, ...]{}
\setkeys{Gin}{width=\maxwidth,height=\maxheight,keepaspectratio}
% Set default figure placement to htbp
\makeatletter
\def\fps@figure{htbp}
\makeatother
\setlength{\emergencystretch}{3em} % prevent overfull lines
\providecommand{\tightlist}{%
  \setlength{\itemsep}{0pt}\setlength{\parskip}{0pt}}
\setcounter{secnumdepth}{-\maxdimen} % remove section numbering


\usepackage{siunitx}
\ifLuaTeX
  \usepackage{selnolig}  % disable illegal ligatures
\fi

\title{Estimating Effect of Tax Incentives on Charitable Giving Considering Self-Selection of Tax Relief in South Korea  }
\author[shortname]{ Hiroki Kato \inst{1} \and  Tsuyoshi Goto \inst{2} \and  Yong-Rok Kim \inst{3} \and }
\institute[shortinst]{ \inst{1} Osaka University \and  \inst{2} Chiba University \and  \inst{3} Kansai University \and }

\date{2022/04/19}


\begin{document}
\frame{\titlepage}

\begin{frame}{Tax Incentives on Donations}
\protect\hypertarget{tax-incentives-on-donations}{}
\begin{itemize}
\tightlist
\item
  Governments set a tax relief for charitable giving

  \begin{itemize}
  \tightlist
  \item
    if subsidizing charitable giving induces a large increase in donations, it is desirable for public good provision.
  \end{itemize}
\item
  Key parameter to evaluate social welfare: the price elasticity of charitable donations (Saez, 2004)

  \begin{itemize}
  \tightlist
  \item
    giving price: relative price to the private consumption
  \end{itemize}
\end{itemize}
\end{frame}

\begin{frame}{Literatures: Price Elasticity of Giving}
\protect\hypertarget{literatures-price-elasticity-of-giving}{}
\begin{itemize}
\tightlist
\item
  Large empirical literatures examining the giving price elasticity estimate log-log demand function to derive the giving price elasticity
\item
  Two types of data

  \begin{enumerate}
  \tightlist
  \item
    tax filing data: Randolph, 1995; Auten et al., 2002; Fack and Landais, 2010; Bakija and Heim, 2011; Almunia et al., 2020
  \item
    panel survey data: Rehavi and Shack, 2013; Yoruk, 2013; Zampelli and Yen, 2016; Backus and Grant, 2019
  \end{enumerate}
\item
  Issue: Although many of these papers consider the endogeneity issue such as the endogenous change of the marginal tax rate by the amount of giving, they pay less attention to the problems caused by the fact that \color{red}{tax payers have to declare their charitable giving to receive tax relief on charitable giving}.
\end{itemize}
\end{frame}

\begin{frame}{Literatures: Tax Compliance}
\protect\hypertarget{literatures-tax-compliance}{}
\begin{itemize}
\tightlist
\item
  Existence of compliance costs to apply measures of tax relives because everyone will apply the measures if there is no compliance cost.

  \begin{itemize}
  \tightlist
  \item
    tax payers apply the measures of tax relives only if their benefits from the measures exceed the compliance costs.
  \end{itemize}
\item
  Recent papers insist this point and suggest that the measures of tax relives may not work as the policy makers expected
\item
  They also suggest compliance costs (e.g.~record-keeping cost and a fee for accountants) are considerably high.

  \begin{itemize}
  \tightlist
  \item
    individual income tax (Benzarti, 2020); corporate income tax (Zwick, 2021); charitable giving (Fack and Landais, 2016;Gillitzer and Skov, 2018; Almunia et al., 2020)
  \end{itemize}
\end{itemize}
\end{frame}

\begin{frame}{What Our Paper Did}
\protect\hypertarget{what-our-paper-did}{}
\begin{itemize}
\tightlist
\item
  We bridge price elasticity of giving and self-selection of tax incentive

  \begin{itemize}
  \tightlist
  \item
    As long as we know, there is no paper in the literature of giving price elasticity consider the self selection problem
  \end{itemize}
\item
  We estimate the giving price elasticity using the South Korean (Korea, hereafter) survey panel data called the National Survey of Tax and Benefit (NaSTaB)
\item
  Why South Korea?

  \begin{enumerate}
  \tightlist
  \item
    Compliance costs for wage earners and self-employed workers are different (IV)
  \item
    We could consider the sample of low-income households, which are sometimes omitted from the tax filer data.
  \item
    We can exploit the South Korean tax reform in 2014 as a main identification strategy of price change (income deduction \(\to\) tax credit)
  \end{enumerate}
\end{itemize}
\end{frame}

\begin{frame}{What Our Paper Found}
\protect\hypertarget{what-our-paper-found}{}
Using the IV representing the compliance cost,

\begin{enumerate}
\tightlist
\item
  Intensive-margin price elasticities are in the range between \(XXX\) and \(XXX\)

  \begin{itemize}
  \tightlist
  \item
    FE model w/o IV: \(XXX\) (similar value to the estimates in the existing literature)
  \end{itemize}
\item
  Extensive-margin price elasticities are in the range between \(YYY\) and \(YYY\)

  \begin{itemize}
  \tightlist
  \item
    FE model w/o IV: \(YYY\)
  \end{itemize}
\end{enumerate}

We examine well-known issues in the robustness check

\begin{itemize}
\tightlist
\item
  intensive-margin tax-price elasticities are in the range of -2 and -1.5
\item
  extensive-margin tax-price elasticities are in the range of -5 and -1.7
\end{itemize}
\end{frame}

\hypertarget{background}{%
\section{Institutional background and Sources of Endogeneity}\label{background}}

\begin{frame}{Framework}
\protect\hypertarget{framework}{}
Consider that a household with pre-tax income \(y_i\) has a choice between private consumption \(x_i\) and charitable giving \(g_i\).

Their budget constraint can be shown as
\begin{align}
x_i + g_i = y_i - R_iK_i - R_iT(y_i, g_i) - (1-R_i)T(y_i). \label{eq:budget}
\end{align}

\begin{itemize}
\tightlist
\item
  \(T\) is tax amount which depends on the pre-tax income and charitable giving. Assume \(T_y(\cdot)>0\) and \(T_{yy}(\cdot)>0\).
\item
  \(R_i\) is the dummy which takes 1 if \(i\) declares the tax relief and 0 otherwise.
\item
  \(K_i\) is a fixed compliance cost for the declaration of charitable giving.

  \begin{itemize}
  \tightlist
  \item
    In the literature about the giving price elasticity, most of papers implicitly assume \(R_i=1\) and \(K_i=0\).
  \end{itemize}
\end{itemize}
\end{frame}

\begin{frame}{Decision Rule of Tax Relief}
\protect\hypertarget{decision-rule-of-tax-relief}{}
\begin{align}
  R_i = \begin{cases}
      1 \text{ if }T(y_i, g_i) + K_i<T(y_i)\\
      0 \text{ if }T(y_i, g_i) + K_i\ge T(y_i). \label{eq:R}
  \end{cases}
\end{align}

where

\begin{align}
  T(y_i, g_i) =
  \begin{cases} 
      T(y_i-g_i)&\text{ if tax relief is applied by income deduction.}\\
      T(y_i)-mg_i &\text{ if tax relief is applied by tax credit.} \label{eq:relief}
  \end{cases}
\end{align}
\end{frame}

\begin{frame}{Giving Price}
\protect\hypertarget{giving-price}{}
Let us differentiate the budget constraint \eqref{eq:budget} by \(x_i\) and \(g_i\).
Then, it derives that the giving price (relative to the private consumption) is
\(\frac{dx_i}{dg_i}=1+R_iT_g(y_i,g_i)\), which we denote \(p\).

\begin{align}
    p=
    \begin{cases} 
        1-R_iT'(y_i-g_i)&\text{ if tax relief is applied by income deduction.}\\
        1-R_im&\text{ if tax relief is applied by tax credit.}\label{giving price}
    \end{cases}
\end{align}
Hereafter,
let us \(q\) to show the amount of tax relief for each declared giving
(i.e.~\(p=1-R_iq\) and \(q=-T_g(y_i, g_i)\)).
The government can change \(q\) by the tax reform.
\end{frame}

\begin{frame}{Marginal Tax Rate}
\protect\hypertarget{marginal-tax-rate}{}
\begin{table}

\caption{\label{tab:mtr}Marginal Income Tax Rate}
\centering
\fontsize{9}{11}\selectfont
\begin{threeparttable}
\begin{tabular}[t]{lccccccc}
\toprule
Income/Year & 2008 & 2009 & 2010 \textasciitilde{} 2011 & 2012 \textasciitilde{} 2013 & 2014 \textasciitilde{} 2016 & 2017 & 2018\\
\midrule
(A) \textasciitilde{} 1200 & 8\% & 6\% & 6\% & 6\% & 6\% & 6\% & 6\%\\
\cmidrule{1-8}
(B) 1200 \textasciitilde{} 4600 & 17\% & 16\% & 15\% & 15\% & 15\% & 15\% & 15\%\\
\cmidrule{1-8}
(C) 4600 \textasciitilde{} 8800 & 26\% & 25\% & 24\% & 24\% & 24\% & 24\% & 24\%\\
\cmidrule{1-8}
(D) 8800 \textasciitilde{} 15000 &  &  &  &  & 35\% &  & 35\%\\
\cmidrule{1-1}
\cmidrule{6-6}
\cmidrule{8-8}
(E) 15000 \textasciitilde{} 30000 &  &  &  & \multirow{-2}{*}{\centering\arraybackslash 35\%} &  & \multirow{-2}{*}{\centering\arraybackslash 35\%} & 38\%\\
\cmidrule{1-1}
\cmidrule{5-5}
\cmidrule{7-8}
(F) 30000 \textasciitilde{} 50000 &  &  &  &  &  & 38\% & 40\%\\
\cmidrule{1-1}
\cmidrule{7-8}
(G) 50000 \textasciitilde{} & \multirow{-4}{*}{\centering\arraybackslash 35\%} & \multirow{-4}{*}{\centering\arraybackslash 35\%} & \multirow{-4}{*}{\centering\arraybackslash 35\%} & \multirow{-2}{*}{\centering\arraybackslash 38\%} & \multirow{-3}{*}{\centering\arraybackslash 38\%} & 40\% & 42\%\\
\bottomrule
\end{tabular}
\begin{tablenotes}
\item Notes: Marginal income tax rates applied from 2008 to 2018 are summarized.
  The income level is shown in terms of 10,000 KRW,
  which is approximately 10 United States dollars (USD)
  at an exchange rate of 1,000 KRW to one USD.
\end{tablenotes}
\end{threeparttable}
\end{table}
\end{frame}

\begin{frame}{Korean Tax System (1)}
\protect\hypertarget{korean-tax-system-1}{}
\begin{itemize}
\tightlist
\item
  To mitigate the administrative cost, the Korean National Tax Service introduce different taxation methods and different ways of giving declaration for wage earners and self-employed workers.
\item
  There is a difference of compliance cost of tax relief between self-employed and wage earners

  \begin{itemize}
  \tightlist
  \item
    self-employed workers have to understand tax system to precisely populate tax return and retain the certificate until they submit tax return
  \item
    wage earners need not to understand tax system and can submit the certificate at any time.
  \end{itemize}
\end{itemize}
\end{frame}

\begin{frame}{Korean Tax System (2)}
\protect\hypertarget{korean-tax-system-2}{}
Wage earners:

\begin{itemize}
\tightlist
\item
  Wage earners pay income tax by tax withholding and can declare their giving via their company at anytime
\item
  Instead of them, their company is supposed to close the comprehensive income tax return including giving declaration through year-end settlement
\end{itemize}

Self-employed workers:

\begin{itemize}
\tightlist
\item
  Self-employed workers have to calculate the amount of income earned during a year and pay income tax through tax return by May of the following year.
\item
  To receive tax relief on charitable giving, they have to submit the certificate of donations when they submit tax return.
\end{itemize}
\end{frame}

\begin{frame}{2014 Tax Reform (1)}
\protect\hypertarget{tax-reform-1}{}
\begin{itemize}
\tightlist
\item
  Before 2014, tax relief on charitable giving is conducted by income deduction in Korea.

  \begin{itemize}
  \tightlist
  \item
    tax payer facing the higher marginal income tax rate can enjoy the lower giving price for each 1 KRW of donation
  \end{itemize}
\item
  In 2014, aiming at the relaxation of regressivity of giving price, the Korean government reformed tax system, where the tax credit was introduced instead of income deduction.

  \begin{itemize}
  \tightlist
  \item
    15\% of the total amount of charitable giving has been allowed as a tax credit,
  \end{itemize}
\end{itemize}
\end{frame}

\begin{frame}{2014 Tax Reform (2)}
\protect\hypertarget{tax-reform-2}{}
\begin{itemize}
\tightlist
\item
  Compared to tax credit system, the high income household, whose (average) income tax rate is more than 15\%,
  get benefit from charitable giving under the income deduction system.
\item
  However, middle or low income households would enjoy tax relief in tax credit system more than income deduction system.
\item
  We exploit the variation of giving price brought from the policy change as a main identification source to estimate the giving price elasticity.
\end{itemize}
\end{frame}

\hypertarget{nastab}{%
\section{Data: National Survey of Tax and Benefit (NaSTaB)}\label{nastab}}

\begin{frame}{About NaSTaB}
\protect\hypertarget{about-nastab}{}
\begin{itemize}
\tightlist
\item
  NaSTaB is annual panel data conducted by
  Korea Institute of Taxation and Finance
\item
  The survey will be administered to 5,634 households
  from across the country

  \begin{itemize}
  \tightlist
  \item
    5,634 heads of household and
    economically active household members aged 15 and older
    complete the survey
  \end{itemize}
\item
  Our study uses data from (1) 2013 to 2018 and
\end{itemize}

\begin{enumerate}
[(1)]
\setcounter{enumi}{1}
\tightlist
\item
  excluding respondents under the age of 23
\end{enumerate}

\begin{itemize}
\tightlist
\item
  This is because we focus on the 2014 tax reform
\end{itemize}
\end{frame}

\begin{frame}{Descriptive Statistics}
\protect\hypertarget{descriptive-statistics}{}
\begin{table}
\centering
\fontsize{9}{11}\selectfont
\begin{tabular}[t]{lccc}
\toprule
  & N & Mean & Std.Dev.\\
\midrule
\addlinespace[0.3em]
\multicolumn{4}{l}{\textbf{Income and giving price}}\\
\hspace{1em}Annual taxable labor income (unit: 10,000KRW) & 36189 & \num{1747.26} & \num{2696.77}\\
\hspace{1em}First giving relative price & 36198 & \num{0.86} & \num{0.04}\\
\addlinespace[0.3em]
\multicolumn{4}{l}{\textbf{Charitable giving}}\\
\hspace{1em}Annual chariatable giving (unit: 10,000KRW) & 36199 & \num{35.64} & \num{153.20}\\
\hspace{1em}Dummary of donation > 0 & 36199 & \num{0.24} & \num{0.42}\\
\hspace{1em}Dummy of declaration of a tax relief & 36199 & \num{0.10} & \num{0.30}\\
\addlinespace[0.3em]
\multicolumn{4}{l}{\textbf{Individual Characteristics}}\\
\hspace{1em}Age & 36199 & \num{53.45} & \num{16.22}\\
\hspace{1em}Female dummy & 36199 & \num{0.43} & \num{0.50}\\
\hspace{1em}University graduate & 36198 & \num{0.42} & \num{0.49}\\
\hspace{1em}High school graduate dummy & 36198 & \num{0.31} & \num{0.46}\\
\hspace{1em}Junior high school graduate dummy & 36198 & \num{0.27} & \num{0.44}\\
\hspace{1em}Wage earner dummy & 27394 & \num{0.56} & \num{0.50}\\
\bottomrule
\end{tabular}
\end{table}
\end{frame}

\begin{frame}{Income Distribution and Giving Price}
\protect\hypertarget{income-distribution-and-giving-price}{}
\begin{figure}[t]

{\centering \includegraphics[width=0.7\linewidth,]{C:/Users/vge00/Desktop/NaSTaB/docs/slide/slide_files/figure-beamer/SummaryPrice-1} 

}

\caption{Income Distribution in 2013 and Relative Giving Price. Notes: The left and right axis measure the relative frequency of respondents (grey bars) and the relative giving price (solid step line and dashed line), respectively. A solid step line and a dashed horizontal line represents the giving price in 2013 and 2014, respectively.}\label{fig:SummaryPrice}
\end{figure}
\end{frame}

\begin{frame}{Identification of Price Elasticity}
\protect\hypertarget{identification-of-price-elasticity}{}
We can create three income groups based on
a change of tax incentive due to 2014 tax reform.

\begin{enumerate}
\tightlist
\item
  \(<\) 120 million KRW

  \begin{itemize}
  \tightlist
  \item
    expand tax incentive (reduce giving price)
  \end{itemize}
\item
  \([\text{120 million KRW}, \text{460 million KRW}]\)

  \begin{itemize}
  \tightlist
  \item
    tax incentive did not change
  \end{itemize}
\item
  \(>\) 460 million KRW

  \begin{itemize}
  \tightlist
  \item
    shrink tax incentive (increase giving price)
  \end{itemize}
\end{enumerate}
\end{frame}

\begin{frame}{Summary of Giving Behavior}
\protect\hypertarget{summary-of-giving-behavior}{}
\begin{figure}[t]

{\centering \includegraphics[width=0.7\linewidth,]{C:/Users/vge00/Desktop/NaSTaB/docs/slide/slide_files/figure-beamer/SummaryGiving-1} 

}

\caption{Proportion of Donors and Average Donations among Donors. Notes: The left and right axises measure prooortion of donors (grey bars) and the average amount of donations among donors (solid line), respectively.}\label{fig:SummaryGiving}
\end{figure}
\end{frame}

\begin{frame}{Summary of Giving Amount by Three Income Groups}
\protect\hypertarget{summary-of-giving-amount-by-three-income-groups}{}
\begin{figure}[t]

{\centering \includegraphics[width=0.7\linewidth,]{C:/Users/vge00/Desktop/NaSTaB/docs/slide/slide_files/figure-beamer/SummaryGivingOverall-1} 

}

\caption{Average Logged Giving by Three Income Groups. Notes: We created three income groups, with the relative price of giving rising (circle), unchanged (triangle), and falling (square) between 2013 and 2014. The group averages are normalized to be zero in 2013.}\label{fig:SummaryGivingOverall}
\end{figure}
\end{frame}

\begin{frame}{Summary of Giving Amount by Three Income Groups (Conditional on Donors)}
\protect\hypertarget{summary-of-giving-amount-by-three-income-groups-conditional-on-donors}{}
\begin{figure}[t]

{\centering \includegraphics[width=0.7\linewidth,]{C:/Users/vge00/Desktop/NaSTaB/docs/slide/slide_files/figure-beamer/SummaryGivingIntensive-1} 

}

\caption{Average Logged Giving by Three Income Groups Conditional on Donors. Notes: We created three income groups, with the relative price of giving rising (circle), unchanged (triangle), and falling (square) between 2013 and 2014. The group averages are normalized to be zero in 2013.}\label{fig:SummaryGivingIntensive}
\end{figure}
\end{frame}

\begin{frame}{Summary of Proportion of Donors by Three Income Groups}
\protect\hypertarget{summary-of-proportion-of-donors-by-three-income-groups}{}
\begin{figure}[t]

{\centering \includegraphics[width=0.7\linewidth,]{C:/Users/vge00/Desktop/NaSTaB/docs/slide/slide_files/figure-beamer/SummaryGivingExtensive-1} 

}

\caption{Proportion of Donors by Three Income Groups. Notes: We created three income groups, with the relative price of giving rising (circle), unchanged (triangle), and falling (square) between 2013 and 2014. The group averages are normalized to be zero in 2013.}\label{fig:SummaryGivingExtensive}
\end{figure}
\end{frame}

\begin{frame}{Distribution of Giving Amount by Application of Tax Relief}
\protect\hypertarget{distribution-of-giving-amount-by-application-of-tax-relief}{}
\begin{figure}[t]

{\centering \includegraphics[width=0.7\linewidth,]{C:/Users/vge00/Desktop/NaSTaB/docs/slide/slide_files/figure-beamer/SummaryGivingIntensiveDist-1} 

}

\caption{Estimated Distribution of Charitable Giving among Donors in Each Year}\label{fig:SummaryGivingIntensiveDist}
\end{figure}
\end{frame}

\begin{frame}{Compliance Rate by Wage Earners or Not}
\protect\hypertarget{compliance-rate-by-wage-earners-or-not}{}
\begin{figure}[t]

{\centering \includegraphics[width=0.7\linewidth,]{C:/Users/vge00/Desktop/NaSTaB/docs/slide/slide_files/figure-beamer/SummaryReliefbyEarner-1} 

}

\caption{Share of Tax Relief by Wage Earners. Notes: A solid line is the share of applying for tax relief among wage eaners. A dashed line is the share of applying for tax relief other than wage earners.}\label{fig:SummaryReliefbyEarner}
\end{figure}
\end{frame}

\begin{frame}{Compliance Rate by Wage Earners or Not (Conditional on Donors)}
\protect\hypertarget{compliance-rate-by-wage-earners-or-not-conditional-on-donors}{}
\begin{figure}[t]

{\centering \includegraphics[width=0.7\linewidth,]{C:/Users/vge00/Desktop/NaSTaB/docs/slide/slide_files/figure-beamer/SummaryReliefbyEarner2-1} 

}

\caption{Share of Tax Relief by Wage Earners Conditional on Donors. Notes: A solid line is the share of applying for tax relief among wage eaners. A dashed line is the share of applying for tax relief other than wage earners.}\label{fig:SummaryReliefbyEarner2}
\end{figure}
\end{frame}

\end{document}
