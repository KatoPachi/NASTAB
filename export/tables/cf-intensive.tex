\begin{table}

\caption{Estimation Results of Control Function Model (Intensive-margin Sample)\label{tab:cf-intensive}}
\centering
\fontsize{8}{10}\selectfont
\begin{threeparttable}
\begin{tabular}[t]{>{\raggedright\arraybackslash}p{25em}>{\centering\arraybackslash}p{15em}}
\toprule
\multicolumn{1}{c}{ } & \multicolumn{1}{c}{Log donation} \\
\cmidrule(l{3pt}r{3pt}){2-2}
  & (1)\\
\midrule
Effective price ($\beta_e$) & \num{-1.233}***\\
 & (\num{0.347})\\
Log income & \num{2.300}**\\
 & (\num{1.159})\\
Residuals of Application ($\psi_1$) & \num{-0.196}**\\
 & (\num{0.082})\\
\midrule
Num.Obs. & \num{7993}\\
R2 Adj. & \num{0.090}\\
\bottomrule
\end{tabular}
\begin{tablenotes}
\item Notes: * p < 0.1, ** p < 0.05, *** p < 0.01. Standard errors clustered at household level are in parentheses. An outcome variable is logged value of amount of charitable giving. For estimation, we use only donors (intensive-margin sample). We control squared age (divided by 100), number of household members, a dummy that indicates having dependents, a set of dummies of industry, a set of dummies of residential area, and time fixed effects. We use an wage earner dummy as an instrument to obtain residuals of application. Instead individual fixed effects, we control a vector of individual-level sample mean of all exogenous variables including instruments (Chamberlain-Mundlak device).
\end{tablenotes}
\end{threeparttable}
\end{table}
