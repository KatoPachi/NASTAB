\begin{table}

\caption{(\#tab:sufficient-stats)Overall Price Elasticities Using Extensive-Margin Elasticities}
\centering
\fontsize{8}{10}\selectfont
\begin{threeparttable}
\begin{tabular}[t]{>{\raggedright\arraybackslash}p{20em}>{\centering\arraybackslash}p{10em}>{\centering\arraybackslash}p{10em}}
\toprule
\multicolumn{1}{c}{ } & \multicolumn{2}{c}{Log donation} \\
\cmidrule(l{3pt}r{3pt}){2-3}
\multicolumn{1}{c}{ } & \multicolumn{1}{c}{FE} & \multicolumn{1}{c}{FE-2SLS} \\
\cmidrule(l{3pt}r{3pt}){2-2} \cmidrule(l{3pt}r{3pt}){3-3}
  & (1) & (2)\\
\midrule
Simulated price ($\beta_a$) & \num{-1.268}** & \\
 & (\num{0.558}) & \\
Actual price ($\beta^{IV}_e$) &  & \num{-4.076}**\\
 &  & (\num{1.712})\\
\midrule
Extensive-margin values ($x$) & \num{-0.511} & \num{-1.477}\\
Num.Obs. & \num{17014} & \num{17014}\\
\bottomrule
\end{tabular}
\begin{tablenotes}
\item Notes: * $p < 0.1$, ** $p < 0.05$, *** $p < 0.01$. Standard errors clustered at the household level are in parentheses. We only use wage earners. In column (1)--(3), an outcome variable is the logged value of donation. We first normalize an amount of donation so that 1 corresponds to the value of the minimum nonzero donation in the data (2,000KRW). Then, we then transform an amount of donation to $m(g)$, where $m(y) = \log(y)$ for $y > 0$ and $m(0) = -x$, where $x$ is extensive-margin elasticities. Covariates consist of logged disposal income, squared age (divided by 100), number of household members, number of children, number of dependents, a dummy that indicates the highest income in a household, a dummy that indicates that a taxpayer is household head, a set of industry dummies, a set of residential area dummies, and individual and time fixed effects. The excluded instrument is a logged value of simulated price in column (2).
\end{tablenotes}
\end{threeparttable}
\end{table}
