\begin{table}

\caption{Estimation of Tax-Price Elasticity for Applicants\label{tab:fe2sls-applicants}}
\centering
\fontsize{8}{10}\selectfont
\begin{threeparttable}
\begin{tabular}[t]{lcccc}
\toprule
\multicolumn{1}{c}{ } & \multicolumn{4}{c}{Log donation} \\
\cmidrule(l{3pt}r{3pt}){2-5}
\multicolumn{1}{c}{ } & \multicolumn{2}{c}{Fixed-effect model} & \multicolumn{2}{c}{Pooled model with sample selection correction} \\
\cmidrule(l{3pt}r{3pt}){2-3} \cmidrule(l{3pt}r{3pt}){4-5}
  & (1) & (2) & (3) & (4)\\
\midrule
Log applicable price & \num{-1.184}** &  & \num{-0.861} & \\
 & (\num{0.504}) &  & (\num{0.653}) & \\
Log effective last-price &  & \num{-1.264}** &  & \num{-0.931}\\
 &  & (\num{0.542}) &  & (\num{0.710})\\
Log income & \num{-1.348} & \num{-1.391} & \num{-2.085} & \num{-2.164}\\
 & (\num{1.621}) & (\num{1.622}) & (\num{1.579}) & (\num{1.595})\\
\midrule
F-statistics of instrument &  & 16441.55 &  & 15232.24\\
Num.Obs. & \num{3574} & \num{3574} & \num{3574} & \num{3574}\\
FE: pid & X & X &  & \\
FE: year & X & X & X & X\\
\bottomrule
\end{tabular}
\begin{tablenotes}
\item Notes: * p < 0.1, ** p < 0.05, *** p < 0.01. Standard errors are clustered at household level. We use only applicants of tax deduction (or tax credit). Fixed effect models (1)--(2) control squared age (divided by 100), number of household members, a dummy that indicates having dependents, a set of dummies of industry, a set of dummies of residential area, and individual and time fixed effects. Models (3)--(4) correct sample selection bias, proposed by \cite{Semykina2010} which is analogous to the control function approach. These models additionally control the inverse mills ratio (IMR) and interactions of IMR with time dummies, and use within-mean of explanatory variables as individual fixed effects. Model (2) and (4) are 2SLS where log appricable price is an instrument of log effective last-price.
\end{tablenotes}
\end{threeparttable}
\end{table}
