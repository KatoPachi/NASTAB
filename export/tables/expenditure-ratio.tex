\begin{table}

\caption{(\#tab:expenditure-ratio)\label{tab:expenditure-ratio}Elasticity of Expenditure Ratio of Donation}
\centering
\fontsize{9}{11}\selectfont
\begin{threeparttable}
\begin{tabular}[t]{l>{\centering\arraybackslash}p{5em}>{\centering\arraybackslash}p{5em}>{\centering\arraybackslash}p{5em}}
\toprule
\multicolumn{1}{c}{ } & \multicolumn{3}{c}{Donation expenditure ratio} \\
\cmidrule(l{3pt}r{3pt}){2-4}
\multicolumn{1}{c}{ } & \multicolumn{2}{c}{FE} & \multicolumn{1}{c}{FE-2SLS} \\
\cmidrule(l{3pt}r{3pt}){2-3} \cmidrule(l{3pt}r{3pt}){4-4}
  & (1) & (2) & (3)\\
\midrule
Simulated price ($\beta_S$) & \num{-0.007}* &  & \\
 & (\num{0.004}) &  & \\
Actual price ($\beta^{FE}_A$) &  & \num{-0.082}*** & \\
 &  & (\num{0.005}) & \\
Actual price ($\beta^{IV}_A$) &  &  & \num{-0.021}*\\
 &  &  & (\num{0.012})\\
Log disposal income & \num{0.001}*** & \num{0.000} & \num{0.001}***\\
 & (\num{0.000}) & (\num{0.000}) & (\num{0.000})\\
\midrule
\addlinespace[0.3em]
\multicolumn{4}{l}{\textit{Implied price elasticity}}\\
\hspace{1em}Estimate & \num{-0.971}* & \num{-10.967}*** & \num{-2.788}*\\
\hspace{1em} & (\num{0.572}) & (\num{0.691}) & (\num{1.622})\\
\addlinespace[0.3em]
\multicolumn{4}{l}{\textit{1st stage information (Excluded instrument: Simulated price)}}\\
\hspace{1em}F-statistics of instrument &  &  & \num{1029.071}\\
Num.Obs. & \num{17047} & \num{17047} & \num{17047}\\
\bottomrule
\end{tabular}
\begin{tablenotes}
\item Notes: * $p < 0.1$, ** $p < 0.05$, *** $p < 0.01$. Standard errors clustered at the household level are in parentheses. We only use wage earners. In column (1)--(3), an outcome variable is donation expenditure ratio, $E = g/y$, where $g$ is donation amount and $y$ is after-tax income. The price elasticity can be obtained by $\hat{\beta}/E$, where $\hat{\beta}$ is estimate. Thus, we calculated the implied price elasticities as $\hat{\beta}/\bar{E}$, where $\bar{E}$ is sample average of $E$ ($0.007$ in our data). In addition to logged income, covariates consist of squared age (divided by 100), number of household members, number of children, number of dependents, a dummy that indicates the highest income in a household, a dummy that indicates that a taxpayer is household head, a set of industry dummies, a set of residential area dummies, and individual and time fixed effects. The excluded instrument is a logged value of simulated price in columns (3) and (6).
\end{tablenotes}
\end{threeparttable}
\end{table}
