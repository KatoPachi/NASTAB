% Options for packages loaded elsewhere
\PassOptionsToPackage{unicode}{hyperref}
\PassOptionsToPackage{hyphens}{url}
%
\documentclass[
  ignorenonframetext,
  aspectratio=169,
]{beamer}
\usepackage{pgfpages}
\setbeamertemplate{caption}[numbered]
\setbeamertemplate{caption label separator}{: }
\setbeamertemplate{navigation symbols}{}
\setbeamertemplate{footline}[page number]
\setbeamertemplate{itemize item}{\small\raise0.5pt\hbox{$\bullet$}}
\setbeamertemplate{itemize subitem}{\tiny\raise1.5pt\hbox{$\bullet$}}
\setbeamertemplate{itemize subsubitem}{\tiny\raise1.5pt\hbox{$\bullet$}}
\setbeamercolor{caption name}{fg=normal text.fg}
\beamertemplatenavigationsymbolsempty
% Prevent slide breaks in the middle of a paragraph
\widowpenalties 1 10000
\raggedbottom
\setbeamertemplate{part page}{
  \centering
  \begin{beamercolorbox}[sep=16pt,center]{part title}
    \usebeamerfont{part title}\insertpart\par
  \end{beamercolorbox}
}
\setbeamertemplate{section page}{
  \centering
  \begin{beamercolorbox}[sep=12pt,center]{part title}
    \usebeamerfont{section title}\insertsection\par
  \end{beamercolorbox}
}
\setbeamertemplate{subsection page}{
  \centering
  \begin{beamercolorbox}[sep=8pt,center]{part title}
    \usebeamerfont{subsection title}\insertsubsection\par
  \end{beamercolorbox}
}
\AtBeginPart{
  \frame{\partpage}
}
\AtBeginSection{
  \ifbibliography
  \else
    \frame{\sectionpage}
  \fi
}
\AtBeginSubsection{
  \frame{\subsectionpage}
}
\usepackage{amsmath,amssymb}
\usepackage{lmodern}
\usepackage{iftex}
\ifPDFTeX
  \usepackage[T1]{fontenc}
  \usepackage[utf8]{inputenc}
  \usepackage{textcomp} % provide euro and other symbols
\else % if luatex or xetex
  \ifXeTeX
    \usepackage{xltxtra} 
    \usepackage{xeCJK}
    \setCJKmainfont{ipaexm.ttf}
    \setCJKsansfont{ipaexg.ttf}
    \setCJKmonofont{ipaexg.ttf}
  \fi
  \usepackage{unicode-math}
  \defaultfontfeatures{Scale=MatchLowercase}
  \defaultfontfeatures[\rmfamily]{Ligatures=TeX,Scale=1}
\fi
% Use upquote if available, for straight quotes in verbatim environments
\IfFileExists{upquote.sty}{\usepackage{upquote}}{}
\IfFileExists{microtype.sty}{% use microtype if available
  \usepackage[]{microtype}
  \UseMicrotypeSet[protrusion]{basicmath} % disable protrusion for tt fonts
}{}
\makeatletter
\@ifundefined{KOMAClassName}{% if non-KOMA class
  \IfFileExists{parskip.sty}{%
    \usepackage{parskip}
  }{% else
    \setlength{\parindent}{0pt}
    \setlength{\parskip}{6pt plus 2pt minus 1pt}}
}{% if KOMA class
  \KOMAoptions{parskip=half}}
\makeatother
\usepackage{xcolor}
\IfFileExists{xurl.sty}{\usepackage{xurl}}{} % add URL line breaks if available
\IfFileExists{bookmark.sty}{\usepackage{bookmark}}{\usepackage{hyperref}}
\hypersetup{
  pdftitle={Charitable Giving, Tax Reform, and Self-selection of Tax Report: Evidence from South Korea},
  hidelinks,
  pdfcreator={LaTeX via pandoc}}
\urlstyle{same} % disable monospaced font for URLs

\usepackage{setspace}
\usepackage{float}

\newif\ifbibliography
\usepackage{longtable,booktabs,array}
\usepackage{threeparttable, threeparttablex, multirow}
\usepackage{calc} % for calculating minipage widths
\usepackage{caption}
% Make caption package work with longtable
\makeatletter
\def\fnum@table{\tablename~\thetable}
\makeatother
\usepackage{graphicx}
\makeatletter
\def\maxwidth{\ifdim\Gin@nat@width>\linewidth\linewidth\else\Gin@nat@width\fi}
\def\maxheight{\ifdim\Gin@nat@height>\textheight\textheight\else\Gin@nat@height\fi}
\makeatother
% Scale images if necessary, so that they will not overflow the page
% margins by default, and it is still possible to overwrite the defaults
% using explicit options in \includegraphics[width, height, ...]{}
\setkeys{Gin}{width=\maxwidth,height=\maxheight,keepaspectratio}
% Set default figure placement to htbp
\makeatletter
\def\fps@figure{htbp}
\makeatother
\setlength{\emergencystretch}{3em} % prevent overfull lines
\providecommand{\tightlist}{%
  \setlength{\itemsep}{0pt}\setlength{\parskip}{0pt}}
\setcounter{secnumdepth}{-\maxdimen} % remove section numbering


\ifLuaTeX
  \usepackage{selnolig}  % disable illegal ligatures
\fi

\title{Charitable Giving, Tax Reform, and Self-selection of Tax Report: Evidence from South Korea  }
\author[shortname]{ Hiroki Kato \inst{1} \and  Tsuyoshi Goto \inst{2} \and  Yong-Rok Kim \inst{3} \and }
\institute[shortinst]{ \inst{1} Osaka University \and  \inst{2} Chiba University \and  \inst{3} Kansai University \and }

\date{2021/11/23}


\begin{document}
\frame{\titlepage}

\begin{frame}{Introduction}
\protect\hypertarget{introduction}{}
\begin{itemize}
\tightlist
\item
  In many countries, tax relief for charitable giving are implemented.
\item
  The elasticity of giving tax relief is known as a key parameter to evaluate the welfare implication (e.g.~Saez, 2004).

  \begin{itemize}
  \tightlist
  \item
    Intuitively, if the elasticity is more than 1 in absolute value, \$1 of tax relief make more than \$1 of charitable giving.
  \end{itemize}
\item
  Many papers investigate the elasticity based on tax return data (e.g.~Almunia et al., 2020, Auten et al., 2002, and so on).
\end{itemize}
\end{frame}

\begin{frame}{Introduction}
\protect\hypertarget{introduction-1}{}
\begin{itemize}
\tightlist
\item
  However, the tax return data record only the declared charitable giving.

  \begin{itemize}
  \tightlist
  \item
    First issue: Actual donations is different from declared donations.(Fack and Landais (2016); Gillitzer and Skov (2018))
  \item
    We use panel survey data in South Korea to deal with this issue.
  \end{itemize}
\item
  Tax payers decide the amount of donation and whether to declare tax relief based on the size of tax incentive and declaration cost.

  \begin{itemize}
  \tightlist
  \item
    Second issue: Neglect of this declaration cost may bias the estimations of elasticity.
  \item
    We use instrumental variable (IV) and control function approach for this issue.
  \end{itemize}
\item
  Based on DID as an identification strategy, we investigate the giving price elasticity of South Korea.
\end{itemize}
\end{frame}

\begin{frame}{Introduction}
\protect\hypertarget{introduction-2}{}
Result

\begin{enumerate}
\tightlist
\item
  Baseline results show that the giving price elasticity is less than -1.4 in terms of intensive margins and less than -1.7 in terms of extensive margins in Korea.
\item
  The estimated giving price elasticity for those who declare charitable giving is around -1.2\(\sim\)-1.6.
\end{enumerate}

These estimates are more elastic than the estimates in the extant research, many of which show around -1.

\begin{enumerate}
\setcounter{enumi}{2}
\tightlist
\item
  (Not in the paper) The estimated declaration cost of giving is KRW (\$).
\item
  (Not in the paper) Given our estimates, increasing the subsidy on charitable giving will be desirable in Korea.
\end{enumerate}
\end{frame}

\begin{frame}{2014 tax reform in South Korea}
\protect\hypertarget{tax-reform-in-south-korea}{}
In Korea, income tax payers can receive tax relief for their charitable giving.

\begin{itemize}
\tightlist
\item
  For the application of tax relief, tax payers have to submit a certificate for charitable giving.
\item
  Wage earners pay their income tax by withholding tax and declare their charitable giving via their company.

  \begin{itemize}
  \tightlist
  \item
    Wage earners can submit the certificate at any time.
  \end{itemize}
\item
  Non wage earners, such as the self-employed, pay their income tax by tax-return and declare their charitable giving via the National Tax Service.

  \begin{itemize}
  \tightlist
  \item
    Non wage earners have to retain the certificate until they submit tax return.
  \end{itemize}
\end{itemize}
\end{frame}

\begin{frame}{2014 tax reform in South Korea}
\protect\hypertarget{tax-reform-in-south-korea-1}{}
Our major price variation comes from the 2014 tax reform.

\begin{itemize}
\tightlist
\item
  Before 2014, tax deduction (所得控除) was used for tax relief on charitable giving.

  \begin{itemize}
  \tightlist
  \item
    I.e. the giving price depended on income level.
  \end{itemize}
\item
  After 2014, tax credit (税額控除) started to be used for tax relief on charitable giving.

  \begin{itemize}
  \tightlist
  \item
    The tax credit rate was determined as 15\%.
  \item
    Giving price is 0.85, irrespective of income level.
  \end{itemize}
\end{itemize}
\end{frame}

\begin{frame}{2014 tax reform in South Korea}
\protect\hypertarget{tax-reform-in-south-korea-2}{}
Model

\begin{itemize}
\tightlist
\item
  Consider private consumption (\(x_{i}\)) and charitable giving (\(g_{i}\)).
\item
  The budget constraint is
\end{itemize}

\[x_{i} + g_{i} = y_{i} - R_iK- R_iT(y_{i}, g_{i})-(1-R_i)T(y_i)\]

where \(y_{i}\) is pre-tax total income, \(R_{i}\) is a dummy of declaration of tax relief and \(T(y_i)\) and \(T(y_{i}, g_{i})\) are respectively the amount of tax when \(i\) does not declare tax relief and when \(i\) declares tax relief.

\begin{itemize}
\tightlist
\item
  Tax payers declare their charitable giving if its benefit exceeds its cost.
  \begin{align}
  R_i=\begin{cases}
  1 \text{ if }T(y_i, g_i) - T(y_i)>K\\
  0 \text{ if }T(y_i, g_i) - T(y_i)\le K.\notag
  \end{cases}
  \end{align}
\end{itemize}
\end{frame}

\begin{frame}{2014 tax reform in South Korea}
\protect\hypertarget{tax-reform-in-south-korea-3}{}
\textbf{Tax deduction system (until 2013)}
\[T(y_{i}, g_{i}) = T(y_{i} - g_{i})\]

\begin{itemize}
\tightlist
\item
  In 2012 and 2013, the marginal tax rate was the same, though it was different from ones before 2011.
\item
  The logged relative giving price is \(R_{i}\ln(1 - T'(y_{i} - g_{i}))\).
\end{itemize}

\textbf{Tax credit system (from 2014)}
\[T(y_{i}, g_{i}) = T(y_{i}) - R_{i} m g_{i}\]

\begin{itemize}
\tightlist
\item
  \(m\) is tax credit rate and is \(m = 0.15\).
\item
  The logged relative giving price is \(R_{i}\ln(1 - 0.15) = R_{i}\ln 0.85\).
\end{itemize}

Note: The logged relative giving price for the non-declared is \(\ln1 =0\).
\end{frame}

\begin{frame}{Source of endogeneity}
\protect\hypertarget{source-of-endogeneity}{}
\begin{enumerate}
\tightlist
\item
  Usage of tax return data only captures declared charitable giving.

  \begin{itemize}
  \tightlist
  \item
    If the charitable giving is not declared, tax relief has a little effect.
  \item
    Some papers use survey data to deal with this problem (e.g.~Rehavi and Shack, 2013).
  \item
    \textbf{Following them, we use survey panel data of Korea.}
  \end{itemize}
\item
  If the declaration cost is ignored, the estimation should be biased.

  \begin{itemize}
  \tightlist
  \item
    The giving price depends not only on marginal tax rate and tax credit rate, but also on the declaration behavior.
  \item
    As far as we know, only Almunia et al.~(2020) deal with this problem, though they used tax return data.
  \item
    \textbf{We use the different declaration cost btw wage earners and the others as an instrumental variable (IV).}
  \end{itemize}
\end{enumerate}
\end{frame}

\hypertarget{references}{%
\section*{References}\label{references}}
\addcontentsline{toc}{section}{References}

\begin{frame}{References}
\hypertarget{refs_main}{}
\end{frame}

\end{document}
