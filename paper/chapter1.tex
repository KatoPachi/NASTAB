
\documentclass[ review  , 3p ]{elsarticle}
%default = preprint (single sapce), review = doublespace
%detail class option: https://www.elsevier.com/__data/assets/pdf_file/0008/56843/elsdoc-1.pdf

% eliminate "Preprinted to Elsevier"
\makeatletter
\def\ps@pprintTitle{%
 \let\@oddhead\@empty
 \let\@evenhead\@empty
 \def\@oddfoot{\centerline{\thepage}}%
 \let\@evenfoot\@oddfoot}
\makeatother

%%% Begin My package additions %%%%%%%%%%%%%%%%%%%
\usepackage[hyphens]{url}



\usepackage{lineno} % add
\providecommand{\tightlist}{%
  \setlength{\itemsep}{0pt}\setlength{\parskip}{0pt}}

\usepackage{graphicx}

\usepackage{zxjatype}
\usepackage{xeCJK}
\setCJKmainfont{ipaexm.ttf}
\setCJKsansfont{ipaexg.ttf}
\setCJKmonofont{ipaexg.ttf}

\usepackage{color}

\usepackage{booktabs}
\usepackage{longtable}
\usepackage{array}
\usepackage{multirow}
\usepackage{wrapfig}
\usepackage{float}
\usepackage{colortbl}
\usepackage{pdflscape}
\usepackage{tabu}
\usepackage{threeparttable}
\usepackage{threeparttablex}
\usepackage[normalem]{ulem}
\usepackage{makecell}
\usepackage{xcolor}


%\usepackage{xpatch}
%\xpatchcmd{\MaketitleBox}{\hrule}{}{}{}% remove first horizontal rule (above abstract)
%\xpatchcmd{\MaketitleBox}{\hrule}{}{}{}% remoce second horizonral rule (below keywords)
%%%%%%%%%%%%%%%% end my additions to header

\usepackage[T1]{fontenc}
\usepackage{lmodern}
\usepackage{amssymb,amsmath}
\usepackage{ifxetex,ifluatex}
\usepackage{fixltx2e} % provides \textsubscript
% use upquote if available, for straight quotes in verbatim environments
\IfFileExists{upquote.sty}{\usepackage{upquote}}{}
\ifnum 0\ifxetex 1\fi\ifluatex 1\fi=0 % if pdftex
  \usepackage[utf8]{inputenc}
\else % if luatex or xelatex
  \usepackage{fontspec}
  \ifxetex
    \usepackage{xltxtra,xunicode}
  \fi
  \defaultfontfeatures{Mapping=tex-text,Scale=MatchLowercase}
  \newcommand{\euro}{€}
\fi
% use microtype if available
\IfFileExists{microtype.sty}{\usepackage{microtype}}{}
\bibliographystyle{elsarticle-harvard}
\usepackage{tabularx}
\ifxetex
  \usepackage[setpagesize=false, % page size defined by xetex
              unicode=false, % unicode breaks when used with xetex
              xetex]{hyperref}
\else
  \usepackage[unicode=true]{hyperref}
\fi
\hypersetup{breaklinks=true,
            bookmarks=true,
            pdfauthor={},
            pdftitle={Short Paper},
            colorlinks=false,
            urlcolor=blue,
            linkcolor=magenta,
            pdfborder={0 0 0}}
\urlstyle{same}  % don't use monospace font for urls

\setcounter{secnumdepth}{5}

\newlength{\cslhangindent}
\setlength{\cslhangindent}{1.5em}
\newlength{\csllabelwidth}
\setlength{\csllabelwidth}{3em}
\newenvironment{CSLReferences}[3] % #1 hanging-ident, #2 entry spacing
 {% don't indent paragraphs
  \setlength{\parindent}{0pt}
  % turn on hanging indent if param 1 is 1
  \ifodd #1 \everypar{\setlength{\hangindent}{\cslhangindent}}\ignorespaces\fi
  % set entry spacing
  \ifnum #2 > 0
  \setlength{\parskip}{#2\baselineskip}
  \fi
 }%
 {}
\usepackage{calc} % for \widthof, \maxof
\newcommand{\CSLBlock}[1]{#1\hfill\break}
\newcommand{\CSLLeftMargin}[1]{\parbox[t]{\maxof{\widthof{#1}}{\csllabelwidth}}{#1}}
\newcommand{\CSLRightInline}[1]{\parbox[t]{\linewidth}{#1}}
\newcommand{\CSLIndent}[1]{\hspace{\cslhangindent}#1}

% Pandoc toggle for numbering sections (defaults to be off)


% Pandoc header


\begin{document}
  \begin{frontmatter}

    \title{Charitable Giving, Tax Reform, and Government Efficiency\tnoteref{1}}
            \tnotetext[1]{This research is base on}
                \author[Osaka University]{
      Hiroki Kato 
       \corref{*} }
     \ead{vge008kh@stundent.econ.osaka-u.ac.jp}   %to avoid auto-link, use \@ instead of @
        \author[Chiba University]{
      Tsuyoshi Goto 
      }
      %to avoid auto-link, use \@ instead of @
        \author[Kobe University]{
      Yong-Rok Kim 
      }
      %to avoid auto-link, use \@ instead of @
            \address[Osaka University]{Graduate School of Economics, Osaka University, Japan}
        \address[Chiba University]{Graduate School of Economics, Chiba University, Japan}
        \address[Kobe University]{Graduate School of Economics, Kobe University, Japan}
            \cortext[*]{Corresponding Author.}
      
        \begin{abstract}
      Brah
    \end{abstract}
      
        \begin{keyword}
      Charitable giving, Giving price, Tax reform, Governement efficiency, South Korea
       \JEL{D91, I10, I18} 
    \end{keyword}
    
  \end{frontmatter}

  \hypertarget{introduction}{%
  \section{Introduction}\label{introduction}}
  
  \hypertarget{charitable-giving-and-taxiation}{%
  \subsection{Charitable Giving and Taxiation}\label{charitable-giving-and-taxiation}}
  
  In many countries, governments set a tax relief for charitable giving.
  This is because, if subsidizing charitable giving induces a large increase in donations, it is desirable. (Bursztyn and Jensen, 2017)
  
  To evaluate the effect of tax relief, it is needed to know the elasticity of charitable donations with respect to their tax price.
  
  \hypertarget{charitable-giving-and-taxiation-1}{%
  \subsection{Charitable Giving and Taxiation}\label{charitable-giving-and-taxiation-1}}
  
  In addition to the tax price charitable donations, the donations may be affected by people's perception towards the government.
  
  This is because the works and missions of private charity often mirror or overlap with one of governments, and the charity is not needed if the government adequately satisfies the needs of society.
  
  Thus, the different perception towards the government may make the different behavior for charitable giving.
  
  We investigate the relation between tax price elasticity of charitable donations and the different perception towards the government using the South Korean dataset.
  
  \hypertarget{south-korean-tax-reform}{%
  \subsection{South Korean tax reform}\label{south-korean-tax-reform}}
  
  We can utilize the effect of the 2014 tax reform in the South Korea.
  
  \begin{itemize}
  \tightlist
  \item
    Before 2014, tax deduction was adopted to subsidize charitable donation behavior.
  \item
    After 2014, tax credit have been adopted.
  \end{itemize}
  
  The main difference is that tax credits reduce taxes directly, while tax deductions indirectly lower the tax burden by decreasing the marginal tax rate, which increases with gross income.
  
  In addition, the dataset contains the information about perception towards the government.
  
  \hypertarget{related-literature}{%
  \subsection{Related Literature}\label{related-literature}}
  
  This study mainly relates to the two strands of studies.
  
  \begin{enumerate}
  \def\labelenumi{\arabic{enumi}.}
  \tightlist
  \item
    Research about tax price elasticity of charitable donations
  \item
    Research about perception towards the government and donation/tax payment.
  \end{enumerate}
  
  \hypertarget{research-about-tax-price-elasticity-of-charitable-donations}{%
  \subsection{Research about tax price elasticity of charitable donations}\label{research-about-tax-price-elasticity-of-charitable-donations}}
  
  Papers in this strand examines the price and income elasticity of charitable donations using the tax deduction applied for donation.
  The estimated price elasticities vary, but the typical one is said as -1 (Andreoni and Payne, 2013).
  
  \begin{itemize}
  \tightlist
  \item
    Auten et al.(2002): -0.79\textasciitilde{}-1.26 (the U.S.)
  \item
    Fack and Landais(2010): -0.15\textasciitilde{}-0.57 (France)
  \item
    Bakija and Heim (2011): -0.61\textasciitilde{}-1.1 (the U.S.)
  \item
    Duquette (2016): -2.15\textasciitilde{}-5.01 (the U.S.)
  \item
    Almunia et al.(2020): -0.24\textasciitilde{}--1.5 (the U.K.)
  \end{itemize}
  
  The study in non-Western country, where the culture of donation may be different, is few.
  Thus, we firstly examine the elasticity of giving in Korea.
  
  \hypertarget{research-about-perception-towards-the-government-and-donationtax-payment.}{%
  \subsection{Research about perception towards the government and donation/tax payment.}\label{research-about-perception-towards-the-government-and-donationtax-payment.}}
  
  Experimental studies show that the giving behavior may be affected by perception towards the government.
  
  \begin{itemize}
  \tightlist
  \item
    Li et al.(2011) suggest that governmental organizations collect less donation than private charities though they have the same mission and work.
  \item
    Sheremeta and Uler(2020) show that individuals provide public good reacting the wasteful spending of government.
  \end{itemize}
  
  This may be because people with distrust in government think that
  
  \begin{enumerate}
  \def\labelenumi{\arabic{enumi}.}
  \tightlist
  \item
    the direct donation is more efficient than public service provision or
  \item
    people can directly allocate and control their funds by donation, unlike public service provision.
  \end{enumerate}
  
  Thus, people having the different trust in the government would have different elasticities of giving.
  
  \hypertarget{data}{%
  \section{Data}\label{data}}
  
  Placeholder
  
  \hypertarget{national-survey-of-tax-and-benefit-nastab}{%
  \subsection{National Survey of Tax and Benefit (NaSTaB)}\label{national-survey-of-tax-and-benefit-nastab}}
  
  \hypertarget{time-series-of-chariable-giving}{%
  \subsection{Time Series of Chariable Giving}\label{time-series-of-chariable-giving}}
  
  \hypertarget{summary-statistics-of-covariates}{%
  \subsection{Summary Statistics of Covariates}\label{summary-statistics-of-covariates}}
  
  \hypertarget{summary-statistics-of-covariates-contd}{%
  \subsection{Summary Statistics of Covariates (Cont'd)}\label{summary-statistics-of-covariates-contd}}
  
  \hypertarget{what-is-giving-price}{%
  \subsection{What is Giving Price?}\label{what-is-giving-price}}
  
  \hypertarget{determination-of-tax-amount}{%
  \subsection{Determination of Tax Amount}\label{determination-of-tax-amount}}
  
  \hypertarget{derive-giving-price}{%
  \subsection{Derive Giving Price}\label{derive-giving-price}}
  
  \hypertarget{construct-giving-price}{%
  \subsection{Construct Giving Price}\label{construct-giving-price}}
  
  \hypertarget{income-distribution-and-giving-price}{%
  \subsection{Income Distribution and Giving Price}\label{income-distribution-and-giving-price}}
  
  \hypertarget{time-series-of-average-donations-by-benefit-group}{%
  \subsection{Time Series of Average Donations By Benefit Group}\label{time-series-of-average-donations-by-benefit-group}}
  
  \hypertarget{time-series-of-extensive-margin-by-benfit-group}{%
  \subsection{Time Series of Extensive Margin by Benfit Group}\label{time-series-of-extensive-margin-by-benfit-group}}
  
  \hypertarget{time-series-of-intensive-margin-by-benfit-group}{%
  \subsection{Time Series of Intensive Margin by Benfit Group}\label{time-series-of-intensive-margin-by-benfit-group}}
  
  \hypertarget{empirical-strategy}{%
  \subsection{Empirical Strategy}\label{empirical-strategy}}
  
  \hypertarget{intensive-margin-and-extensive-margin}{%
  \subsection{Intensive Margin and Extensive Margin}\label{intensive-margin-and-extensive-margin}}
  
  \hypertarget{main-results}{%
  \section{Main Results}\label{main-results}}
  
  Placeholder
  
  \hypertarget{price-and-income-elasticity}{%
  \subsection{Price and Income Elasticity}\label{price-and-income-elasticity}}
  
  \hypertarget{baseline-regressions-result}{%
  \subsection{Baseline Regressions: Result}\label{baseline-regressions-result}}
  
  \hypertarget{intensive-margin-and-extensive-margin-result}{%
  \subsection{Intensive Margin and Extensive Margin: Result}\label{intensive-margin-and-extensive-margin-result}}
  
  \hypertarget{robustness-check}{%
  \subsection{Robustness Check}\label{robustness-check}}
  
  \hypertarget{robustness-check-1}{%
  \subsection{Robustness Check 1}\label{robustness-check-1}}
  
  \hypertarget{robustness-check-1-result}{%
  \subsection{Robustness Check 1: Result}\label{robustness-check-1-result}}
  
  \hypertarget{robustness-check-1-intensive-and-extensive-margin}{%
  \subsection{Robustness Check 1: Intensive and Extensive Margin}\label{robustness-check-1-intensive-and-extensive-margin}}
  
  \hypertarget{robust-check-2}{%
  \subsection{Robust Check 2}\label{robust-check-2}}
  
  \hypertarget{robustness-check-2-result}{%
  \subsection{Robustness Check 2: Result}\label{robustness-check-2-result}}
  
  \hypertarget{robustness-check-2-intensive-and-extensive-margin}{%
  \subsection{Robustness Check 2: Intensive and Extensive Margin}\label{robustness-check-2-intensive-and-extensive-margin}}
  
  \hypertarget{governement-efficient-and-price-elasticity}{%
  \section{Governement Efficient and Price Elasticity}\label{governement-efficient-and-price-elasticity}}
  
  Placeholder
  
  \hypertarget{government-efficiency}{%
  \subsection{Government Efficiency}\label{government-efficiency}}
  
  \hypertarget{construct-efficient-index}{%
  \subsection{Construct Efficient Index}\label{construct-efficient-index}}
  
  \hypertarget{histrogram-of-efficient-index}{%
  \subsection{Histrogram of Efficient Index}\label{histrogram-of-efficient-index}}
  
  \hypertarget{heterogenous-price-elasticity-by-governement-efficiency}{%
  \subsection{Heterogenous Price Elasticity by Governement Efficiency}\label{heterogenous-price-elasticity-by-governement-efficiency}}
  
  \hypertarget{efficient-groups-descriptive-stats}{%
  \subsection{Efficient Groups: Descriptive Stats}\label{efficient-groups-descriptive-stats}}
  
  \hypertarget{efficient-groups-descriptive-statis-extensive-margin}{%
  \subsection{Efficient Groups: Descriptive Statis (Extensive Margin)}\label{efficient-groups-descriptive-statis-extensive-margin}}
  
  \hypertarget{efficient-groups-descriptive-stats-intensive-margin}{%
  \subsection{Efficient Groups: Descriptive Stats (Intensive Margin)}\label{efficient-groups-descriptive-stats-intensive-margin}}
  
  \hypertarget{efficient-groups-estimation-results}{%
  \subsection{Efficient Groups: Estimation Results}\label{efficient-groups-estimation-results}}
  
  \hypertarget{robustness-check-2}{%
  \subsection{Robustness Check}\label{robustness-check-2}}
  
  \hypertarget{robustness-check-1-1}{%
  \subsection{Robustness Check 1}\label{robustness-check-1-1}}
  
  \hypertarget{robustness-check-1-estimation-results}{%
  \subsection{Robustness Check 1: Estimation Results}\label{robustness-check-1-estimation-results}}
  
  \hypertarget{robustness-check-2-1}{%
  \subsection{Robustness Check 2}\label{robustness-check-2-1}}
  
  \hypertarget{robustness-check-2-result-1}{%
  \subsection{Robustness Check 2: Result}\label{robustness-check-2-result-1}}
  
  \hypertarget{robustness-check-2-result-extensive-margin}{%
  \subsection{Robustness Check 2: Result (Extensive Margin)}\label{robustness-check-2-result-extensive-margin}}
  
  \hypertarget{robustness-check-2-result-intensive-margin}{%
  \subsection{Robustness Check 2: Result (Intensive Margin)}\label{robustness-check-2-result-intensive-margin}}
  
  \hypertarget{conclusions}{%
  \section{Conclusions}\label{conclusions}}
  
  \hypertarget{conclusions-1}{%
  \subsection{Conclusions}\label{conclusions-1}}
  
  \clearpage
  
  \hypertarget{references}{%
  \subsection*{References}\label{references}}
  \addcontentsline{toc}{subsection}{References}
  
  \hypertarget{refs}{}
  \leavevmode\hypertarget{ref-Bursztyn2017}{}%
  Bursztyn, L., Jensen, R., 2017. Social image and economic behavior in the field: Identifying, understanding, and shaping social pressure. Annual Review of Economics 9, 131--153. doi:\href{https://doi.org/10.1146/annurev-economics-063016-103625}{10.1146/annurev-economics-063016-103625}

\end{document}


