
\documentclass[ review  , 3p ]{elsarticle}
%default = preprint (single sapce), review = doublespace
%detail class option: https://www.elsevier.com/__data/assets/pdf_file/0008/56843/elsdoc-1.pdf

% eliminate "Preprinted to Elsevier"
\makeatletter
\def\ps@pprintTitle{%
 \let\@oddhead\@empty
 \let\@evenhead\@empty
 \def\@oddfoot{\centerline{\thepage}}%
 \let\@evenfoot\@oddfoot}
\makeatother

%%% Begin My package additions %%%%%%%%%%%%%%%%%%%
\usepackage[hyphens]{url}



\usepackage{lineno} % add
\providecommand{\tightlist}{%
  \setlength{\itemsep}{0pt}\setlength{\parskip}{0pt}}

\usepackage{graphicx}

\usepackage{zxjatype}
\usepackage{xeCJK}
\setCJKmainfont{ipaexm.ttf}
\setCJKsansfont{ipaexg.ttf}
\setCJKmonofont{ipaexg.ttf}

\usepackage{color}

\usepackage{booktabs}
\usepackage{longtable}
\usepackage{array}
\usepackage{multirow}
\usepackage{wrapfig}
\usepackage{float}
\usepackage{colortbl}
\usepackage{pdflscape}
\usepackage{tabu}
\usepackage{threeparttable}
\usepackage{threeparttablex}
\usepackage[normalem]{ulem}
\usepackage{makecell}
\usepackage{xcolor}


%\usepackage{xpatch}
%\xpatchcmd{\MaketitleBox}{\hrule}{}{}{}% remove first horizontal rule (above abstract)
%\xpatchcmd{\MaketitleBox}{\hrule}{}{}{}% remoce second horizonral rule (below keywords)
%%%%%%%%%%%%%%%% end my additions to header

\usepackage[T1]{fontenc}
\usepackage{lmodern}
\usepackage{amssymb,amsmath}
\usepackage{ifxetex,ifluatex}
\usepackage{fixltx2e} % provides \textsubscript
% use upquote if available, for straight quotes in verbatim environments
\IfFileExists{upquote.sty}{\usepackage{upquote}}{}
\ifnum 0\ifxetex 1\fi\ifluatex 1\fi=0 % if pdftex
  \usepackage[utf8]{inputenc}
\else % if luatex or xelatex
  \usepackage{fontspec}
  \ifxetex
    \usepackage{xltxtra,xunicode}
  \fi
  \defaultfontfeatures{Mapping=tex-text,Scale=MatchLowercase}
  \newcommand{\euro}{€}
\fi
% use microtype if available
\IfFileExists{microtype.sty}{\usepackage{microtype}}{}
\bibliographystyle{elsarticle-harvard}
\usepackage{tabularx}
\ifxetex
  \usepackage[setpagesize=false, % page size defined by xetex
              unicode=false, % unicode breaks when used with xetex
              xetex]{hyperref}
\else
  \usepackage[unicode=true]{hyperref}
\fi
\hypersetup{breaklinks=true,
            bookmarks=true,
            pdfauthor={},
            pdftitle={Short Paper},
            colorlinks=false,
            urlcolor=blue,
            linkcolor=magenta,
            pdfborder={0 0 0}}
\urlstyle{same}  % don't use monospace font for urls

\setcounter{secnumdepth}{5}

\newlength{\cslhangindent}
\setlength{\cslhangindent}{1.5em}
\newlength{\csllabelwidth}
\setlength{\csllabelwidth}{3em}
\newenvironment{CSLReferences}[3] % #1 hanging-ident, #2 entry spacing
 {% don't indent paragraphs
  \setlength{\parindent}{0pt}
  % turn on hanging indent if param 1 is 1
  \ifodd #1 \everypar{\setlength{\hangindent}{\cslhangindent}}\ignorespaces\fi
  % set entry spacing
  \ifnum #2 > 0
  \setlength{\parskip}{#2\baselineskip}
  \fi
 }%
 {}
\usepackage{calc} % for \widthof, \maxof
\newcommand{\CSLBlock}[1]{#1\hfill\break}
\newcommand{\CSLLeftMargin}[1]{\parbox[t]{\maxof{\widthof{#1}}{\csllabelwidth}}{#1}}
\newcommand{\CSLRightInline}[1]{\parbox[t]{\linewidth}{#1}}
\newcommand{\CSLIndent}[1]{\hspace{\cslhangindent}#1}

% Pandoc toggle for numbering sections (defaults to be off)


% Pandoc header


\begin{document}
  \begin{frontmatter}

    \title{Charitable Giving, Tax Reform, and Government Efficiency\tnoteref{1}}
            \tnotetext[1]{This research is base on}
                \author[Osaka University]{
      Hiroki Kato 
       \corref{*} }
     \ead{vge008kh@stundent.econ.osaka-u.ac.jp}   %to avoid auto-link, use \@ instead of @
        \author[Chiba University]{
      Tsuyoshi Goto 
      }
      %to avoid auto-link, use \@ instead of @
        \author[Kobe University]{
      Yong-Rok Kim 
      }
      %to avoid auto-link, use \@ instead of @
            \address[Osaka University]{Graduate School of Economics, Osaka University, Japan}
        \address[Chiba University]{Graduate School of Economics, Chiba University, Japan}
        \address[Kobe University]{Graduate School of Economics, Kobe University, Japan}
            \cortext[*]{Corresponding Author.}
      
        \begin{abstract}
      Brah
    \end{abstract}
      
        \begin{keyword}
      Charitable giving, Giving price, Tax reform, Governement efficiency, South Korea
       \JEL{D91, I10, I18} 
    \end{keyword}
    
  \end{frontmatter}

  \hypertarget{introduction}{%
  \section{Introduction}\label{introduction}}
  
  In many countries, governments set a tax relief for charitable giving. This is because, if subsidizing charitable giving induces a large increase in donations, it is desirable for public good provision. To evaluate the effect of tax relief, many papers investigate the elasticity of charitable donations with respect to their tax price (Almunia et al., n.d.; Auten et al., 2002; Bakija and Heim, 2011; Fack and Landais, 2010; Randolph, n.d.). Focusing on the tax deduction or tax credit on the charity, they show that the price elasticity of giving is about -1 or more in terms of absolute value, which means that the tax relief for the charitable giving is good in the sense that 1\% tax relief derives more than 1\% donation.
  
  However, if the government can provide public good more efficiently than the direct donation, the donation may not be preferable because the public good provision via donation would be costly then.
  Moreover, when the government is much more efficient than charities, people may not donate so much even if they have a warm-glow preference. Saez (2004) suggests that the change of the relative price between public good provision by donation and government will change the behavior of people and the price elasticity of donation.
  However, the evaluation about the efficiency of the government is usually subjective and different for people. If someone regard the government as efficient, the perceived relative price of giving would be high for them. Thus, the giving behavior would be affected by the subjective perception towards the government.
  
  Considering these points, this paper investigates (1) the price elasticity of giving and (2) whether the different perception towards the government cause the different giving behavior using South Korean panel data.
  Our first main concern is the price elasticity of charity. South Korea (Korea hereafter) experienced the tax reform in 2014, from when the tax relief on charitable giving was conducted by tax credit, though tax deduction had been used before 2014. Thus, we exploit this tax reform as an exogenous policy change to derive the price elasticity of giving. Since the extant research focus on the tax reform within the scheme of tax deduction or tax credit, this paper firstly deals with the tax reform from tax deduction system to tax credit system.
  Our result classifies that the price elasticity of giving in Korea is -1.07 \textasciitilde{} -1.26, which is within the range of the extant research.
  
  Our second concern is the relationship between the giving behavior and the perception towards the government. As we explained, people feeling administrative inefficiency would consider the direct donation is more efficient and would have more willingness to donate. Using the Korean field data, we investigate this and show that the amount of donation is not different between those who regard government as inefficient and the others, though the giving price elasticity of the former is more elastic than the latter. This means that those who think of government as inefficient have more willingness to donate for 1\% reduction of giving price.
  
  This paper contributes two strands of charitable giving literature: the elasticity of charitable donations with respect to their tax price and the perception of government's inefficiency. The examples of papers in the first strand are (Almunia et al., n.d.; Auten et al., 2002; Bakija and Heim, 2011; Fack and Landais, 2010; Randolph, n.d.). They typically use the tax return data, the main part of which is the data about wealthy people. Since our data is based on survey, which represents the income distribution of population, we believe that we can estimate the giving price elasticity of population more precisely. Using the data with low-income households may be difficult to estimate the giving price elasticity in terms of intensive margin since they are expected to donate less than high-income households. To address this issue, we estimate not only the elasticity of intensive margin, as most of papers do, but also the elasticity of extensive margin following Almunia et al. (n.d.).
  Moreover, we use the data of Korea, a non-Western country, which the extant research did not examine\footnote{This point may be important since @Kim2021 reports that the giving behavior is strongly affected by the cultural matter such as the religious belief.}.
  
  In the second strand, there are some experimental studies and papers considering the tax evasion. Using an experiment, Li et al. (2011) compare people's willingness to give money for private charities and government agencies both of whose missions are the same. They show that people tend to donate for private charities more than government agency though they do not directly investigate the relationship between people's perception toward the government and giving behavior. Sheremeta and Uler (2020) show that people increase the voluntary public good provision when they face the wasteful government spending in the experimental setting. Although the government in their setting does not provide public good, they suggest that the willingness for donation may increase if people perceive the inefficiency of government. In the tax evasion literature, several paper suggests the perceived inefficiency of government reduce tax morale (Anderson, n.d.; Frey and Torgler, 2007; Hammar et al., 2009). We contribute on this literature by showing the relation between the perception of government efficiency and the giving behavior.
  
  This paper consist of XXX sections. Section 2 and 3 respectively explain the institutional background and data. Section 4 deals with the analysis of giving price elasticity and section 5 shows the analysis of perceptions toward the government. We discuss the result in section 6 and section 7 concludes.
  
  \hypertarget{institutional-background}{%
  \section{Institutional background}\label{institutional-background}}
  
  In this section, we describe the income tax relief for charitable giving in Korea and used dataset.
  
  \hypertarget{tax-relief-for-charitable-giving-and-tax-reform-in-2014-need-modification-by-kim-san}{%
  \subsection{Tax relief for charitable giving and tax reform in 2014 (Need modification by Kim san)}\label{tax-relief-for-charitable-giving-and-tax-reform-in-2014-need-modification-by-kim-san}}
  
  The tax incentives for charitable giving in Korea stared in 2000. The income tax deduction was initially used as a tax incentive and the limit of charitable giving deduction was set as the 5\% of income. The government gradually expanded the limit and it finally became 30\% in 2012. However, the government reformed tax system again in 2014, where the tax credit was introduced instead of tax deduction. Since then, 15\% of the total amount of charitable giving has been allowed as a tax credit.
  
  The tax deduction and tax credit may have different effects on giving behavior.
  Consider that a household has a choice between private consumptions (\(x_i\)) and charitable giving (\(g_i\)). Let \(y_i\) be pre-tax total income.
  Then, the budget constraint is
  
  \[
      x_i + g_i = y_i - T_i(y_i, g_i).
  \]
  \(T_i\) is tax amount which depends on the pre-tax income and charitable giving.
  On one hand, tax deduction reduces taxable income by giving, that is,
  
  \[
      T_i = \tau(y_i - g_i) \cdot (y_i - g_i),
  \]
  
  where \(\tau(\cdot)\) is the marginal income tax rate which is determined by \(y_i - g_i\). The budget constraint will be
  
  \[
      x_i + [1 - \tau(y_i - g_i)]g_i = [1 - \tau(y_i - g_i)] y_i.
  \]
  
  Thus, the giving price compared to the price of private consumption is \(p_i^{d} \equiv 1 - \tau(y_i - g_i)\) in tax deduction system.
  On the other hand, tax credit reduces tax amount directly, that is,
  
  \[
      T_i = \tau(y_i)\cdot y_i - m g_i,
  \]
  
  where \(m \in [0, 1]\) is the tax credit rate. Under the tax credit system, the budget constraint is
  
  \[
      x_i + (1 - m) g_i = [1 - \tau(y_i)] y_i.
  \]
  
  Thus, the giving price of tax credit system will be \(p_i^c = 1 - m\).
  
  In the South Korea, the tax policy about charitable giving drastically changed in 2014.
  
  \begin{itemize}
  \tightlist
  \item
    tax deduction (before 2014): \(\text{Price}_i = 1 - \tau(y_i - g_i)\)
  
    \begin{itemize}
    \tightlist
    \item
      the giving price is endogenous because people can manipulate \(\tau(y_i - g_i)\) using the charitable giving \(g_i\). Since this problem is caused by \emph{last} donations, we use the giving price applying to the \emph{first} donations (\textbf{first price}). The first price is calculate by \(\tau(y_i)\) where \(y_i\) is the annual taxable income reported in the NaSTaB.
    \end{itemize}
  \item
    tax credit (after 2014): \(\text{Price}_i = 1 - m\)
  
    \begin{itemize}
    \tightlist
    \item
      In the South Korea, the tax credit rate determines exogeneity, \(m = 0.15\).
    \end{itemize}
  \end{itemize}
  
  As well as other countries, the income tax in Korea can be calculated as following two steps:
  \begin{align}
  &\text{[Taxable income] = [Total income] – [Amount of tax deduction]}\\
  &\text{[Finalized tax amount] = ([Taxable income] × [Tax rate]) – [Amount of tax credit]}
  \end{align}
  If donors gives one unit of donation additionally, the marginal income tax relief for the donation (giving price hereafter) will equivalent to one minus income tax rate when the tax deduction is applied while the giving price under the tax credit will be one minus the allowance rate of tax credit, which is 30\% in Korea. Since the progressive tax rate is applied and the marginal tax rate was set as Table \ref{tax_rate}, the high income household .
  
  We can utilize the effect of the 2014 tax reform in the South Korea.
  • Before 2014, tax deduction was adopted to subsidize charitable donation behavior.
  • After 2014, tax credit have been adopted.
  The main difference is that tax credits reduce taxes directly, while tax deductions indirectly lower the
  tax burden by decreasing the marginal tax rate, which increases with gross income.
  In addition, the dataset contains the information about perception towards the government
  
  \hypertarget{data}{%
  \section{Data}\label{data}}
  
  \hypertarget{national-survey-of-tax-and-benefit-nastab}{%
  \subsection{National Survey of Tax and Benefit (NaSTaB)}\label{national-survey-of-tax-and-benefit-nastab}}
  
  \begin{itemize}
  \tightlist
  \item
    The Korea Institute of Taxation and Finance implements the financial panel survey to study the tax burden of households and the benefits that households receive from goverment.
  \item
    The subjects of this survey are general household and household members living in 15 cities and provinces nationwide.
  \item
    This survey is based on a face-to-face interview. If it is difficult for investigators to meet subjects, another family member answers on behalf of him.
  \item
    Survey items: Annual taxable income (last year), charitable donations (last year), trust for politicians (5-Likert scale), and other covariates (age, education, gender etc.).
  \item
    Survey period: 2008 \textasciitilde{} 2019
  
    \begin{itemize}
    \tightlist
    \item
      We use survey data after 2013 to focus on tax policy change in 2014.
    \end{itemize}
  \end{itemize}
  
  \hypertarget{time-series-of-chariable-giving}{%
  \subsection{Time Series of Chariable Giving}\label{time-series-of-chariable-giving}}
  
  \begin{figure}
  
  {\centering \includegraphics[width=0.9\linewidth]{C:/Users/gt510/Documents/Korea/_assets/SummaryOutcome} 
  
  }
  
  \caption{Proportion of Donors and Average Donations among Donors}\label{fig:unnamed-chunk-1}
  \end{figure}
  
  \hypertarget{summary-statistics-of-covariates}{%
  \subsection{Summary Statistics of Covariates}\label{summary-statistics-of-covariates}}
  
  \begin{table}
  
  \caption{\label{tab:kableSummaryCovariate}Summary Statistics of Covariates}
  \centering
  \begin{tabular}[t]{lcccc}
  \toprule
   & 2012 & 2013 & 2014 & 2015\\
  \midrule
  Female & 0.51 & 0.51 & 0.52 & 0.52\\
  Age & 38.39 & 39.10 & 39.67 & 40.51\\
  Annual taxable income & 1699.86 & 1764.04 & 1838.76 & 1872.54\\
  University graduate & 0.28 & 0.28 & 0.29 & 0.30\\
  High school graduate & 0.30 & 0.30 & 0.31 & 0.31\\
  \#.Respondents & 14138 & 13984 & 13787 & 13524\\
  \#.Households & 4756 & 4807 & 4819 & 4832\\
  \bottomrule
  \end{tabular}
  \end{table}
  
  \hypertarget{summary-statistics-of-covariates-contd}{%
  \subsection{Summary Statistics of Covariates (Cont'd)}\label{summary-statistics-of-covariates-contd}}
  
  \begin{table}
  
  \caption{\label{tab:kableSummaryCovariate2}Summary Statistics of Covariates (Continued)}
  \centering
  \begin{tabular}[t]{lccc}
  \toprule
   & 2016 & 2017 & 2018\\
  \midrule
  Female & 0.52 & 0.52 & 0.52\\
  Age & 41.07 & 41.89 & 42.55\\
  Annual taxable income & 1906.91 & 1951.55 & 2039.47\\
  University graduate & 0.31 & 0.33 & 0.34\\
  High school graduate & 0.31 & 0.31 & 0.31\\
  \#.Respondents & 13238 & 12963 & 12795\\
  \#.Households & 4790 & 4770 & 4765\\
  \bottomrule
  \end{tabular}
  \end{table}
  
  \hypertarget{what-is-giving-price}{%
  \subsection{What is Giving Price?}\label{what-is-giving-price}}
  
  Consider allocation between private consumptions (\(x_i\)) and charitable giving (\(g_i\)).
  Let \(y_i\) be pre-tax total income.
  Then, the budget constraint is
  
  \[
      x_i + g_i = y_i - T_i(y_i, g_i),
  \]
  
  where \(T_i\) is tax amount depending on the pre-tax income and charitable giving.
  
  \hypertarget{determination-of-tax-amount}{%
  \subsection{Determination of Tax Amount}\label{determination-of-tax-amount}}
  
  Tax deduction reduces taxable income by giving, that is,
  
  \[
      T_i = \tau(y_i - g_i) \cdot (y_i - g_i),
  \]
  
  where \(\tau(\cdot)\) is the marginal income tax rate which is determined by \(y_i - g_i\).
  
  Tax credit reduces tax amount directly, that is,
  
  \[
      T_i = \tau(y_i)\cdot y_i - m g_i,
  \]
  
  where \(m \in [0, 1]\) is the tax credit rate.
  
  \hypertarget{derive-giving-price}{%
  \subsection{Derive Giving Price}\label{derive-giving-price}}
  
  Under the tax deduction system, the budget constraint is
  
  \[
      x_i + [1 - \tau(y_i - g_i)]g_i = [1 - \tau(y_i - g_i)] y_i.
  \]
  
  Thus, the giving price of tax deduction system is \(p_i^{d} = 1 - \tau(y_i - g_i)\).
  
  Under the tax credit system, the budget constraint is
  
  \[
      x_i + (1 - m) g_i = [1 - \tau(y_i)] y_i.
  \]
  
  Thus, the giving price of tax credit system is \(p_i^c = 1 - m\).
  
  \hypertarget{construct-giving-price}{%
  \subsection{Construct Giving Price}\label{construct-giving-price}}
  
  In the South Korea, the tax policy about charitable giving drastically changed in 2014.
  
  \begin{itemize}
  \tightlist
  \item
    tax deduction (before 2014): \(\text{Price}_i = 1 - \tau(y_i - g_i)\)
  
    \begin{itemize}
    \tightlist
    \item
      the giving price is endogenous because people can manipulate \(\tau(y_i - g_i)\) using the charitable giving \(g_i\). Since this problem is caused by \emph{last} donations, we use the giving price applying to the \emph{first} donations (\textbf{first price}). The first price is calculate by \(\tau(y_i)\) where \(y_i\) is the annual taxable income reported in the NaSTaB.
    \end{itemize}
  \item
    tax credit (after 2014): \(\text{Price}_i = 1 - m\)
  
    \begin{itemize}
    \tightlist
    \item
      In the South Korea, the tax credit rate determines exogeneity, \(m = 0.15\).
    \end{itemize}
  \end{itemize}
  
  \hypertarget{income-distribution-and-giving-price}{%
  \subsection{Income Distribution and Giving Price}\label{income-distribution-and-giving-price}}
  
  \begin{figure}
  
  {\centering \includegraphics[width=0.9\linewidth]{C:/Users/gt510/Documents/Korea/_assets/SummaryPriceChange} 
  
  }
  
  \caption{Income Distribution and Giving Price in 2013}\label{fig:unnamed-chunk-2}
  \end{figure}
  
  \hypertarget{time-series-of-average-donations-by-benefit-group}{%
  \subsection{Time Series of Average Donations By Benefit Group}\label{time-series-of-average-donations-by-benefit-group}}
  
  \begin{figure}
  
  {\centering \includegraphics[width=0.9\linewidth]{C:/Users/gt510/Documents/Korea/_assets/AverageDonationsByBenefitGroup} 
  
  }
  
  \caption{Time Series of Average Donations by Benefit Group}\label{fig:unnamed-chunk-3}
  \end{figure}
  
  \hypertarget{time-series-of-extensive-margin-by-benfit-group}{%
  \subsection{Time Series of Extensive Margin by Benfit Group}\label{time-series-of-extensive-margin-by-benfit-group}}
  
  \begin{figure}
  
  {\centering \includegraphics[width=0.9\linewidth]{C:/Users/gt510/Documents/Korea/_assets/ExtensiveByBenefitGroup} 
  
  }
  
  \caption{Time Series of Proportion of Donors by Benefit Group}\label{fig:unnamed-chunk-4}
  \end{figure}
  
  \hypertarget{time-series-of-intensive-margin-by-benfit-group}{%
  \subsection{Time Series of Intensive Margin by Benfit Group}\label{time-series-of-intensive-margin-by-benfit-group}}
  
  \begin{figure}
  
  {\centering \includegraphics[width=0.9\linewidth]{C:/Users/gt510/Documents/Korea/_assets/IntensiveByBenefitGroup} 
  
  }
  
  \caption{Time Series of Average Donations among Donors by Benefit Group}\label{fig:unnamed-chunk-5}
  \end{figure}
  
  \hypertarget{empirical-strategy}{%
  \subsection{Empirical Strategy}\label{empirical-strategy}}
  
  Our baseline regression equation is
  
  \[
      \log(\text{Giving}_{ijt}) = 
      \alpha_i + \beta_1 \log(\text{Price}_{ijt}) + \delta X_{ijt} + \lambda_t + \epsilon_{ijt}.
  \]
  
  \begin{itemize}
  \tightlist
  \item
    \(\log(\text{Giving}_{ijt})\) is logarithm of individual \(i\)'s charitable giving in year \(t\).
  \item
    \(\log(\text{Price}_{ijt})\) is logarithm of individual \(i\)'s giving price in year \(t\).
  \item
    \(\beta_1\) represents the price elasticity of giving.
  \item
    \(\alpha_i\) and \(\lambda_t\) are individual and time fixed effect, respectively.
  \end{itemize}
  
  \hypertarget{intensive-margin-and-extensive-margin}{%
  \subsection{Intensive Margin and Extensive Margin}\label{intensive-margin-and-extensive-margin}}
  
  Let \(D_{ijt}\) be a dummy variable taking 1 if individual \(i\) whose resident area \(j\) in year \(t\) donate in year \(t\)
  
  \begin{itemize}
  \tightlist
  \item
    Intensive margin: Estiamte \(\beta_1\) where outcome variable is \(\log(\text{Giving}_{ijt})\), using units with \(D_{ijt} = 1\).
  \item
    Extensive margin: Estimate \(\beta_1\) where outcome variable is \(D_{ijt}\).
  
    \begin{itemize}
    \tightlist
    \item
      Extensive-margin price elasticity can be calculated by \(\beta_1/\bar{D}\) where \(\bar{D}\) is the sample mean of \(D_{ijt}\).
    \end{itemize}
  \end{itemize}
  
  Covariates in each column corresponds to a column in a previous slide.
  
  \hypertarget{main-results}{%
  \section{Main Results}\label{main-results}}
  
  \hypertarget{price-and-income-elasticity}{%
  \subsection{Price and Income Elasticity}\label{price-and-income-elasticity}}
  
  \begin{table}
  
  \caption{\label{tab:kableEstimateElasticityPart1}Main Results}
  \centering
  \begin{threeparttable}
  \begin{tabular}[t]{lccccc}
  \toprule
   & (1) & (2) & (3) & (4) & (5)\\
  \midrule
  ln(giving price) & -1.071*** & -1.264*** & -1.298*** & -1.117*** & -1.121***\\
   & (0.201) & (0.212) & (0.229) & (0.228) & (0.228)\\
  Time FE & Y & Y & Y & Y & Y\\
  Age & N & Y & Y & Y & Y\\
  Year X Education & N & N & Y & Y & Y\\
  Year X Gender & N & N & N & Y & Y\\
  Resident Area & N & N & N & N & Y\\
  N & 54213 & 54213 & 54211 & 54211 & 54211\\
  \bottomrule
  \end{tabular}
  \begin{tablenotes}
  \item Notes: $^{*}$ $p < 0.1$, $^{**}$ $p < 0.05$, $^{***}$ $p < 0.01$. Standard errors are clustered at individual level. When controlling age, we alson include its squared term.
  \end{tablenotes}
  \end{threeparttable}
  \end{table}
  
  \begin{table}
  
  \caption{\label{tab:kableEstimateElasticityPart2}Main Results: Intensive- and Extensive-Margin Elasticity}
  \centering
  \begin{threeparttable}
  \begin{tabular}[t]{lccccc}
  \toprule
   & (1) & (2) & (3) & (4) & (5)\\
  \midrule
  \addlinespace[0.3em]
  \multicolumn{6}{l}{\textbf{Intensive-Margin Elasticity}}\\
  \hspace{1em}ln(giving price) & -0.593*** & -0.843*** & -1.022*** & -0.887*** & -0.891***\\
  \hspace{1em} & (0.202) & (0.212) & (0.231) & (0.242) & (0.243)\\
  \hspace{1em}N & 11704 & 11704 & 11704 & 11704 & 11704\\
  \addlinespace[0.3em]
  \multicolumn{6}{l}{\textbf{Extensive-Margin Elasticity}}\\
  \hspace{1em}ln(giving price) & -0.258*** & -0.290*** & -0.274*** & -0.238*** & -0.239***\\
  \hspace{1em} & (0.046) & (0.048) & (0.052) & (0.052) & (0.052)\\
  \hspace{1em}Price elasticity at mean & -1.699*** & -1.907*** & -1.807*** & -1.569*** & -1.573***\\
  \hspace{1em} & (0.301) & (0.316) & (0.341) & (0.341) & (0.341)\\
  \hspace{1em}Time FE & Y & Y & Y & Y & Y\\
  \hspace{1em}Age & N & Y & Y & Y & Y\\
  \hspace{1em}Year X Education & N & N & Y & Y & Y\\
  \hspace{1em}Year X Gender & N & N & N & Y & Y\\
  \hspace{1em}Resident Area & N & N & N & N & Y\\
  \hspace{1em}N & 54213 & 54213 & 54211 & 54211 & 54211\\
  \bottomrule
  \end{tabular}
  \begin{tablenotes}
  \item Notes: $^{*}$ $p < 0.1$, $^{**}$ $p < 0.05$, $^{***}$ $p < 0.01$. Standard errors are clustered at individual level. When controlling age, we alson include its squared term. The implied extensive-marign price elasticity is evaluated at the sample mean of $D_{ijt}$.
  \end{tablenotes}
  \end{threeparttable}
  \end{table}
  
  \hypertarget{robustness-check}{%
  \subsection{Robustness Check}\label{robustness-check}}
  
  \begin{table}
  
  \caption{\label{tab:kablePanelIVEstimateElasticity}Panel IV Results}
  \centering
  \begin{threeparttable}
  \begin{tabular}[t]{lccc}
  \toprule
  \multicolumn{1}{c}{Lag $k$} & \multicolumn{1}{c}{$k = 1$} & \multicolumn{1}{c}{$k = 2$} & \multicolumn{1}{c}{$k = 3$} \\
  \cmidrule(l{3pt}r{3pt}){1-1} \cmidrule(l{3pt}r{3pt}){2-2} \cmidrule(l{3pt}r{3pt}){3-3} \cmidrule(l{3pt}r{3pt}){4-4}
   & (1) & (2) & (3)\\
  \midrule
  ln(giving price) & -1.279*** & -1.155*** & -1.150***\\
   & (0.478) & (0.414) & (0.369)\\
  Individual FE & Y & Y & Y\\
  Time FE & Y & Y & Y\\
  Other Controls & Y & Y & Y\\
  F-stat of IV & 10315.94 & 11506.64 & 11569.61\\
  N & 51548 & 49217 & 46399\\
  \bottomrule
  \end{tabular}
  \begin{tablenotes}
  \item Notes: $^{*}$ $p < 0.1$, $^{**}$ $p < 0.05$, $^{***}$ $p < 0.01$. Standard errors are clustered at individual level. Other controls are age (its squared value), the interaction between year dummies and education dummies, the interaction between year dummies and gender dummies, and resident area. The instumental variable is $\log(\text{Price}_{ijt}/\text{Price}_{ij(t-k)})$.
  \end{tablenotes}
  \end{threeparttable}
  \end{table}
  
  \begin{table}
  
  \caption{\label{tab:kablePanelIVEstimateElasticityIntExt}Panel IV Results: Intensive- and Extensive-Margin Elasticity}
  \centering
  \begin{threeparttable}
  \begin{tabular}[t]{lccc}
  \toprule
  \multicolumn{1}{c}{Lag $k$} & \multicolumn{1}{c}{$k = 1$} & \multicolumn{1}{c}{$k = 2$} & \multicolumn{1}{c}{$k = 3$} \\
  \cmidrule(l{3pt}r{3pt}){1-1} \cmidrule(l{3pt}r{3pt}){2-2} \cmidrule(l{3pt}r{3pt}){3-3} \cmidrule(l{3pt}r{3pt}){4-4}
   & (1) & (2) & (3)\\
  \midrule
  \addlinespace[0.3em]
  \multicolumn{4}{l}{\textbf{Intensive Margin}}\\
  \hspace{1em}ln(giving price) & -0.0004 & 0.0261 & -0.4378\\
  \hspace{1em} & (0.5687) & (0.4410) & (0.3763)\\
  \hspace{1em}F-stat of IV & 1679.78 & 2040.66 & 2419.05\\
  \hspace{1em}N & 11332 & 10954 & 10451\\
  \addlinespace[0.3em]
  \multicolumn{4}{l}{\textbf{Extensive Margin}}\\
  \hspace{1em}ln(giving price) & -0.3036*** & -0.2944*** & -0.2472***\\
  \hspace{1em} & (0.1101) & (0.0934) & (0.0847)\\
  \hspace{1em}Price elasticity at mean & -2.000*** & -1.939*** & -1.628***\\
  \hspace{1em} & (0.725) & (0.615) & (0.558)\\
  \hspace{1em}Individual FE & Y & Y & Y\\
  \hspace{1em}Time FE & Y & Y & Y\\
  \hspace{1em}Other Controls & Y & Y & Y\\
  \hspace{1em}F-stat of IV & 10315.94 & 11506.64 & 11569.61\\
  \hspace{1em}N & 51548 & 49217 & 46399\\
  \bottomrule
  \end{tabular}
  \begin{tablenotes}
  \item Notes: $^{*}$ $p < 0.1$, $^{**}$ $p < 0.05$, $^{***}$ $p < 0.01$. Standard errors are clustered at individual level. Other controls are age (its squared value), the interaction between year dummies and education dummies, the interaction between year dummies and gender dummies, and resident area. The instumental variable is $\log(\text{Price}_{ijt}/\text{Price}_{ij(t-k)})$. The implied extensive-marign price elasticity is evaluated at the sample mean of $D_{ijt}$.
  \end{tablenotes}
  \end{threeparttable}
  \end{table}
  
  \begin{table}
  
  \caption{\label{tab:kableShortEstimateElasticity}Results with 2013 and 2014 Data}
  \centering
  \begin{threeparttable}
  \begin{tabular}[t]{lcccc}
  \toprule
  \multicolumn{1}{c}{Model} & \multicolumn{1}{c}{FE} & \multicolumn{3}{c}{Panel IV} \\
  \cmidrule(l{3pt}r{3pt}){1-1} \cmidrule(l{3pt}r{3pt}){2-2} \cmidrule(l{3pt}r{3pt}){3-5}
  \multicolumn{1}{c}{Lag $k$} & \multicolumn{1}{c}{ } & \multicolumn{1}{c}{$k = 1$} & \multicolumn{1}{c}{$k = 2$} & \multicolumn{1}{c}{$k = 3$} \\
  \cmidrule(l{3pt}r{3pt}){1-1} \cmidrule(l{3pt}r{3pt}){3-3} \cmidrule(l{3pt}r{3pt}){4-4} \cmidrule(l{3pt}r{3pt}){5-5}
   & (1) & (2) & (3) & (4)\\
  \midrule
  ln(giving price) & -1.466*** & -1.535*** & -1.683*** & -1.151***\\
   & (0.327) & (0.360) & (0.378) & (0.385)\\
  Individual FE & Y & Y & Y & Y\\
  Time FE & Y & Y & Y & Y\\
  Other Controls & Y & Y & Y & Y\\
  F-stat of IV &  & 7420.10 & 4490.74 & 5034.58\\
  N & 15134 & 13727 & 12902 & 12420\\
  \bottomrule
  \end{tabular}
  \begin{tablenotes}
  \item Notes: $^{*}$ $p < 0.1$, $^{**}$ $p < 0.05$, $^{***}$ $p < 0.01$. Standard errors are clustered at individual level. Other controls are age (its squared value), the interaction between year dummies and education dummies, the interaction between year dummies and gender dummies, and resident area. The instumental variable is $\log(\text{Price}_{ijt}/\text{Price}_{ij(t-k)})$.
  \end{tablenotes}
  \end{threeparttable}
  \end{table}
  
  \begin{table}
  
  \caption{\label{tab:kableShortEstimateElasticityIntExt}Intensive- and Extensive-Margin Elasticity with 2013 and 2014 Data}
  \centering
  \begin{threeparttable}
  \begin{tabular}[t]{lcccc}
  \toprule
  \multicolumn{1}{c}{Model} & \multicolumn{1}{c}{FE} & \multicolumn{3}{c}{Panel IV with FE} \\
  \cmidrule(l{3pt}r{3pt}){1-1} \cmidrule(l{3pt}r{3pt}){2-2} \cmidrule(l{3pt}r{3pt}){3-5}
  \multicolumn{1}{c}{Lag $k$} & \multicolumn{1}{c}{ } & \multicolumn{1}{c}{$k = 1$} & \multicolumn{1}{c}{$k = 2$} & \multicolumn{1}{c}{$k = 3$} \\
  \cmidrule(l{3pt}r{3pt}){1-1} \cmidrule(l{3pt}r{3pt}){3-3} \cmidrule(l{3pt}r{3pt}){4-4} \cmidrule(l{3pt}r{3pt}){5-5}
   & (1) & (2) & (3) & (4)\\
  \midrule
  \addlinespace[0.3em]
  \multicolumn{5}{l}{\textbf{Intensive Margin}}\\
  \hspace{1em}ln(giving price) & -0.759** & -0.736* & -0.819** & -0.543\\
  \hspace{1em} & (0.344) & (0.418) & (0.404) & (0.371)\\
  \hspace{1em}F-stat of IV &  & 1920.08 & 1762.03 & 1706.53\\
  \hspace{1em}N & 2938 & 2746 & 2615 & 2512\\
  \addlinespace[0.3em]
  \multicolumn{5}{l}{\textbf{Extensive Margin}}\\
  \hspace{1em}ln(giving price) & -0.332*** & -0.341*** & -0.380*** & -0.291***\\
  \hspace{1em} & (0.074) & (0.083) & (0.085) & (0.089)\\
  \hspace{1em}Elasticity & -2.186*** & -2.249*** & -2.504*** & -1.920***\\
  \hspace{1em} & (0.488) & (0.547) & (0.559) & (0.583)\\
  \hspace{1em}Individual FE & Y & Y & Y & Y\\
  \hspace{1em}Time FE & Y & Y & Y & Y\\
  \hspace{1em}Other Controls & Y & Y & Y & Y\\
  \hspace{1em}F-stat of IV &  & 7420.10 & 4490.74 & 5034.58\\
  \hspace{1em}N & 15134 & 13727 & 12902 & 12420\\
  \bottomrule
  \end{tabular}
  \begin{tablenotes}
  \item Notes: $^{*}$ $p < 0.1$, $^{**}$ $p < 0.05$, $^{***}$ $p < 0.01$. Standard errors are clustered at individual level. Other controls are age (its squared value), the interaction between year dummies and education dummies, the interaction between year dummies and gender dummies, and resident area. The instumental variable is $\log(\text{Price}_{ijt}/\text{Price}_{ij(t-k)})$. The implied extensive-marign price elasticity is evaluated at the sample mean of $D_{ijt}$.
  \end{tablenotes}
  \end{threeparttable}
  \end{table}
  
  \hypertarget{governement-efficient-and-price-elasticity}{%
  \section{Governement Efficient and Price Elasticity}\label{governement-efficient-and-price-elasticity}}
  
  \hypertarget{government-efficiency}{%
  \subsection{Government Efficiency}\label{government-efficiency}}
  
  From the 2015 survey,
  NaSTaB asks the current and ideal balance between tax burden and welfare size.
  
  These variables provide us to investigate the relationship between price elasticity and govenrment's efficiency
  more directly.
  
  Thus, we did same excercise, using the current balance bewteen tax burden and welfare size.
  
  \hypertarget{construct-efficient-index}{%
  \subsection{Construct Efficient Index}\label{construct-efficient-index}}
  
  Questionnaire of tax-welfare balance index is
  
  \includegraphics[width=0.5\textwidth,height=\textheight]{_assets/BalanceQuestion.jpg}
  
  To rule out government's policies, we use individual fixed effect as the \textbf{efficient index}
  
  \hypertarget{histrogram-of-efficient-index}{%
  \subsection{Histrogram of Efficient Index}\label{histrogram-of-efficient-index}}
  
  \begin{figure}
  
  {\centering \includegraphics[width=0.9\linewidth]{C:/Users/gt510/Documents/Korea/_assets/HistogramBalanceid} 
  
  }
  
  \caption{Histogram of Efficient Index}\label{fig:unnamed-chunk-6}
  \end{figure}
  
  \hypertarget{heterogenous-price-elasticity-by-governement-efficiency}{%
  \subsection{Heterogenous Price Elasticity by Governement Efficiency}\label{heterogenous-price-elasticity-by-governement-efficiency}}
  
  To see the heterogenous price elasticity by efficient index,
  We estimated the baseline regression model (5) (see Table \ref{tab:kableEstimateElasticity}),
  using sample grouped by the efficient index.
  
  \begin{itemize}
  \tightlist
  \item
    Three quantile groups: we divide units \(i\) into the first, second, and third quantile of efficient index (1Q, 2Q, and 3Q, respectively).
  \end{itemize}
  
  \hypertarget{efficient-groups-descriptive-stats}{%
  \subsection{Efficient Groups: Descriptive Stats}\label{efficient-groups-descriptive-stats}}
  
  \begin{figure}
  
  {\centering \includegraphics[width=0.9\linewidth]{C:/Users/gt510/Documents/Korea/_assets/SummaryOutcomeByEfficientGroup3} 
  
  }
  
  \caption{Time Series of Average Donations by Subgroup}\label{fig:unnamed-chunk-7}
  \end{figure}
  
  \hypertarget{efficient-groups-descriptive-statis-extensive-margin}{%
  \subsection{Efficient Groups: Descriptive Statis (Extensive Margin)}\label{efficient-groups-descriptive-statis-extensive-margin}}
  
  \begin{figure}
  
  {\centering \includegraphics[width=0.9\linewidth]{C:/Users/gt510/Documents/Korea/_assets/SummaryOutcomeExtensiveByEfficientGroup3} 
  
  }
  
  \caption{Time Series of Proportion of Donors by Subgroup}\label{fig:unnamed-chunk-8}
  \end{figure}
  
  \hypertarget{efficient-groups-descriptive-stats-intensive-margin}{%
  \subsection{Efficient Groups: Descriptive Stats (Intensive Margin)}\label{efficient-groups-descriptive-stats-intensive-margin}}
  
  \begin{figure}
  
  {\centering \includegraphics[width=0.9\linewidth]{C:/Users/gt510/Documents/Korea/_assets/SummaryOutcomeIntensiveByEfficientGroup3} 
  
  }
  
  \caption{Time Series of Average Donations among Donors by Subgroup}\label{fig:unnamed-chunk-9}
  \end{figure}
  
  \hypertarget{efficient-groups-estimation-results}{%
  \subsection{Efficient Groups: Estimation Results}\label{efficient-groups-estimation-results}}
  
  \begin{table}
  
  \caption{\label{tab:kableEstimateElasticityByEfficientGroup3}Price Elasticity by Three Quantile Efficient Groups}
  \centering
  \fontsize{8}{10}\selectfont
  \begin{tabular}[t]{lccc}
  \toprule
   & 1Q & 2Q & 3Q\\
  \midrule
  \addlinespace[0.3em]
  \multicolumn{4}{l}{\textbf{Overall}}\\
  \hspace{1em}ln(giving price) & -1.321*** & -0.844** & -0.929**\\
  \hspace{1em} & (0.388) & (0.404) & (0.404)\\
  \hspace{1em}N & 17119 & 16662 & \vphantom{1} 17525\\
  \addlinespace[0.3em]
  \multicolumn{4}{l}{\textbf{Intensive Margin}}\\
  \hspace{1em}ln(giving price) & -0.792** & -0.360 & -1.111**\\
  \hspace{1em} & (0.383) & (0.423) & (0.497)\\
  \hspace{1em}N & 3696 & 3591 & 4105\\
  \addlinespace[0.3em]
  \multicolumn{4}{l}{\textbf{Extensive Margin}}\\
  \hspace{1em}ln(giving price) & -0.276*** & -0.225** & -0.174*\\
  \hspace{1em} & (0.087) & (0.094) & (0.091)\\
  \hspace{1em}Elasticity & -1.380*** & -1.115** & -0.787*\\
  \hspace{1em} & (0.435) & (0.466) & (0.412)\\
  \hspace{1em}N & 17119 & 16662 & 17525\\
  \bottomrule
  \end{tabular}
  \end{table}
  
  \hypertarget{robustness-check-1}{%
  \subsection{Robustness Check}\label{robustness-check-1}}
  
  \begin{enumerate}
  \def\labelenumi{\arabic{enumi}.}
  \tightlist
  \item
    Efficient index captures both government efficiency on concerns about budget deficits
  \item
    Effect of presidential transition on efficient index
  \item
    Effect of presidential transition on donation behavior
  \item
    Income and donations are determined simultaneously
  \item
    Last price elasticity
  \item
    Self-selection of receiving tax benefit
  \item
    Transitory and permanent elasticity
  \end{enumerate}
  
  \hypertarget{robustness-check-1-1}{%
  \subsection{Robustness Check 1}\label{robustness-check-1-1}}
  
  \begin{itemize}
  \tightlist
  \item
    Efficient index may capture both government efficiency on concerns about budget deficits
  
    \begin{itemize}
    \tightlist
    \item
      NASTAB asks respondents to answer the ideal balance b/w tax burdern and welfare size.
    \item
      We constructed the \textbf{ideal} efficient index, using the FE model to estimate the efficient index.
    \item
      We droped units with the ideal efficient index is less than 0 from each quantile group and repeated the same excercise.
    \item
      This is because respondents whose the ideal efficient index is less than 0 think governments should try to avoid budget deficits (high tax, low welfare).
    \end{itemize}
  \item
    Presidential transition effect on perceived efficiency
  
    \begin{itemize}
    \tightlist
    \item
      We constructed president-specific (ideal) efficient index and implemented the pair-wise t-test.
    \item
      As a result, average difference of these two indexs are not statistically siginificant zero.
    \end{itemize}
  \end{itemize}
  
  \hypertarget{robustness-check-1-estimation-results}{%
  \subsection{Robustness Check 1: Estimation Results}\label{robustness-check-1-estimation-results}}
  
  \begin{table}
  
  \caption{\label{tab:kableEstimateElasticityByPositiveEfficientGroup3}Heterogenous Price Elasticity by Efficiency Using Units with Ideal Efficient Index > 0}
  \centering
  \fontsize{8}{10}\selectfont
  \begin{tabular}[t]{lccc}
  \toprule
   & 1Q & 2Q & 3Q\\
  \midrule
  \addlinespace[0.3em]
  \multicolumn{4}{l}{\textbf{Overall}}\\
  \hspace{1em}ln(giving price) & -1.996*** & -1.122* & -0.063\\
  \hspace{1em} & (0.648) & (0.597) & (0.488)\\
  \hspace{1em}N & 7527 & 6900 & \vphantom{1} 9339\\
  \addlinespace[0.3em]
  \multicolumn{4}{l}{\textbf{Intensive Margin}}\\
  \hspace{1em}ln(giving price) & -1.138* & -0.900 & -0.952\\
  \hspace{1em} & (0.640) & (0.652) & (0.741)\\
  \hspace{1em}N & 1541 & 1474 & 2023\\
  \addlinespace[0.3em]
  \multicolumn{4}{l}{\textbf{Extensive Margin}}\\
  \hspace{1em}ln(giving price) & -0.317** & -0.220 & 0.014\\
  \hspace{1em} & (0.136) & (0.137) & (0.119)\\
  \hspace{1em}Elasticity & -1.582** & -1.091 & 0.062\\
  \hspace{1em} & (0.681) & (0.679) & (0.537)\\
  \hspace{1em}N & 7527 & 6900 & 9339\\
  \bottomrule
  \end{tabular}
  \end{table}
  
  \hypertarget{robustness-check-2}{%
  \subsection{Robustness Check 2}\label{robustness-check-2}}
  
  We check the following two potential concerns
  
  \begin{itemize}
  \tightlist
  \item
    Presidential transition effect on donation behavior
  \item
    Income and donations are determined simultaneously
  \end{itemize}
  
  To address these problems, we estimated the FE model and Panel IV model with FE where instrument is \(\log(\text{Price}_{ijt}/\text{Price}_{ij(t-k)})\) for \(k = 1, 2, 3\), using data in 2013 and 2014.
  Moreover, we droped units with the ideal efficient index \textless{} 0 from each quantile group.
  
  Note that f-statistics of IV is greater than 500 when we estimate overall elasticity and extensive-margin elasticity, and greater than 100 when we estimate the intensive-margin elasticity.
  
  \hypertarget{robustness-check-2-result}{%
  \subsection{Robustness Check 2: Result}\label{robustness-check-2-result}}
  
  \begin{table}
  
  \caption{\label{tab:tabShortEstimateElasticityByEfficientGroup3}Robustness Check of Heterogenous Price Elasiticity by Government Efficiency}
  \centering
  \fontsize{8}{10}\selectfont
  \begin{tabular}[t]{lccc}
  \toprule
   & 1Q & 2Q & 3Q\\
  \midrule
  \addlinespace[0.3em]
  \multicolumn{4}{l}{\textbf{FE Model}}\\
  \hspace{1em}ln(giving price) & -1.989** & -1.047 & -1.881**\\
  \hspace{1em} & (0.913) & (1.078) & (0.835)\\
  \hspace{1em}N & 2021 & 1841 & 2504\\
  \addlinespace[0.3em]
  \multicolumn{4}{l}{\textbf{Panel IV (k = 1)}}\\
  \hspace{1em}ln(giving price) & -1.881* & -1.093 & -2.189**\\
  \hspace{1em} & (0.992) & (1.272) & (0.851)\\
  \hspace{1em}N & 1842 & 1689 & 2292\\
  \addlinespace[0.3em]
  \multicolumn{4}{l}{\textbf{Panel IV (k = 2)}}\\
  \hspace{1em}ln(giving price) & -1.958* & -1.594 & -1.684\\
  \hspace{1em} & (1.101) & (1.212) & (1.024)\\
  \hspace{1em}N & 1723 & 1582 & 2174\\
  \addlinespace[0.3em]
  \multicolumn{4}{l}{\textbf{Panel IV (k = 3)}}\\
  \hspace{1em}ln(giving price) & -1.608 & -0.317 & -1.544\\
  \hspace{1em} & (1.079) & (1.219) & (0.999)\\
  \hspace{1em}N & 1645 & 1529 & 2096\\
  \bottomrule
  \end{tabular}
  \end{table}
  
  \hypertarget{robustness-check-2-result-extensive-margin}{%
  \subsection{Robustness Check 2: Result (Extensive Margin)}\label{robustness-check-2-result-extensive-margin}}
  
  \begin{table}
  
  \caption{\label{tab:tabShortEstimateElasticityExtensiveByEfficientGroup3}Robustness Check of Heterogenous Extensive-Margin Price Elasiticity by Government Efficiency}
  \centering
  \fontsize{8}{10}\selectfont
  \begin{tabular}[t]{lccc}
  \toprule
   & 1Q & 2Q & 3Q\\
  \midrule
  \addlinespace[0.3em]
  \multicolumn{4}{l}{\textbf{FE Model}}\\
  \hspace{1em}Implied Elasticity & -1.558 & -0.649 & -2.453**\\
  \hspace{1em} & (1.119) & (1.326) & (1.147)\\
  \hspace{1em}N & 2021 & 1841 & 2504\\
  \addlinespace[0.3em]
  \multicolumn{4}{l}{\textbf{Panel IV (k = 1)}}\\
  \hspace{1em}Implied Elasticity & -1.345 & -0.517 & -2.934**\\
  \hspace{1em} & (1.255) & (1.612) & (1.180)\\
  \hspace{1em}N & 1842 & 1689 & 2292\\
  \addlinespace[0.3em]
  \multicolumn{4}{l}{\textbf{Panel IV (k = 2)}}\\
  \hspace{1em}Implied Elasticity & -1.396 & -1.557 & -1.998\\
  \hspace{1em} & (1.264) & (1.486) & (1.439)\\
  \hspace{1em}N & 1723 & 1582 & 2174\\
  \addlinespace[0.3em]
  \multicolumn{4}{l}{\textbf{Panel IV (k = 3)}}\\
  \hspace{1em}Implied Elasticity & -1.056 & -0.460 & -1.795\\
  \hspace{1em} & (1.262) & (1.477) & (1.355)\\
  \hspace{1em}N & 1645 & 1529 & 2096\\
  \bottomrule
  \end{tabular}
  \end{table}
  
  \hypertarget{robustness-check-2-result-intensive-margin}{%
  \subsection{Robustness Check 2: Result (Intensive Margin)}\label{robustness-check-2-result-intensive-margin}}
  
  \begin{table}
  
  \caption{\label{tab:tabShortEstimateElasticityIntensiveByEfficientGroup3}Robustness Check of Heterogenous Intenstive-Margin Price Elasiticity by Government Efficiency}
  \centering
  \fontsize{8}{10}\selectfont
  \begin{tabular}[t]{lccc}
  \toprule
   & 1Q & 2Q & 3Q\\
  \midrule
  \addlinespace[0.3em]
  \multicolumn{4}{l}{\textbf{FE Model}}\\
  \hspace{1em}ln(giving price) & -0.753 & -2.301* & -1.362\\
  \hspace{1em} & (0.848) & (1.310) & (0.956)\\
  \hspace{1em}N & 404 & 361 & 479\\
  \addlinespace[0.3em]
  \multicolumn{4}{l}{\textbf{Panel IV (k = 1)}}\\
  \hspace{1em}ln(giving price) & -0.749 & -3.220* & -1.471\\
  \hspace{1em} & (1.088) & (1.812) & (1.013)\\
  \hspace{1em}N & 380 & 340 & 449\\
  \addlinespace[0.3em]
  \multicolumn{4}{l}{\textbf{Panel IV (k = 2)}}\\
  \hspace{1em}ln(giving price) & -0.770 & -1.216 & -0.771\\
  \hspace{1em} & (0.997) & (1.504) & (0.966)\\
  \hspace{1em}N & 357 & 322 & 433\\
  \addlinespace[0.3em]
  \multicolumn{4}{l}{\textbf{Panel IV (k = 3)}}\\
  \hspace{1em}ln(giving price) & 0.573 & -0.691 & -1.472\\
  \hspace{1em} & (1.117) & (1.088) & (0.991)\\
  \hspace{1em}N & 337 & 307 & 414\\
  \bottomrule
  \end{tabular}
  \end{table}
  
  \hypertarget{conclusions}{%
  \section{Conclusions}\label{conclusions}}
  
  \hypertarget{conclusions-1}{%
  \subsection{Conclusions}\label{conclusions-1}}
  
  \clearpage
  
  \hypertarget{references}{%
  \subsection*{References}\label{references}}
  \addcontentsline{toc}{subsection}{References}
  
  \hypertarget{refs}{}
  \leavevmode\hypertarget{ref-Almunia2020}{}%
  Almunia, M., Guceri, I., Lockwood, B., Scharf, K., n.d. More giving or more givers? The effects of tax incentives on charitable donations in the UK. Journal of Public Economics 183. doi:\href{https://doi.org/\%7B10.1016/j.jpubeco.2019.104114\%7D}{\{10.1016/j.jpubeco.2019.104114\}}
  
  \leavevmode\hypertarget{ref-Anderson2017}{}%
  Anderson, J.E., n.d. Trust in Government and Willingness to Pay Taxes in Transition Countries. Comparative Economic Studies 59, 1--22. doi:\href{https://doi.org/\%7B10.1057/s41294-016-0017-x\%7D}{\{10.1057/s41294-016-0017-x\}}
  
  \leavevmode\hypertarget{ref-Auten2002}{}%
  Auten, G.E., Sieg, H., Clotfelter, C.T., 2002. Charitable Giving, Income, and Taxes: An Analysis of Panel Data. American Economic Review 92, 371--382.
  
  \leavevmode\hypertarget{ref-Bakija2011}{}%
  Bakija, J., Heim, B.T., 2011. How does charitable giving respond to incentives and income? New estimates from panel data. National Tax Journal 64, 615--650. doi:\href{https://doi.org/\%7B10.17310/ntj.2011.2S.08\%7D}{\{10.17310/ntj.2011.2S.08\}}
  
  \leavevmode\hypertarget{ref-Fack2010}{}%
  Fack, G., Landais, C., 2010. Are tax incentives for charitable giving efficient? Evidence from france. American Economic Journal - Economic Policy 2, 117--141. doi:\href{https://doi.org/\%7B10.1257/pol.2.2.117\%7D}{\{10.1257/pol.2.2.117\}}
  
  \leavevmode\hypertarget{ref-Frey2007}{}%
  Frey, B.S., Torgler, B., 2007. Tax morale and conditional cooperation. Journal of Comparative Economics 35, 136--159. doi:\href{https://doi.org/10.1016/j.jce.2006.10.006}{10.1016/j.jce.2006.10.006}
  
  \leavevmode\hypertarget{ref-Hammar2009}{}%
  Hammar, H., Jagers, S.C., Nordblom, K., 2009. Perceived tax evasion and the importance of trust. The Journal of Socio-Economics 38, 238--245. doi:\href{https://doi.org/https://doi.org/10.1016/j.socec.2008.07.003}{https://doi.org/10.1016/j.socec.2008.07.003}
  
  \leavevmode\hypertarget{ref-Li2011}{}%
  Li, S.X., Eckel, C.C., Grossman, P.J., Brown, T.L., 2011. Giving to government: Voluntary taxation in the lab. Journal of Public Economics 95, 1190--1201. doi:\href{https://doi.org/\%7B10.1016/j.jpubeco.2011.03.005\%7D}{\{10.1016/j.jpubeco.2011.03.005\}}
  
  \leavevmode\hypertarget{ref-Randolph1995}{}%
  Randolph, W.C., n.d. Dynamic income, progressive taxes, and the timing of charitable contributions. Journal of Political Economy 103, 709--738. doi:\href{https://doi.org/\%7B10.1086/262000\%7D}{\{10.1086/262000\}}
  
  \leavevmode\hypertarget{ref-Saez2004}{}%
  Saez, E., 2004. The optimal treatment of tax expenditures. Journal of Public Economics 88, 2657--2684. doi:\href{https://doi.org/\%7B10.1016/j.jpubeco.2003.09.004\%7D}{\{10.1016/j.jpubeco.2003.09.004\}}
  
  \leavevmode\hypertarget{ref-Sheremeta2020}{}%
  Sheremeta, R.M., Uler, N., 2020. The impact of taxes and wasteful government spending on giving. Experimental Economics. doi:\href{https://doi.org/10.1007/s10683-020-09673-9}{10.1007/s10683-020-09673-9}

\end{document}


