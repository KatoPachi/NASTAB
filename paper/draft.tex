
\documentclass[ review  , 3p ]{elsarticle}
%default = preprint (single sapce), review = doublespace
%detail class option: https://www.elsevier.com/__data/assets/pdf_file/0008/56843/elsdoc-1.pdf

% eliminate "Preprinted to Elsevier"
\makeatletter
\def\ps@pprintTitle{%
 \let\@oddhead\@empty
 \let\@evenhead\@empty
 \def\@oddfoot{\centerline{\thepage}}%
 \let\@evenfoot\@oddfoot}
\makeatother

%%% Begin My package additions %%%%%%%%%%%%%%%%%%%
\usepackage[hyphens]{url}



\usepackage{lineno} % add
\providecommand{\tightlist}{%
  \setlength{\itemsep}{0pt}\setlength{\parskip}{0pt}}

\usepackage{graphicx}
\usepackage{booktabs} % book-quality tables

\usepackage{zxjatype}
\usepackage{xeCJK}
\setCJKmainfont{ipaexm.ttf}
\setCJKsansfont{ipaexg.ttf}
\setCJKmonofont{ipaexg.ttf}

\usepackage{threeparttable}
\usepackage{color}

%\usepackage{xpatch}
%\xpatchcmd{\MaketitleBox}{\hrule}{}{}{}% remove first horizontal rule (above abstract)
%\xpatchcmd{\MaketitleBox}{\hrule}{}{}{}% remoce second horizonral rule (below keywords)
%%%%%%%%%%%%%%%% end my additions to header

\usepackage[T1]{fontenc}
\usepackage{lmodern}
\usepackage{amssymb,amsmath}
\usepackage{ifxetex,ifluatex}
\usepackage{fixltx2e} % provides \textsubscript
% use upquote if available, for straight quotes in verbatim environments
\IfFileExists{upquote.sty}{\usepackage{upquote}}{}
\ifnum 0\ifxetex 1\fi\ifluatex 1\fi=0 % if pdftex
  \usepackage[utf8]{inputenc}
\else % if luatex or xelatex
  \usepackage{fontspec}
  \ifxetex
    \usepackage{xltxtra,xunicode}
  \fi
  \defaultfontfeatures{Mapping=tex-text,Scale=MatchLowercase}
  \newcommand{\euro}{€}
\fi
% use microtype if available
\IfFileExists{microtype.sty}{\usepackage{microtype}}{}
\bibliographystyle{elsarticle-harvard}
\usepackage{longtable}
\usepackage{tabularx}
\ifxetex
  \usepackage[setpagesize=false, % page size defined by xetex
              unicode=false, % unicode breaks when used with xetex
              xetex]{hyperref}
\else
  \usepackage[unicode=true]{hyperref}
\fi
\hypersetup{breaklinks=true,
            bookmarks=true,
            pdfauthor={},
            pdftitle={Short Paper},
            colorlinks=false,
            urlcolor=blue,
            linkcolor=magenta,
            pdfborder={0 0 0}}
\urlstyle{same}  % don't use monospace font for urls

\setcounter{secnumdepth}{5}

\newlength{\cslhangindent}
\setlength{\cslhangindent}{1.5em}
\newlength{\csllabelwidth}
\setlength{\csllabelwidth}{3em}
\newenvironment{CSLReferences}[3] % #1 hanging-ident, #2 entry spacing
 {% don't indent paragraphs
  \setlength{\parindent}{0pt}
  % turn on hanging indent if param 1 is 1
  \ifodd #1 \everypar{\setlength{\hangindent}{\cslhangindent}}\ignorespaces\fi
  % set entry spacing
  \ifnum #2 > 0
  \setlength{\parskip}{#2\baselineskip}
  \fi
 }%
 {}
\usepackage{calc} % for \widthof, \maxof
\newcommand{\CSLBlock}[1]{#1\hfill\break}
\newcommand{\CSLLeftMargin}[1]{\parbox[t]{\maxof{\widthof{#1}}{\csllabelwidth}}{#1}}
\newcommand{\CSLRightInline}[1]{\parbox[t]{\linewidth}{#1}}
\newcommand{\CSLIndent}[1]{\hspace{\cslhangindent}#1}

% Pandoc toggle for numbering sections (defaults to be off)


% Pandoc header


\begin{document}
  \begin{frontmatter}

    \title{Charitable Giving, Tax Reform, and Government Efficiency\tnoteref{1}}
            \tnotetext[1]{This research is base on}
                \author[Osaka University]{
      Hiroki Kato 
       \corref{*} }
     \ead{vge008kh@stundent.econ.osaka-u.ac.jp}   %to avoid auto-link, use \@ instead of @
        \author[Chiba University]{
      Tsuyoshi Goto 
      }
      %to avoid auto-link, use \@ instead of @
        \author[Kobe University]{
      Yong-Rok Kim 
      }
      %to avoid auto-link, use \@ instead of @
            \address[Osaka University]{Graduate School of Economics, Osaka University, Japan}
        \address[Chiba University]{Graduate School of Economics, Chiba University, Japan}
        \address[Kobe University]{Graduate School of Economics, Kobe University, Japan}
            \cortext[*]{Corresponding Author.}
      
        \begin{abstract}
      Brah
    \end{abstract}
      
        \begin{keyword}
      Charitable giving, Giving price, Tax reform, Governement efficiency, South Korea
       \JEL{D91, I10, I18} 
    \end{keyword}
    
  \end{frontmatter}

  \hypertarget{introduction}{%
  \section{Introduction}\label{introduction}}
  
  \hypertarget{charitable-giving-and-taxiation}{%
  \subsection{Charitable Giving and Taxiation}\label{charitable-giving-and-taxiation}}
  
  In many countries, governments set a tax relief for charitable giving. This is because, if subsidizing charitable giving induces a large increase in donations, it is desirable for public good provision. To evaluate the effect of tax relief, many papers investigate the elasticity of charitable donations with respect to their tax price \citep{Randolph1995, Auten2002, Fack2010, Bakija2011, Almunia2020}. Focusing on the tax deduction on the charity, they show that the price elasticity of giving is about -1 or more in terms of absolute value, which means that the tax relief for the charitable giving is good in the sense that 1\% tax relief derives more than 1\% donation.
  
  However, if the government can provide public good more efficiently than the direct donation, the donation may not be preferable because the public good provision via donation would be costly then.
  Moreover, when the government is much more efficient than charities, people may not donate so much even if they have a warm-glow preference. \cite{Saez2004} suggests that the change of the relative price between public good provision by donation and government will change the behavior of people and the price elasticity of donation.
  However, the evaluation about the efficiency of the government is usually subjective and different for people. If someone regard the government as efficient, the perceived relative price of giving would be high for them. Thus, the giving behavior would be affected by the subjective perception towards the government.
  
  Considering these points, this paper investigates (1) the price elasticity of giving and (2) whether the different perception towards the government cause the different giving behavior using South Korean panel data.
  Our first main concern is the price elasticity of charity. South Korea (Korea hereafter) experienced the tax reform in 2014, from when the tax relief on charitable giving was conducted by tax credit, though tax deduction had been used before 2014. Thus, we exploit this tax reform as an exogenous policy change to derive the price elasticity of giving. Since the extant research focus on the tax reform within the scheme of tax deduction, this paper firstly deals with the tax reform from tax deduction system to tax credit system. Moreover, tax credit is less affected by the manipulation of tax price than tax deduction, since the income tax rate can be manipulated depending on the amount of deduction in tax deduction scheme though it cannot in tax credit scheme. This would enable us to conduct less biased estimates.
  Our result classifies that the price elasticity of giving in Korea is -1.07 \textasciitilde{} -1.26, which is within the range of the extant research.
  
  Our second concern is the relationship between the giving behavior and the perception towards the government. As we explained, people feeling administrative inefficiency would consider the direct donation is more efficient and would have more willingness to donate. Using the Korean field data, we investigate this and show that the amount of donation is not different between those who regard government as inefficient and the others, though the giving price elasticity of the former is more elastic than the latter. This means that those who think of government as inefficient have more willingness to donate for 1\% reduction of giving price.
  
  This paper contributes two strands of charitable giving literature: the elasticity of charitable donations with respect to their tax price and the perception of government's inefficiency. The examples of papers in the first strand are \cite{Randolph1995, Auten2002, Fack2010, Bakija2011, Almunia2020}. They typically use the tax return data, the main part of which is the data about wealthy people. Since our data is based on survey, which represents the income distribution of population, we believe that we can estimate the giving price elasticity of population more precisely. Moreover, we use the data of Korea, a non-Western country, which the extant research did not examine.
  
  In the second strand, there are some experimental studies and papers considering the tax evasion. Using an experiment, \cite{Li2011} compare people's willingness to give money for private charities and government agencies both of whose missions are the same. They show that people tend to donate for private charities more than government agency though they do not directly investigate the relationship between people's perception toward the government and giving behavior. \cite{Sheremeta2020} show that people increase the voluntary public good provision when they face the wasteful government spending in the experimental setting. Although the government in their setting does not provide public good, they suggest that the willingness for donation may increase if people perceive the inefficiency of government. In the tax evasion literature, several paper suggests the perceived inefficiency of government reduce tax morale \citep{Frey2007, Hammar2009, Anderson2017}. We contribute on this literature by showing the relation between the perception of government efficiency and the giving behavior.
  
  \hypertarget{charitable-giving-and-taxiation-1}{%
  \subsection{Charitable Giving and Taxiation}\label{charitable-giving-and-taxiation-1}}
  
  In addition to the tax price charitable donations, the donations may be affected by people's perception towards the government.
  
  This is because the works and missions of private charity often mirror or overlap with one of governments, and the charity is not needed if the government adequately satisfies the needs of society.
  
  Thus, the different perception towards the government may make the different behavior for charitable giving.
  
  We investigate the relation between tax price elasticity of charitable donations and the different perception towards the government using the South Korean dataset.
  
  \hypertarget{south-korean-tax-reform}{%
  \subsection{South Korean tax reform}\label{south-korean-tax-reform}}
  
  We can utilize the effect of the 2014 tax reform in the South Korea.
  
  \begin{itemize}
  \tightlist
  \item
    Before 2014, tax deduction was adopted to subsidize charitable donation behavior.
  \item
    After 2014, tax credit have been adopted.
  \end{itemize}
  
  The main difference is that tax credits reduce taxes directly, while tax deductions indirectly lower the tax burden by decreasing the marginal tax rate, which increases with gross income.
  
  In addition, the dataset contains the information about perception towards the government.
  
  \hypertarget{related-literature}{%
  \subsection{Related Literature}\label{related-literature}}
  
  This study mainly relates to the two strands of studies.
  
  \begin{enumerate}
  \def\labelenumi{\arabic{enumi}.}
  \tightlist
  \item
    Research about tax price elasticity of charitable donations
  \item
    Research about perception towards the government and donation/tax payment.
  \end{enumerate}
  
  \hypertarget{research-about-tax-price-elasticity-of-charitable-donations}{%
  \subsection{Research about tax price elasticity of charitable donations}\label{research-about-tax-price-elasticity-of-charitable-donations}}
  
  Papers in this strand examines the price and income elasticity of charitable donations using the tax deduction applied for donation.
  The estimated price elasticities vary, but the typical one is said as -1 (Andreoni and Payne, 2013).
  
  \begin{itemize}
  \tightlist
  \item
    Auten et al.(2002): -0.79\textasciitilde-1.26 (the U.S.)
  \item
    Fack and Landais(2010): -0.15\textasciitilde-0.57 (France)
  \item
    Bakija and Heim (2011): -0.61\textasciitilde-1.1 (the U.S.)
  \item
    Duquette (2016): -2.15\textasciitilde-5.01 (the U.S.)
  \item
    Almunia et al.(2020): -0.24\textasciitilde--1.5 (the U.K.)
  \end{itemize}
  
  The study in non-Western country, where the culture of donation may be different, is few.
  Thus, we firstly examine the elasticity of giving in Korea.
  
  \hypertarget{research-about-perception-towards-the-government-and-donationtax-payment.}{%
  \subsection{Research about perception towards the government and donation/tax payment.}\label{research-about-perception-towards-the-government-and-donationtax-payment.}}
  
  Experimental studies show that the giving behavior may be affected by perception towards the government.
  
  \begin{itemize}
  \tightlist
  \item
    Li et al.(2011) suggest that governmental organizations collect less donation than private charities though they have the same mission and work.
  \item
    Sheremeta and Uler(2020) show that individuals provide public good reacting the wasteful spending of government.
  \end{itemize}
  
  This may be because people with distrust in government think that
  
  \begin{enumerate}
  \def\labelenumi{\arabic{enumi}.}
  \tightlist
  \item
    the direct donation is more efficient than public service provision or
  \item
    people can directly allocate and control their funds by donation, unlike public service provision.
  \end{enumerate}
  
  Thus, people having the different trust in the government would have different elasticities of giving.
  
  \clearpage
  
  \hypertarget{references}{%
  \subsection*{References}\label{references}}
  \addcontentsline{toc}{subsection}{References}

\end{document}


