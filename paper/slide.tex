% Options for packages loaded elsewhere
\PassOptionsToPackage{unicode}{hyperref}
\PassOptionsToPackage{hyphens}{url}
%
\documentclass[
  ignorenonframetext,
  aspectratio=169,
]{beamer}
\usepackage{pgfpages}
\setbeamertemplate{caption}[numbered]
\setbeamertemplate{caption label separator}{: }
\setbeamertemplate{navigation symbols}{}
\setbeamertemplate{footline}[page number]
\setbeamertemplate{itemize item}{\small\raise0.5pt\hbox{$\bullet$}}
\setbeamertemplate{itemize subitem}{\tiny\raise1.5pt\hbox{$\bullet$}}
\setbeamertemplate{itemize subsubitem}{\tiny\raise1.5pt\hbox{$\bullet$}}
\setbeamercolor{caption name}{fg=normal text.fg}
\beamertemplatenavigationsymbolsempty
% Prevent slide breaks in the middle of a paragraph
\widowpenalties 1 10000
\raggedbottom
\setbeamertemplate{part page}{
  \centering
  \begin{beamercolorbox}[sep=16pt,center]{part title}
    \usebeamerfont{part title}\insertpart\par
  \end{beamercolorbox}
}
\setbeamertemplate{section page}{
  \centering
  \begin{beamercolorbox}[sep=12pt,center]{part title}
    \usebeamerfont{section title}\insertsection\par
  \end{beamercolorbox}
}
\setbeamertemplate{subsection page}{
  \centering
  \begin{beamercolorbox}[sep=8pt,center]{part title}
    \usebeamerfont{subsection title}\insertsubsection\par
  \end{beamercolorbox}
}
\AtBeginPart{
  \frame{\partpage}
}
\AtBeginSection{
  \ifbibliography
  \else
    \frame{\sectionpage}
  \fi
}
\AtBeginSubsection{
  \frame{\subsectionpage}
}
\usepackage{amsmath,amssymb}
\usepackage{lmodern}
\usepackage{iftex}
\ifPDFTeX
  \usepackage[T1]{fontenc}
  \usepackage[utf8]{inputenc}
  \usepackage{textcomp} % provide euro and other symbols
\else % if luatex or xetex
  \ifXeTeX
    \usepackage{xltxtra} 
    \usepackage{xeCJK}
    \setCJKmainfont{ipaexm.ttf}
    \setCJKsansfont{ipaexg.ttf}
    \setCJKmonofont{ipaexg.ttf}
  \fi
  \usepackage{unicode-math}
  \defaultfontfeatures{Scale=MatchLowercase}
  \defaultfontfeatures[\rmfamily]{Ligatures=TeX,Scale=1}
\fi
% Use upquote if available, for straight quotes in verbatim environments
\IfFileExists{upquote.sty}{\usepackage{upquote}}{}
\IfFileExists{microtype.sty}{% use microtype if available
  \usepackage[]{microtype}
  \UseMicrotypeSet[protrusion]{basicmath} % disable protrusion for tt fonts
}{}
\makeatletter
\@ifundefined{KOMAClassName}{% if non-KOMA class
  \IfFileExists{parskip.sty}{%
    \usepackage{parskip}
  }{% else
    \setlength{\parindent}{0pt}
    \setlength{\parskip}{6pt plus 2pt minus 1pt}}
}{% if KOMA class
  \KOMAoptions{parskip=half}}
\makeatother
\usepackage{xcolor}
\IfFileExists{xurl.sty}{\usepackage{xurl}}{} % add URL line breaks if available
\IfFileExists{bookmark.sty}{\usepackage{bookmark}}{\usepackage{hyperref}}
\hypersetup{
  pdftitle={Estimating Effect of Tax Incentives on Charitable Giving Considering Self-Selection of Tax Relief in South Korea},
  hidelinks,
  pdfcreator={LaTeX via pandoc}}
\urlstyle{same} % disable monospaced font for URLs

\usepackage{setspace}
\usepackage{float}

\newif\ifbibliography
\usepackage{longtable,booktabs,array}
\usepackage{threeparttable, threeparttablex, multirow}
\usepackage{calc} % for calculating minipage widths
\usepackage{caption}
% Make caption package work with longtable
\makeatletter
\def\fnum@table{\tablename~\thetable}
\makeatother
\usepackage{graphicx}
\makeatletter
\def\maxwidth{\ifdim\Gin@nat@width>\linewidth\linewidth\else\Gin@nat@width\fi}
\def\maxheight{\ifdim\Gin@nat@height>\textheight\textheight\else\Gin@nat@height\fi}
\makeatother
% Scale images if necessary, so that they will not overflow the page
% margins by default, and it is still possible to overwrite the defaults
% using explicit options in \includegraphics[width, height, ...]{}
\setkeys{Gin}{width=\maxwidth,height=\maxheight,keepaspectratio}
% Set default figure placement to htbp
\makeatletter
\def\fps@figure{htbp}
\makeatother
\setlength{\emergencystretch}{3em} % prevent overfull lines
\providecommand{\tightlist}{%
  \setlength{\itemsep}{0pt}\setlength{\parskip}{0pt}}
\setcounter{secnumdepth}{-\maxdimen} % remove section numbering
\newlength{\cslhangindent}
\setlength{\cslhangindent}{1.5em}
\newlength{\csllabelwidth}
\setlength{\csllabelwidth}{3em}
\newlength{\cslentryspacingunit} % times entry-spacing
\setlength{\cslentryspacingunit}{\parskip}
\newenvironment{CSLReferences}[2] % #1 hanging-ident, #2 entry spacing
 {% don't indent paragraphs
  \setlength{\parindent}{0pt}
  % turn on hanging indent if param 1 is 1
  \ifodd #1
  \let\oldpar\par
  \def\par{\hangindent=\cslhangindent\oldpar}
  \fi
  % set entry spacing
  \setlength{\parskip}{#2\cslentryspacingunit}
 }%
 {}
\usepackage{calc}
\newcommand{\CSLBlock}[1]{#1\hfill\break}
\newcommand{\CSLLeftMargin}[1]{\parbox[t]{\csllabelwidth}{#1}}
\newcommand{\CSLRightInline}[1]{\parbox[t]{\linewidth - \csllabelwidth}{#1}\break}
\newcommand{\CSLIndent}[1]{\hspace{\cslhangindent}#1}


\ifLuaTeX
  \usepackage{selnolig}  % disable illegal ligatures
\fi

\title{Estimating Effect of Tax Incentives on Charitable Giving Considering Self-Selection of Tax Relief in South Korea  }
\author[shortname]{ Hiroki Kato \inst{1} \and  Tsuyoshi Goto \inst{2} \and  Yong-Rok Kim \inst{3} \and }
\institute[shortinst]{ \inst{1} Osaka University \and  \inst{2} Chiba University \and  \inst{3} Kansai University \and }

\date{2021/12/23}


\begin{document}
\frame{\titlepage}

\begin{frame}{Introduction}
\protect\hypertarget{introduction}{}
\begin{itemize}
\tightlist
\item
  In many countries, tax relief for charitable giving are implemented.
\item
  The elasticity of giving tax relief is known as a key parameter to evaluate the welfare implication (Saez, 2004).

  \begin{itemize}
  \tightlist
  \item
    Intuitively, if the elasticity is more than 1 in absolute value, \$1 of tax relief make more than \$1 of charitable giving.
  \end{itemize}
\item
  Many papers investigate the elasticity based on tax return data (Almunia et al., 2020; Auten et al., 2002).
\end{itemize}
\end{frame}

\begin{frame}{Introduction}
\protect\hypertarget{introduction-1}{}
\begin{itemize}
\tightlist
\item
  However, the tax return data record only the declared charitable giving.

  \begin{itemize}
  \tightlist
  \item
    First issue: \textbf{Actual donations is different from declared donations.} (Fack and Landais, 2016; Gillitzer and Skov, 2018)
  \item
    We use panel survey data in South Korea to deal with this issue.
  \end{itemize}
\item
  Tax payers decide the amount of donation and whether to declare tax relief based on the size of tax incentive and declaration cost.

  \begin{itemize}
  \tightlist
  \item
    Second issue: Neglect of this declaration cost may bias the estimations of elasticity.
  \item
    We use instrumental variable (IV) and control function approach for this issue.
  \end{itemize}
\item
  Based on DID as an identification strategy, we investigate the giving price elasticity of South Korea.
\end{itemize}
\end{frame}

\begin{frame}{Introduction}
\protect\hypertarget{introduction-2}{}
Result

\begin{enumerate}
\tightlist
\item
  Baseline results show that the giving price elasticity is less than -1.4 in terms of intensive margins and less than -1.7 in terms of extensive margins in Korea.
\item
  The estimated giving price elasticity for those who declare charitable giving is around -1.2 -1.6.

  \begin{itemize}
  \tightlist
  \item
    These estimates are more elastic than the estimates in the extant research, many of which show around -1.
  \end{itemize}
\item
  By reducing application cost, we can increase charitable giving.
\item
  Given our estimates, increasing the subsidy on charitable giving will be desirable in Korea.
\end{enumerate}
\end{frame}

\hypertarget{conceptual-framework}{%
\section{Conceptual Framework}\label{conceptual-framework}}

\begin{frame}{Optimization Problem}
\protect\hypertarget{optimization-problem}{}
Following Almunia et al. (2020),
consider allocation problem between private consumption (\(x_{it}\)) and charitable giving (\(g_{it}\))

\begin{align}
  \max_{x_{it}, g_{it}, R_{it}} &U(x_{it}, g_{it}, G_t)
  = u_i(x_{it}, g_{it}, G_t) - R_{it}K(Z_{it}), \\
  \text{s.t.}\:\:
  &x_{it} + g_{it}
  = y_{it} - R_{it} T_{it}(y_{it}, g_{it}) - (1 - R_{it}) T_{it}(y_{it}), \\
  &G_t = g_{it} + G_{-it},
\end{align}

where \(y_{it}\) is pre-tax total income, \(R_{it}\) is a dummy of declaration of tax relief and \(T_{it}(y_{it})\) and \(T_{it}(y_{it}, g_{it})\) are respectively the amount of tax when \(i\) does not declare tax relief and when \(i\) declares tax relief in year \(t\). \(G_{-it}\) is public goods supplied by others. \(K(Z_{it})\) is application cost which is a function of instrument \(Z_{it}\).
\end{frame}

\begin{frame}{Remarks on Optimization Problem}
\protect\hypertarget{remarks-on-optimization-problem}{}
We assume

\begin{itemize}
\tightlist
\item
  No saving
\item
  \(G_{-it}\) is large enough to \(\frac{\partial u_i}{\partial G}(x, g, G) \approx 0\)
\end{itemize}

Given \(R_{it}\), optimal level of donations sloves

\begin{align}
  \max_{g_{it}} u_i(
    y_{it} - R_{it} T_{it}(y_{it}, g_{it}) - (1 - R_{it}) T_{it}(y_{it}) - g_{it}, g_{it}, g_{it} + G_{-it}
  ).
\end{align}

\begin{itemize}
\tightlist
\item
  We can ignore application cost \(K(Z_{it})\) when solving optimal giving level because the application cost does not depend on \(g_{it}\)
\end{itemize}
\end{frame}

\begin{frame}{First-Order Condition}
\protect\hypertarget{first-order-condition}{}
\begin{align}
  - \frac{\partial u_i}{\partial x_{it}}
  \left( R_{it} \frac{\partial T_{it}}{\partial g_{it}}(y_{it}, g_{it}) + 1 \right)
  + \frac{\partial u_i}{\partial g_{it}} = 0
\end{align}

\begin{itemize}
\tightlist
\item
  \(\partial T_{it} / \partial g_{it} < 0\) is tax incentive of charitable giving.

  \begin{itemize}
  \tightlist
  \item
    Let \(s_{it} \equiv |\partial T_{it} / \partial g_{it}|\) be size of tax incentive.
  \item
    Relative giving price is \(1 - s_{it}\)
  \item
    As we explain later, there is \emph{within} variation of \(s_{it}\) due to tax reform.
  \end{itemize}
\end{itemize}

Define \(g_i(1 - s_{it}, y_{it})\) and \(g_i(1, y_{it})\) to be the optimal levels of donations (potential outcomes) for choices \(R_{it} = 1, 0\) respectively.
\end{frame}

\begin{frame}{Self-Selection of Tax Relief}
\protect\hypertarget{self-selection-of-tax-relief}{}
We can write indirect utility as

\begin{align}
  &v_i(1 - s_{it}, y_{it}, G_{-it}) - K(Z_{it}),  \\
  &v_i(1, y_{it}, G_{-it}).
\end{align}

Thus, individual \(i\) applies for tax relief in year \(t\),
that is, \(R_{it} = 1\) iff

\begin{align}
  \Delta v_{it} \equiv
  v_i(1 - s_{it}, y_{it}, G_{-it}) - v_i(1, y_{it}, G_{-it})
  \ge K(Z_{it}).
\end{align}
\end{frame}

\hypertarget{institutional-background-in-south-korea}{%
\section{Institutional Background in South Korea}\label{institutional-background-in-south-korea}}

\begin{frame}{2014 Tax Reform}
\protect\hypertarget{tax-reform}{}
Since 2014, tax relief of charitable giving has changed from \textbf{income deduction (所得控除)} to \textbf{tax credit (税額控除)}.

\begin{itemize}
\tightlist
\item
  Income deductions are more advantageous for high-income groups than low-income groups because the higher the income, the greater the decrease in tax burden.
\item
  The 2014 tax reform aimed to alleviate the regressiveness of taxes and improve the equilibrium of taxation by changing from income deductions to tax credits.
\item
  We exploit this reform as a main source of variation for identification

  \begin{itemize}
  \tightlist
  \item
    Difference-in-Difference
  \end{itemize}
\end{itemize}
\end{frame}

\begin{frame}{2014 Tax Reform}
\protect\hypertarget{tax-reform-1}{}
\textbf{Tax deduction system (until 2013)}

\begin{align}
  T_{it}(y_{it}, g_{it}) = T_{it}(y_{it} - g_{it})
\end{align}

\begin{itemize}
\tightlist
\item
  Tax incentive is \(s_{it} = T'(y_{it} - g_{it})\)
\item
  In 2012 and 2013, the marginal tax rate was the same, though it was different from ones before 2011.
\item
  The giving price depended on income level (\(y_{it}\)) and giving level (\(g_{it}\))
\end{itemize}

\textbf{Tax credit system (from 2014)}

\begin{align}
  T_{it}(y_{it}, g_{it}) = T_{it}(y_{it}) - m g_{it}
\end{align}

\begin{itemize}
\tightlist
\item
  \(m\) is tax credit rate and is \(m = 0.15\)
\item
  Tax incentive is \(s_{it} = m\)
\end{itemize}
\end{frame}

\hypertarget{data}{%
\section{Data}\label{data}}

\begin{frame}{National Survey of Tax and Benefit (NaSTaB)}
\protect\hypertarget{national-survey-of-tax-and-benefit-nastab}{}
\begin{itemize}
\tightlist
\item
  NaSTaB has been implemented by The Korea Institute of Taxation and Finance since 2008
\item
  The NaSTaB is an annual panel data on households' tax burden and public benefits
\item
  The unit of analysis is 5,634 households throughout the country

  \begin{itemize}
  \tightlist
  \item
    5,634 family heads and family members with more than 15 years old and with income or economically active
  \end{itemize}
\item
  Our analysis uses the NaSTaB data from (i) 2013-2018 and (ii) excluding respondents under the age of 23.

  \begin{itemize}
  \tightlist
  \item
    using the NaSTaB data before 2012 captures the effects of other tax reform than the reform in 2014.
  \item
    we exclude respondents whose age is under 23 because they are not likely to have income or assets.
  \end{itemize}
\end{itemize}
\end{frame}

\begin{frame}{Descriptive Statistics}
\protect\hypertarget{descriptive-statistics}{}
\begin{table}

\caption{\label{tab:SummaryCovariate}Descriptive Statistics}
\centering
\fontsize{6}{8}\selectfont
\begin{tabular}[t]{lcccccc}
\toprule
  & N & Mean & Std.Dev. & Min & Median & Max\\
\midrule
\addlinespace[0.3em]
\multicolumn{7}{l}{\textbf{Income and giving price}}\\
\hspace{1em}Annual taxable labor income (unit: 10,000KRW) & 36249 & 1754.32 & 2702.16 & 0.00 & 0.00 & 50000.00\\
\hspace{1em}Relative first price of giving & 36258 & 0.86 & 0.04 & 0.62 & 0.85 & 0.94\\
\addlinespace[0.3em]
\multicolumn{7}{l}{\textbf{Charitable giving}}\\
\hspace{1em}Annual chariatable giving (unit: 10,000KRW) & 36259 & 35.86 & 153.36 & 0.00 & 0.00 & 10000.00\\
\hspace{1em}Dummary of donation > 0 & 36259 & 0.24 & 0.43 & 0.00 & 0.00 & 1.00\\
\hspace{1em}Dummy of declaration of a tax relief & 36259 & 0.10 & 0.30 & 0.00 & 0.00 & 1.00\\
\addlinespace[0.3em]
\multicolumn{7}{l}{\textbf{Individual Characteristics}}\\
\hspace{1em}Age & 36259 & 53.43 & 16.21 & 24.00 & 51.00 & 103.00\\
\hspace{1em}Female dummy & 36259 & 0.43 & 0.50 & 0.00 & 0.00 & 1.00\\
\hspace{1em}University graduate & 36258 & 0.42 & 0.49 & 0.00 & 0.00 & 1.00\\
\hspace{1em}High school graduate dummy & 36258 & 0.31 & 0.46 & 0.00 & 0.00 & 1.00\\
\hspace{1em}Junior high school graduate dummy & 36258 & 0.27 & 0.44 & 0.00 & 0.00 & 1.00\\
\hspace{1em}Wage earner dummy & 27453 & 0.56 & 0.50 & 0.00 & 1.00 & 1.00\\
\hspace{1em}\#.Tax accountant / population & 36259 & 1.04 & 0.51 & 0.32 & 0.92 & 2.24\\
\bottomrule
\end{tabular}
\end{table}
\end{frame}

\begin{frame}{Right-Skewed Income Distribution and Price Variation for Identification}
\protect\hypertarget{right-skewed-income-distribution-and-price-variation-for-identification}{}
\begin{figure}[t]

{\centering \includegraphics[width=0.7\linewidth,]{C:/Users/katoo/Desktop/NASTAB/paper/slide_files/figure-beamer/SummaryPrice-1} 

}

\caption{Income Distribution in 2013 and Relative Giving Price. Notes: The left and right axis measure the relative frequency of respondents (grey bars) and the relative giving price (solid step line and dashed line), respectively. A solid step line and a dashed horizontal line represents the giving price in 2013 and 2014, respectively.}\label{fig:SummaryPrice}
\end{figure}
\end{frame}

\begin{frame}{Message from Figure \ref{fig:SummaryPrice}: Right-Skewed Distribution}
\protect\hypertarget{message-from-figure-reffigsummaryprice-right-skewed-distribution}{}
\begin{itemize}
\tightlist
\item
  NaSTaB contains the annual taxable labor income last year
\item
  Our sample includes subjects with no labor income (e.g.~housewives)

  \begin{itemize}
  \tightlist
  \item
    Table \ref{tab:SummaryCovariate}: the average income is 17.54 million KRW
  \end{itemize}
\item
  National Tax Statistical Yearbook 2012-2018 (Korean National Tax Service): the average annual taxable income is 32.77 million KRW

  \begin{itemize}
  \tightlist
  \item
    Sample: employees who submitted the tax return
  \end{itemize}
\end{itemize}
\end{frame}

\begin{frame}{Message from Figure \ref{fig:SummaryPrice}: Price Variation}
\protect\hypertarget{message-from-figure-reffigsummaryprice-price-variation}{}
Based on changes in tax incentive due to the 2014 tax reform,
we can devide into three income groups:

\begin{enumerate}
\tightlist
\item
  less than 120 million KRW

  \begin{itemize}
  \tightlist
  \item
    2014 tax reform has expanded tax incentive (decreased giving price)
  \end{itemize}
\item
  between 120 million KRW and 460 million KRW

  \begin{itemize}
  \tightlist
  \item
    2014 tax reform has unchanged tax incentive
  \end{itemize}
\item
  more than 460 million KRW.

  \begin{itemize}
  \tightlist
  \item
    2014 tax reform has decreased tax incentive (increaed giving price)
  \end{itemize}
\end{enumerate}

This is main source of identification for effect of tax incentive on giving.
\end{frame}

\begin{frame}{Donors Decreased Immediately After Tax Reform}
\protect\hypertarget{donors-decreased-immediately-after-tax-reform}{}
\begin{figure}[t]

{\centering \includegraphics[width=0.7\linewidth,]{C:/Users/katoo/Desktop/NASTAB/paper/slide_files/figure-beamer/SummaryGiving-1} 

}

\caption{Proportion of Donors and Average Donations among Donors. Notes: The left and right axises measure prooortion of donors (grey bars) and the average amount of donations among donors (solid line), respectively.}\label{fig:SummaryGiving}
\end{figure}
\end{frame}

\begin{frame}{Message from Figure \ref{fig:SummaryGiving}}
\protect\hypertarget{message-from-figure-reffigsummarygiving}{}
\begin{itemize}
\tightlist
\item
  Proportion of donors across years: 24\% (Table \ref{tab:SummaryCovariate})
\item
  2014: Proportion of donors is lower than before the tax reform

  \begin{itemize}
  \tightlist
  \item
    After that, proportion of donors has continued to increase, finally surpassing that before the tax reform
  \end{itemize}
\end{itemize}

Notes:

\begin{itemize}
\tightlist
\item
  average donation conditional on donors has been stable across year

  \begin{itemize}
  \tightlist
  \item
    1.5 million KRW (7\% of average income)
  \item
    Table \ref{tab:SummaryCovariate}: Unconditional average donation is 358,600 KRW (2\% of average income)
  \end{itemize}
\end{itemize}
\end{frame}

\begin{frame}{Price Effect Can Be Observed}
\protect\hypertarget{price-effect-can-be-observed}{}
\begin{figure}[t]

{\centering \includegraphics[width=0.7\linewidth,]{C:/Users/katoo/Desktop/NASTAB/paper/slide_files/figure-beamer/SummaryGivingOverall-1} 

}

\caption{Average Logged Giving by Three Income Groups. Notes: We created three income groups, with the relative price of giving rising (circle), unchanged (triangle), and falling (square) between 2013 and 2014. The group averages are normalized to be zero in 2013.}\label{fig:SummaryGivingOverall}
\end{figure}
\end{frame}

\begin{frame}{Message from Figure \ref{fig:SummaryGivingOverall}}
\protect\hypertarget{message-from-figure-reffigsummarygivingoverall}{}
\begin{itemize}
\tightlist
\item
  Price effect = Tax incentive increases charitable giving

  \begin{itemize}
  \tightlist
  \item
    Donations for income groups with unchanged or increased tax incentives have exceeded those in 2013 since 2015
  \item
    Donations for income groups with reduced tax incentives have been lower than before tax reform since 2015
  \end{itemize}
\end{itemize}

Notes:

\begin{enumerate}
\tightlist
\item
  Prior to the 2014 tax reform, donations did not change in all groups
\item
  Donations for all groups were lower than in 2013

  \begin{itemize}
  \tightlist
  \item
    Donations for income groups with reduced tax incentive due to the 2014 tax reform was 40\% of that in 2013
  \item
    Announcement effect? Learning effect (瀧井先生のコメント)?
  \end{itemize}
\end{enumerate}
\end{frame}

\begin{frame}{Price Effect Can Be Partially Observed for Intensive Margin}
\protect\hypertarget{price-effect-can-be-partially-observed-for-intensive-margin}{}
\begin{figure}[t]

{\centering \includegraphics[width=0.7\linewidth,]{C:/Users/katoo/Desktop/NASTAB/paper/slide_files/figure-beamer/SummaryGivingIntensive-1} 

}

\caption{Average Logged Giving by Three Income Groups Conditional on Donors. Notes: We created three income groups, with the relative price of giving rising (circle), unchanged (triangle), and falling (square) between 2013 and 2014. The group averages are normalized to be zero in 2013.}\label{fig:SummaryGivingIntensive}
\end{figure}
\end{frame}

\begin{frame}{Price Effect Can Be Observed for Extensive Margin}
\protect\hypertarget{price-effect-can-be-observed-for-extensive-margin}{}
\begin{figure}[t]

{\centering \includegraphics[width=0.7\linewidth,]{C:/Users/katoo/Desktop/NASTAB/paper/slide_files/figure-beamer/SummaryGivingExtensive-1} 

}

\caption{Proportion of Donors by Three Income Groups. Notes: We created three income groups, with the relative price of giving rising (circle), unchanged (triangle), and falling (square) between 2013 and 2014. The group averages are normalized to be zero in 2013.}\label{fig:SummaryGivingExtensive}
\end{figure}
\end{frame}

\begin{frame}{Message from Figure \ref{fig:SummaryGivingIntensive} and \ref{fig:SummaryGivingExtensive}}
\protect\hypertarget{message-from-figure-reffigsummarygivingintensive-and-reffigsummarygivingextensive}{}
\begin{itemize}
\tightlist
\item
  Intensive margin (How much donors give): Price effect can be partially observed

  \begin{itemize}
  \tightlist
  \item
    tax incentiveが変化しない所得層が寄付を最も増やしていたが、tax incentiveが減少した所得層はあまり寄付を増やさなかった
  \item
    tax incentiveが変化しない所得層の方がtax incentiveが拡充された所得層よりも寄付を増やしていた(price effectではない)
  \end{itemize}
\item
  Extensive margin (Whethre respondents donate): Price effect can be observed

  \begin{itemize}
  \tightlist
  \item
    全体の寄付の増加率のグラフ(Figure \ref{fig:SummaryGivingOverall})と同じ傾向
  \end{itemize}
\end{itemize}
\end{frame}

\begin{frame}{Application for Tax Relief Decreased After Tax Reform}
\protect\hypertarget{application-for-tax-relief-decreased-after-tax-reform}{}
\begin{figure}[t]

{\centering \includegraphics[width=0.7\linewidth,]{C:/Users/katoo/Desktop/NASTAB/paper/slide_files/figure-beamer/SummaryReliefbyIncome-1} 

}

\caption{Proportion of Having Applied for Tax Relief by Three Income Groups. Notes: We created three income groups, with the relative price of giving rising (circle), unchanged (triangle), and falling (square) between 2013 and 2014.}\label{fig:SummaryReliefbyIncome}
\end{figure}
\end{frame}

\begin{frame}{Message from Figure \ref{fig:SummaryReliefbyIncome}}
\protect\hypertarget{message-from-figure-reffigsummaryreliefbyincome}{}
\begin{enumerate}
\tightlist
\item
  2014年税制改革以降、寄付控除を申請した人の割合は減少した

  \begin{itemize}
  \tightlist
  \item
    全体を通した寄付控除の申請比率:10\% (Table \ref{tab:SummaryCovariate})
  \item
    tax incentiveが減少した所得層の寄付控除の申請比率が減少した
  \end{itemize}
\item
  tax incentiveが増加した所得層の寄付者の割合が増えたにも関わらず、寄付控除の申請比率は伸びていない

  \begin{itemize}
  \tightlist
  \item
    寄付控除を申告することで得するはずなのに、それをしていない人がいる \(\to\) 申請コストが存在する
  \end{itemize}
\end{enumerate}
\end{frame}

\begin{frame}{Similar Distribution of Giving Regardless of Application}
\protect\hypertarget{similar-distribution-of-giving-regardless-of-application}{}
\begin{figure}[t]

{\centering \includegraphics[width=0.7\linewidth,]{C:/Users/katoo/Desktop/NASTAB/paper/slide_files/figure-beamer/SummaryGivingIntensiveDist-1} 

}

\caption{Distribution of Charitable Giving among Those Who Donated}\label{fig:SummaryGivingIntensiveDist}
\end{figure}
\end{frame}

\begin{frame}{Message from Figure \ref{fig:SummaryGivingIntensiveDist}}
\protect\hypertarget{message-from-figure-reffigsummarygivingintensivedist}{}
\begin{itemize}
\tightlist
\item
  すべての年においても、寄付者に限定した寄付の分布は寄付控除の申請の有無に依存しない

  \begin{itemize}
  \tightlist
  \item
    寄付控除を申請していない人の寄付額の分布が申請している人よりも左側にあるならば、申請コストの程度は小さいと予想される
  \item
    分布の形状がほとんど一致しているので、申請コストの程度はかなり大きいと予想される
  \end{itemize}
\end{itemize}
\end{frame}

\begin{frame}{Wage Earners Are More Likely to Apply (1)}
\protect\hypertarget{wage-earners-are-more-likely-to-apply-1}{}
\begin{figure}[t]

{\centering \includegraphics[width=0.7\linewidth,]{C:/Users/katoo/Desktop/NASTAB/paper/slide_files/figure-beamer/SummaryReliefbyEarner-1} 

}

\caption{Share of Tax Relief by Wage Earners. Notes: A solid line is the share of applying for tax relief among wage eaners. A dashed line is the share of applying for tax relief other than wage earners.}\label{fig:SummaryReliefbyEarner}
\end{figure}
\end{frame}

\begin{frame}{Wage Earners Are More Likely to Apply (2)}
\protect\hypertarget{wage-earners-are-more-likely-to-apply-2}{}
\begin{figure}[t]

{\centering \includegraphics[width=0.7\linewidth,]{C:/Users/katoo/Desktop/NASTAB/paper/slide_files/figure-beamer/SummaryReliefbyEarner2-1} 

}

\caption{Share of Tax Relief by Wage Earners Conditional on Donors. Notes: A solid line is the share of applying for tax relief among wage eaners. A dashed line is the share of applying for tax relief other than wage earners.}\label{fig:SummaryReliefbyEarner2}
\end{figure}
\end{frame}

\begin{frame}{Message from Figure \ref{fig:SummaryReliefbyEarner} and \ref{fig:SummaryReliefbyEarner2}}
\protect\hypertarget{message-from-figure-reffigsummaryreliefbyearner-and-reffigsummaryreliefbyearner2}{}
\begin{itemize}
\tightlist
\item
  給与所得者はそうでない人よりも寄付控除を申請しやすい

  \begin{itemize}
  \tightlist
  \item
    自営業者(非給与所得者の主)は寄付の証明書を常に保管しているようなことはない(申請期限前に問い合わせが殺到する)
  \item
    給与所得者の方が申請しやすい環境にあることがデータからもわかる
  \item
    給与所得者かどうかで寄付者の比率が異なる可能性を考慮して、寄付者に限定しても同じ傾向である(Figure \ref{fig:SummaryReliefbyEarner2})
  \end{itemize}
\item
  2014年税制改革以降、寄付控除の申請比率は減少している(とくに、給与所得者)
\end{itemize}
\end{frame}

\hypertarget{first-stage-result-who-applied-for-tax-relief}{%
\section{First-Stage Result: Who Applied for Tax Relief?}\label{first-stage-result-who-applied-for-tax-relief}}

\hypertarget{estimating-conventional-price-elasticities}{%
\section{Estimating Conventional Price Elasticities}\label{estimating-conventional-price-elasticities}}

\hypertarget{control-function-approach}{%
\section{Control Function Approach}\label{control-function-approach}}

\hypertarget{welfare-implication}{%
\section{Welfare Implication}\label{welfare-implication}}

\hypertarget{conclusion}{%
\section{Conclusion}\label{conclusion}}

\hypertarget{references}{%
\section*{References}\label{references}}
\addcontentsline{toc}{section}{References}

\begin{frame}{References}
\hypertarget{refs}{}
\begin{CSLReferences}{1}{0}
\leavevmode\vadjust pre{\hypertarget{ref-Almunia2020}{}}%
Almunia, M., Guceri, I., Lockwood, B., Scharf, K., 2020. More giving or more givers? The effects of tax incentives on charitable donations in the UK. Journal of Public Economics 183. doi:\href{https://doi.org/10.1016/j.jpubeco.2019.104114}{10.1016/j.jpubeco.2019.104114}

\leavevmode\vadjust pre{\hypertarget{ref-Auten2002}{}}%
Auten, G.E., Sieg, H., Clotfelter, C.T., 2002. \href{http://www.jstor.org/stable/3083340}{Charitable giving, income, and taxes: An analysis of panel data}. American Economic Review 92, 371--382.

\leavevmode\vadjust pre{\hypertarget{ref-Fack2016}{}}%
Fack, G., Landais, C., 2016. The effect of tax enforcement on tax elasticities: Evidence from charitable contributions in france. Journal of Public Economics 133, 23--40. doi:\url{https://doi.org/10.1016/j.jpubeco.2015.10.004}

\leavevmode\vadjust pre{\hypertarget{ref-Gillitzer2018}{}}%
Gillitzer, C., Skov, P.E., 2018. {The use of third-party information reporting for tax deductions: evidence and implications from charitable deductions in Denmark}. Oxford Economic Papers 70, 892--916. doi:\href{https://doi.org/10.1093/oep/gpx055}{10.1093/oep/gpx055}

\leavevmode\vadjust pre{\hypertarget{ref-Saez2004}{}}%
Saez, E., 2004. The optimal treatment of tax expenditures. Journal of Public Economics 88, 2657--2684. doi:\href{https://doi.org/10.1016/j.jpubeco.2003.09.004}{10.1016/j.jpubeco.2003.09.004}

\end{CSLReferences}
\end{frame}

\end{document}
